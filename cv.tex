%$Id: cv.tex,v 1.230 2024/12/23 14:54:06 brandenb Exp $
%\documentstyle[11pt]{article}
\documentclass{article}
\usepackage{html,url,color}
\setlength{\oddsidemargin}{+3mm}
\setlength{\textwidth}{160mm}
\setlength{\textheight}{232mm}
\setlength{\topmargin}{-14mm}
%
% $Id: journ.tex,v 1.184 2024/12/25 14:14:29 brandenb Exp $
%
%  journals
%
\newcommand{\nor}[1]{\htmladdnormallink{http://www.nordita.org/\-pre\-prints/\-#1.ps.gz}{http://www.nordita.org/preprints/#1.ps.gz}}
\newcommand{\doiS}[1]{\html{\htmladdnormallink{(DOI)}{http://dx.doi.org/#1}}}
\newcommand{\doiB}[1]{\html{\htmladdnormallink{(DOI}{http://dx.doi.org/#1}}}
\newcommand{\doiT}[1]{\html{\htmladdnormallink{, DOI}{http://dx.doi.org/#1}}}
\newcommand{\doiE}[1]{\html{\htmladdnormallink{, DOI)}{http://dx.doi.org/#1}}}
%
\newcommand{\arxivB}[1]{\html{\htmladdnormallink{{\rm(arXiv:#1}}{http://arXiv.org/abs/#1}}}
\newcommand{\arxivS}[1]{\html{\htmladdnormallink{{\rm(arXiv:#1)}}{http://arXiv.org/abs/#1}}}
\newcommand{\arxiv}[1]{\htmladdnormallink{{\rm arXiv:#1}}{http://arXiv.org/abs/#1}}
%
%\newcommand{\arxivB}[1]{\html{\htmladdnormallink{{\rm(arXiv:#1}}{http://xxx.lanl.gov/abs/#1}}}
%\newcommand{\arxivS}[1]{\html{\htmladdnormallink{{\rm(arXiv:#1)}}{http://xxx.lanl.gov/abs/#1}}}
%\newcommand{\arxiv}[1]{\htmladdnormallink{{\rm arXiv:#1}}{http://xxx.lanl.gov/abs/#1}}
%
\newcommand{\astroph}[1]{\htmladdnormallink{{\rm astro-ph/#1}}{http://arXiv.org/abs/astro-ph/#1}}
\newcommand{\condmat}[1]{\htmladdnormallink{{\rm cond-mat/#1}}{http://arXiv.org/abs/cond-mat/#1}}
\newcommand{\physics}[1]{\htmladdnormallink{{\rm physics/#1}}{http://arXiv.org/abs/physics/#1}}
\newcommand{\hepph}[1]{\htmladdnormallink{{\rm hep-ph/#1}}{http://arXiv.org/abs/hep-ph/#1}}
\newcommand{\qbio}[1]{\htmladdnormallink{{\rm q-bio/#1}}{http://arXiv.org/abs/q-bio/#1}}
\newcommand{\astrophS}[1]{\html{\htmladdnormallink{{\rm(astro-ph/#1}}{http://arXiv.org/abs/astro-ph/#1}}}
\newcommand{\astrophB}[1]{\html{\htmladdnormallink{{\rm(astro-ph/#1}}{http://arXiv.org/abs/astro-ph/#1}}}
\newcommand{\condmatB}[1]{\html{\htmladdnormallink{{\rm(cond-mat/#1}}{http://arXiv.org/abs/cond-mat/#1}}}
\newcommand{\physicsS}[1]{\html{\htmladdnormallink{{\rm(physics/#1}}{http://arXiv.org/abs/physics/#1})}}
\newcommand{\physicsB}[1]{\html{\htmladdnormallink{{\rm(physics/#1}}{http://arXiv.org/abs/physics/#1}}}
\newcommand{\hepphB}[1]{\html{\htmladdnormallink{{\rm(hep-ph/#1}}{http://arXiv.org/abs/hep-ph/#1}}}
\newcommand{\qbioB}[1]{\html{\htmladdnormallink{{\rm(q-bio/#1}}{http://arXiv.org/abs/q-bio/#1}}}
%
%  French mirror site (:.,.+5s/harvard\.edu/u-strasbg\.fr/g)
%  French mirror site (:.,.+11s/cdsads\.u-strasbg\.fr/esoads\.eso\.org/g)
%
%\newcommand{\nadsB}[1]{\html{\htmladdnormallink{(ADS}{http://esoads.eso.org/abs/#1}}}
%\newcommand{\dadsT}[1]{\html{\htmladdnormallink{, ADS}{http://esoads.eso.org/doi/#1}}}
%\newcommand{\nadsT}[1]{\html{\htmladdnormallink{, ADS}{http://esoads.eso.org/abs/#1}}}
%\newcommand{\NadsT}[1]{\html{\htmladdnormallink{, ADS}{https://ui.adsabs.harvard.edu/abs/#1/abstract}}}
%\newcommand{\nadsE}[1]{\html{\htmladdnormallink{, ADS)}{http://esoads.eso.org/abs/#1}}}
%\newcommand{\nadsS}[1]{\html{\htmladdnormallink{(ADS)}{http://esoads.eso.org/abs/#1}}}
%
%\newcommand{\ads}[1]{\html{\htmladdnormallink{ADS}{http://esoads.eso.org/abs/#1}}}
%\newcommand{\adsB}[1]{\html{\htmladdnormallink{(ADS}{http://esoads.eso.org/abs/#1}}}
%\newcommand{\adsE}[1]{\html{\htmladdnormallink{, ADS)}{http://esoads.eso.org/abs/#1}}}
%\newcommand{\adsC}[1]{\html{\htmladdnormallink{ADS, }{http://esoads.eso.org/abs/#1}}}
%\newcommand{\adsS}[1]{\html{\htmladdnormallink{(ADS)}{http://esoads.eso.org/abs/#1}}}
%\newcommand{\adsT}[1]{\html{\htmladdnormallink{, ADS}{http://esoads.eso.org/abs/#1}}}
%
%\newcommand{\ads}[1]{\html{\htmladdnormallink{ADS}{http://esoads.eso.org/cgi-bin/nph-bib_query?bibcode=#1}}}
%\newcommand{\adsB}[1]{\html{\htmladdnormallink{(ADS}{http://esoads.eso.org/cgi-bin/nph-bib_query?bibcode=#1}}}
%\newcommand{\adsE}[1]{\html{\htmladdnormallink{, ADS)}{http://esoads.eso.org/cgi-bin/nph-bib_query?bibcode=#1}}}
%\newcommand{\adsC}[1]{\html{\htmladdnormallink{ADS, }{http://esoads.eso.org/cgi-bin/nph-bib_query?bibcode=#1}}}
%\newcommand{\adsS}[1]{\html{\htmladdnormallink{(ADS)}{http://esoads.eso.org/cgi-bin/nph-bib_query?bibcode=#1}}}
%\newcommand{\adsT}[1]{\html{\htmladdnormallink{, ADS}{http://esoads.eso.org/cgi-bin/nph-bib_query?bibcode=#1}}}
%
%  Harvard site
%
%\newcommand{\nadsB}[1]{\html{\htmladdnormallink{(ADS}{http://adsabs.harvard.edu/abs/#1}}}
%\newcommand{\dadsT}[1]{\html{\htmladdnormallink{, ADS}{http://adsabs.harvard.edu/doi/#1}}}
\newcommand{\dadsT}[1]{\html{\htmladdnormallink{, ADS}{http://ui.adsabs.harvard.edu/doi/#1}}}
%\newcommand{\nadsT}[1]{\html{\htmladdnormallink{, ADS}{http://adsabs.harvard.edu/abs/#1}}}
%\newcommand{\nadsE}[1]{\html{\htmladdnormallink{, ADS}{http://adsabs.harvard.edu/abs/#1}}}
%\newcommand{\nadsS}[1]{\html{\htmladdnormallink{(ADS)}{http://adsabs.harvard.edu/abs/#1}}}
%
%\newcommand{\ads}[1]{\html{\htmladdnormallink{ADS}{http://adsabs.harvard.edu/cgi-bin/nph-bib_query?bibcode=#1}}}
%\newcommand{\adsB}[1]{\html{\htmladdnormallink{(ADS}{http://adsabs.harvard.edu/cgi-bin/nph-bib_query?bibcode=#1}}}
%\newcommand{\adsE}[1]{\html{\htmladdnormallink{, ADS}{http://adsabs.harvard.edu/cgi-bin/nph-bib_query?bibcode=#1}}}
%\newcommand{\adsC}[1]{\html{\htmladdnormallink{ADS, }{http://adsabs.harvard.edu/cgi-bin/nph-bib_query?bibcode=#1}}}
%\newcommand{\adsS}[1]{\html{\htmladdnormallink{(ADS)}{http://adsabs.harvard.edu/cgi-bin/nph-bib_query?bibcode=#1}}}
%\newcommand{\adsT}[1]{\html{\htmladdnormallink{, ADS}{http://adsabs.harvard.edu/cgi-bin/nph-bib_query?bibcode=#1}}}
%
\newcommand{\ads}[1]{\html{\htmladdnormallink{ADS}{http://ui.adsabs.harvard.edu/abs/#1}}}
\newcommand{\adsB}[1]{\html{\htmladdnormallink{(ADS}{http://ui.adsabs.harvard.edu/abs/#1}}}
\newcommand{\adsE}[1]{\html{\htmladdnormallink{, ADS}{http://ui.adsabs.harvard.edu/abs/#1}}}
\newcommand{\adsC}[1]{\html{\htmladdnormallink{ADS, }{http://ui.adsabs.harvard.edu/abs/#1}}}
\newcommand{\adsS}[1]{\html{\htmladdnormallink{(ADS)}{http://ui.adsabs.harvard.edu/abs/#1}}}
\newcommand{\adsT}[1]{\html{\htmladdnormallink{, ADS}{http://ui.adsabs.harvard.edu/abs/#1}}}
%
\newcommand{\ownB}[1]{\html{\htmladdnormallink{(PDF}{../Own_Papers/#1.pdf}}}
\newcommand{\ownS}[1]{\html{\htmladdnormallink{(PDF)}{../Own_Papers/#1.pdf}}}
\newcommand{\ownT}[1]{\html{\htmladdnormallink{, PDF}{../Own_Papers/#1.pdf}}}
\newcommand{\ownE}[1]{\html{\htmladdnormallink{, PDF)}{../Own_Papers/#1.pdf}}}
\newcommand{\supE}[1]{\html{\htmladdnormallink{, Supp)}{../Own_Papers/#1.pdf}}}
\newcommand{\adE}[1]{\html{\htmladdnormallink{, ad)}{../Own_Papers/#1.pdf}}}
%\newcommand{\own}[1]{\html{\htmladdnormallink{PDF}{http://www.nordita.org/~brandenb/Own_Papers/#1.pdf}}}
%\newcommand{\ownS}[1]{\html{\htmladdnormallink{(PDF)}{http://www.nordita.org/~brandenb/Own_Papers/#1.pdf}}}
%\newcommand{\ownB}[1]{\html{\htmladdnormallink{(PDF}{http://www.nordita.org/~brandenb/Own_Papers/#1.pdf}}}
%\newcommand{\ownE}[1]{\html{\htmladdnormallink{, PDF)}{../Own_Papers/#1.pdf}}}
%\newcommand{\ownT}[1]{\html{\htmladdnormallink{, PDF}{http://www.nordita.org/~brandenb/Own_Papers/#1.pdf}}}
\newcommand{\psT}[1]{\html{\htmladdnormallink{, PS}{http://www.nordita.org/~brandenb/papers/#1.ps.gz}}}
\newcommand{\psB}[1]{\html{\htmladdnormallink{(PS}{http://www.nordita.org/~brandenb/papers/#1.ps.gz}}}
\newcommand{\htmlB}[1]{\html{\htmladdnormallink{(HTML}{http://www.nordita.org/~brandenb/papers/#1.html}}}
\newcommand{\htmlS}[1]{\html{\htmladdnormallink{(HTML}{http://www.nordita.org/~brandenb/papers/#1.html}}}
\newcommand{\htmlT}[1]{\html{\htmladdnormallink{, HTML}{http://www.nordita.org/~brandenb/papers/#1.html}}}
\newcommand{\htmlE}[1]{\html{\htmladdnormallink{, HTML}{http://www.nordita.org/~brandenb/papers/#1.html}}}
\newcommand{\httpB}[1]{\html{\htmladdnormallink{(HTML}{#1}}}
\newcommand{\httpS}[1]{\html{\htmladdnormallink{(HTML)}{#1}}}
\newcommand{\httpT}[1]{\html{\htmladdnormallink{, HTML}{#1}}}
\newcommand{\httpE}[1]{\html{\htmladdnormallink{, HTML)}{#1}}}

%\newcommand{\astrophADS}[2]{(\htmladdnormallink{astro-ph/#1}{http://xxx.lanl.gov/abs/astro-ph/#1}, 
%{\begin{htmlonly}\htmladdnormallink{ADS}{http://adsabs.harvard.edu/cgi-bin/nph-bib_query?bibcode=#2}\end{htmlonly}}}
\newcommand{\yab}[5]{:~#1, ``#5,'' {\em Astrobiol.\ }{\bf #2}, #3--#4}
\newcommand{\yaraa}[5]{:~#1, ``#5,'' {\em Annu.\ Rev.\ Astron.\ Astrophys.\ }{\bf #2}, #3--#4}
\newcommand{\yarfm}[5]{:~#1, ``#5,'' {\em Annu.\ Rev.\ Fluid Mech.\ }{\bf #2}, #3--#4}
\newcommand{\yica}[5]{:~#1, ``#5,'' {\em Icarus }{\bf #2}, #3--#4}
\newcommand{\ysph}[5]{:~#1, ``#5,'' {\em Solar Phys.\ }{\bf #2}, #3--#4}
\newcommand{\ysphs}[5]{:~#1, ``#5'' {\em Solar Phys.\ }{\bf #2}, #3--#4}
\newcommand{\ymn}[5]{:~#1, ``#5,'' {\em Mon.\ Not.\ Roy.\ Astron.\ Soc.\ }{\bf #2}, #3--#4}
\newcommand{\yan}[5]{:~#1, ``#5,'' {\em Astron.\ Nachr.\ }{\bf #2}, #3--#4}
\newcommand{\yanN}[4]{:~#1, ``#4,'' {\em Astron.\ Nachr.\ }{\bf #2}, #3}
\newcommand{\yana}[5]{:~#1, ``#5,'' {\em Astron.\ Astrophys.\ }{\bf #2}, #3--#4}
\newcommand{\yanaS}[5]{:~#1, ``#5'' {\em Astron.\ Astrophys.\ }{\bf #2}, #3--#4}
\newcommand{\yanaN}[4]{:~#1, ``#4,'' {\em Astron.\ Astrophys.\ }{\bf #2}, #3}
\newcommand{\yanas}[5]{:~#1, ``#5'' {\em Astron.\ Astrophys.\ }{\bf #2}, #3--#4}
\newcommand{\yass}[5]{:~#1, ``#5,'' {\em Astrophys.\ Spa.\ Sci.\ }{\bf #2}, #3--#4}
\newcommand{\yaas}[4]{:~#1, ``#4,'' {\em Bull.\ Am.\ Astron.\ Soc.\ }{\bf #2}, #3}
\newcommand{\yaps}[4]{:~#1, ``#4,'' {\em Bull.\ Am.\ Phys.\ Soc.\ }{\bf #2}, #3}
\newcommand{\ynat}[5]{:~#1, ``#5,'' {\em Nature }{\bf #2}, #3--#4}
\newcommand{\ysci}[5]{:~#1, ``#5,'' {\em Science }{\bf #2}, #3--#4}
\newcommand{\yjcap}[4]{:~#1, ``#4,'' {\em J. Cosmol. Astropart. Phys. }{\bf #2}, #3}
\newcommand{\yapj}[5]{:~#1, ``#5,'' {\em Astrophys.\ J.\ }{\bf #2}, #3--#4}
\newcommand{\yps}[5]{:~#1, ``#5,'' {\em Phys.\ Scr.\ }{\bf #2}, #3--#4}
\newcommand{\ypsN}[4]{:~#1, ``#4,'' {\em Phys.\ Scr.\ }{\bf #2}, #3}
\newcommand{\yapjSN}[4]{:~#1, ``#4'' {\em Astrophys.\ J.\ }{\bf #2}, #3}
\newcommand{\yapjN}[4]{:~#1, ``#4,'' {\em Astrophys.\ J.\ }{\bf #2}, #3}
\newcommand{\yjppN}[4]{:~#1, ``#4,'' {\em J.\ Plasma Phys.\ }{\bf #2}, #3}
\newcommand{\yapjlN}[4]{:~#1, ``#4,'' {\em Astrophys.\ J.\ Lett.\ }{\bf #2}, #3}
\newcommand{\yapjs}[5]{:~#1, ``#5,'' {\em Astrophys.\ J. Suppl.\ }{\bf #2}, #3--#4}
\newcommand{\yapjl}[5]{:~#1, ``#5,'' {\em Astrophys.\ J.\ Lett.\ }{\bf #2}, #3--#4}
\newcommand{\yapjlS}[5]{:~#1, ``#5'' {\em Astrophys.\ J.\ Lett.\ }{\bf #2}, #3--#4}
\newcommand{\yapjlNS}[4]{:~#1, ``#4'' {\em Astrophys.\ J.\ Lett.\ }{\bf #2}, #3}
\newcommand{\yjfm}[5]{:~#1, ``#5,'' {\em J.\ Fluid Mech.\ }{\bf #2}, #3--#4}
\newcommand{\yjfmN}[4]{:~#1, ``#4,'' {\em J.\ Fluid Mech.\ }{\bf #2}, #3}
\newcommand{\ypepi}[5]{:~#1, ``#5,'' {\em Phys. Earth Planet. Int.}{\bf #2}, #3--#4}
\newcommand{\ypnas}[5]{:~#1, ``#5,'' {\em Proc.\ Natl.\ Acad.\ Sci.}{\bf #2}, #3--#4}
\newcommand{\ygrlN}[4]{:~#1, ``#4,'' {\em Geophys.\ Res.\ Lett.\ }{\bf #2}, #3}
\newcommand{\ygrl}[5]{:~#1, ``#5,'' {\em Geophys.\ Res.\ Lett.\ }{\bf #2}, #3--#4}
\newcommand{\ygafd}[5]{:~#1, ``#5,'' {\em Geophys.\ Astrophys.\ Fluid Dyn.\ }{\bf #2}, #3--#4}
\newcommand{\ygafdS}[5]{:~#1, ``#5'' {\em Geophys.\ Astrophys.\ Fluid Dyn.\ }{\bf #2}, #3--#4}
\newcommand{\yoleb}[5]{:~#1, ``#5,'' {\em Orig.\ Life Evol.\ Biosph.\ }{\bf #2}, #3--#4}
\newcommand{\sproc}[5]{:~#1, ``#2,'' in {\em #3}, ed.\ #4, #5, submitted}
\newcommand{\sprocS}[5]{:~#1, ``#2'' in {\em #3}, ed.\ #4, #5, submitted}
\newcommand{\eproc}[6]{:~#1, ``#2,'' in {\em #3}, \url{#4}, #5}
\newcommand{\pproc}[5]{:~#1, ``#2,'' in {\em #3}, ed.\ #4, #5, in press}
\newcommand{\pprocS}[5]{:~#1, ``#2'' in {\em #3}, ed.\ #4, #5, in press}
\newcommand{\ypr}[5]{:~#1, ``#5,'' {\em Phys.\ Rev.\ }{\bf #2}, #3--#4}
\newcommand{\ypra}[5]{:~#1, ``#5,'' {\em Phys.\ Rev. A }\/ {\bf #2}, #3--#4}
\newcommand{\yprb}[5]{:~#1, ``#5,'' {\em Phys.\ Rev. B }\/ {\bf #2}, #3--#4}
\newcommand{\yprd}[5]{:~#1, ``#5,'' {\em Phys.\ Rev. D }\/ {\bf #2}, #3--#4}
\newcommand{\ypre}[5]{:~#1, ``#5,'' {\em Phys.\ Rev. E }\/ {\bf #2}, #3--#4}
\newcommand{\yprdN}[4]{:~#1, ``#4,'' {\em Phys.\ Rev. D }\/ {\bf #2}, #3}
\newcommand{\yprdNS}[4]{:~#1, ``#4'' {\em Phys.\ Rev. D }\/ {\bf #2}, #3}
\newcommand{\ypreN}[4]{:~#1, ``#4,'' {\em Phys.\ Rev. E }\/ {\bf #2}, #3}
\newcommand{\ypreNS}[4]{:~#1, ``#4'' {\em Phys.\ Rev. E }\/ {\bf #2}, #3}
\newcommand{\yprN}[5]{:~#1, ``#5,'' {\em Phys.\ Rev.\ }{\bf #2}, #3, #4}
\newcommand{\yprlN}[4]{:~#1, ``#4,'' {\em Phys.\ Rev.\ Lett.\ }{\bf #2}, #3}
\newcommand{\yprr}[4]{:~#1, ``#4,'' {\em Phys.\ Rev. Res. }\/ {\bf #2}, #3}
\newcommand{\yproc}[7]{:~#1, ``#4,'' in {\em #5}, ed.\ #6, #7, pp.~#2--#3}
\newcommand{\yprocS}[7]{:~#1, ``#4'' in {\em #5}, ed.\ #6, #7, pp.~#2--#3}
\newcommand{\yprocN}[6]{:~#1, ``#3,'' in {\em #4}, ed.\ #5, #6, #2}
\newcommand{\yjour}[6]{:~#1, ``#6,'' {\em #2} {\bf #3}, #4--#5}
\newcommand{\yjourN}[5]{:~#1, ``#5,'' {\em #2} {\bf #3}, #4}
\newcommand{\yjourS}[6]{:~#1, ``#6'' {\em #2} {\bf #3}, #4--#5}
\newcommand{\tjour}[3]{:~#1, ``#3,'' {\em #2}, to be submitted}
\newcommand{\sjour}[3]{:~#1, ``#3,'' {\em #2}, submitted}
\newcommand{\sjourS}[3]{:~#1, ``#3'' {\em #2}, submitted}
\newcommand{\ppjour}[4]{:~#1, ``#4'' {\em #2} {\bf #3}}
\newcommand{\pjour}[3]{:~#1, ``#3,'' {\em #2}, in press}
\newcommand{\djour}[4]{:~#1, ``#3,'' {\em #2}, DOI: #4}
\newcommand{\dnjour}[5]{:~#1, ``#4,'' {\em #2}, {\bf #3}. DOI: #5}
\newcommand{\ypreprint}[2]{:~#1, #2}
\newcommand{\yprep}[2]{:~#1, ``#2''}
\newcommand{\ybook}[3]{:~#1,  {\em #2}. #3}
\newcommand{\yija}[5]{:~#1, ``#5,'' {\em Int.\ J.\ Astrobiol.\ }{\bf #2}, #3--#4}
\newcommand{\yijaS}[5]{:~#1, ``#5'' {\em Int.\ J.\ Astrobiol.\ }{\bf #2}, #3--#4}
\newcommand{\ypf}[5]{:~#1, ``#5,'' {\em Phys.\ Fluids}\/ {\bf #2}, #3--#4}
\newcommand{\ypfN}[4]{:~#1, ``#4,'' {\em Phys.\ Fluids}\/ {\bf #2}, #3}
\newcommand{\ypp}[5]{:~#1, ``#5,'' {\em Phys.\ Plasmas}\/ {\bf #2}, #3--#4}
\newcommand{\yppN}[4]{:~#1, ``#4,'' {\em Phys.\ Plasmas}\/ {\bf #2}, #3}
\newcommand{\yepl}[5]{:~#1, ``#5,'' {\em Europhys.\ Lett.\ }{\bf #2}, #3--#4}
\newcommand{\yprl}[5]{:~#1, ``#5,'' {\em Phys.\ Rev.\ Lett.\ }{\bf #2}, #3--#4}
\newcommand{\ybif}[5]{:~#1, ``#5,'' {\em Int.\ J.\ Bifurc.\ Chaos}\/ {\bf #2}, #3--#4}
\newcommand{\ycsf}[5]{:~#1, ``#5,'' {\em Chaos, Solitons \& Fractals }{\bf #2}, #3--#4}
\newcommand{\ycsfS}[5]{:~#1, ``#5'' {\em Chaos, Solitons \& Fractals }{\bf #2}, #3--#4}
%
\newcommand{\sph}{{\em Solar~Phys.~}}
\newcommand{\ana}{{\em Astron.~Astrophys.~}}
\newcommand{\gafd}{{\em Geophys.~Astrophys.~Fluid Dyn.~}}
%
\newcommand{\sica}[2]{:~#1, ``#2,'' {\em Icarus}, submitted}
\newcommand{\pica}[2]{:~#1, ``#2,'' {\em Icarus}, in press}
\newcommand{\sepl}[2]{:~#1, ``#2,'' {\em Europhys.\ Lett.}, submitted}
\newcommand{\pepl}[2]{:~#1, ``#2,'' {\em Europhys.\ Lett.}, in press}
\newcommand{\tjcap}[2]{:~#1, ``#2,'' {\em J. Cosmol. Astropart. Phys.}, to be submitted}
\newcommand{\sjcap}[2]{:~#1, ``#2,'' {\em J. Cosmol. Astropart. Phys.}, submitted}
\newcommand{\pjcap}[2]{:~#1, ``#2,'' {\em J. Cosmol. Astropart. Phys.}, in press}
\newcommand{\tbs}[2]{:~#1, ``#2,'' to be submitted}
\newcommand{\tprl}[2]{:~#1, ``#2,'' {\em Phys.\ Rev.\ Lett.}, to be submitted}
\newcommand{\sprl}[2]{:~#1, ``#2,'' {\em Phys.\ Rev.\ Lett.}, submitted}
\newcommand{\sprf}[2]{:~#1, ``#2,'' {\em Phys.\ Rev.\ Fluids}, submitted}
\newcommand{\pprf}[2]{:~#1, ``#2,'' {\em Phys.\ Rev.\ Fluids}, in press}
\newcommand{\yprf}[4]{:~#1, ``#4,'' {\em Phys.\ Rev.\ Fluids}, {\bf #2}, #3}
\newcommand{\sprb}[2]{:~#1, ``#2,'' {\em Phys.\ Rev.\ B}, submitted}
\newcommand{\sprd}[2]{:~#1, ``#2,'' {\em Phys.\ Rev.\ D}, submitted}
\newcommand{\sprdS}[2]{:~#1, ``#2'' {\em Phys.\ Rev.\ D}, submitted}
\newcommand{\sprr}[2]{:~#1, ``#2,'' {\em Phys.\ Rev.\ Res.}, submitted}
\newcommand{\tprd}[2]{:~#1, ``#2,'' {\em Phys.\ Rev.\ D}, to be submitted}
\newcommand{\tprdS}[2]{:~#1, ``#2'' {\em Phys.\ Rev.\ D}, to be submitted}
\newcommand{\spre}[2]{:~#1, ``#2,'' {\em Phys.\ Rev.\ E}, submitted}
\newcommand{\spreS}[2]{:~#1, ``#2'' {\em Phys.\ Rev.\ E}, submitted}
\newcommand{\pprl}[2]{:~#1, ``#2,'' {\em Phys.\ Rev.\ Lett.}, in press}
\newcommand{\pprd}[2]{:~#1, ``#2,'' {\em Phys.\ Rev.\ D}, in press}
\newcommand{\pprdS}[2]{:~#1, ``#2'' {\em Phys.\ Rev.\ D}, in press}
\newcommand{\ppre}[2]{:~#1, ``#2,'' {\em Phys.\ Rev.\ E}, in press}
\newcommand{\pprr}[2]{:~#1, ``#2,'' {\em Phys.\ Rev.\ Res.}, in press}
\newcommand{\ppreS}[2]{:~#1, ``#2'' {\em Phys.\ Rev.\ E}, in press}
\newcommand{\tpr}[3]{:~#1, ``#3,'' {\em Phys.\ Rev.\ }{\bf #2}, to be submitted}
\newcommand{\spr}[3]{:~#1, ``#3,'' {\em Phys.\ Rev.\ }{\bf #2}, submitted}
\newcommand{\ppr}[3]{:~#1, ``#3,'' {\em Phys.\ Rev.\ }{\bf #2}, in press}
\newcommand{\pppp}[3]{:~#1, ``#2,'' {\em Phys.\ Plasmas}, in press, scheduled for the #3 issue}
\newcommand{\ppp}[2]{:~#1, ``#2,'' {\em Phys.\ Plasmas}, in press}
\newcommand{\spp}[2]{:~#1, ``#2,'' {\em Phys.\ Plasmas}, submitted}
\newcommand{\tpp}[2]{:~#1, ``#2,'' {\em Phys.\ Plasmas}, to be submitted}
\newcommand{\tppS}[2]{:~#1, ``#2'' {\em Phys.\ Plasmas}, to be submitted}
\newcommand{\tbsan}[2]{:~#1, ``#2,'' {\em Astron.\ Nachr.}, to be submitted}
\newcommand{\san}[2]{:~#1, ``#2,'' {\em Astron.\ Nachr.}, submitted}
\newcommand{\pan}[2]{:~#1, ``#2,'' {\em Astron.\ Nachr.}, in press}
\newcommand{\dan}[3]{:~#1, ``#2,'' {\em Astron.\ Nachr.}, DOI: #3}
\newcommand{\sab}[2]{:~#1, ``#2,'' {\em Astrobiol.}, submitted}
\newcommand{\pab}[2]{:~#1, ``#2,'' {\em Astrobiol.}, in press}
\newcommand{\soleb}[2]{:~#1, ``#2,'' {\em Orig.\ Life Evol.\ Biosph.}, submitted}
\newcommand{\poleb}[2]{:~#1, ``#2,'' {\em Orig.\ Life Evol.\ Biosph.}, in press}
\newcommand{\doleb}[3]{:~#1, ``#2,'' {\em Orig.\ Life Evol.\ Biosph.}, DOI: #3}
\newcommand{\sija}[2]{:~#1, ``#2,'' {\em Int.\ J.\ Astrobiol.}, submitted}
\newcommand{\sana}[2]{:~#1, ``#2,'' {\em Astron.\ Astrophys.}, submitted}
\newcommand{\tana}[2]{:~#1, ``#2,'' {\em Astron.\ Astrophys.}, to be submitted}
\newcommand{\sanas}[2]{:~#1, ``#2'' {\em Astron.\ Astrophys.}, submitted}
\newcommand{\pana}[2]{:~#1, ``#2,'' {\em Astron.\ Astrophys.}, in press}
\newcommand{\dana}[3]{:~#1, ``#2,'' {\em Astron.\ Astrophys.}, DOI: #3}
\newcommand{\panas}[2]{:~#1, ``#2'' {\em Astron.\ Astrophys.}, in press}
\newcommand{\sgafd}[2]{:~#1, ``#2,'' {\em Geophys.\ Astrophys.\ Fluid Dyn.}, submitted}
\newcommand{\sgafdS}[2]{:~#1, ``#2'' {\em Geophys.\ Astrophys.\ Fluid Dyn.}, submitted}
\newcommand{\pgafdS}[2]{:~#1, ``#2'' {\em Geophys.\ Astrophys.\ Fluid Dyn.}, in press}
\newcommand{\pgafd}[2]{:~#1, ``#2,'' {\em Geophys.\ Astrophys.\ Fluid Dyn.}, in press}
\newcommand{\dgafd}[3]{:~#1, ``#2,'' {\em Geophys.\ Astrophys.\ Fluid Dyn.}, DOI: #3}
\newcommand{\ppgafd}[3]{:~#1, ``#3,'' {\em Geophys.\ Astrophys.\ Fluid Dyn.} {\bf #2}}
\newcommand{\tapj}[2]{:~#1, ``#2,'' {\em Astrophys.\ J.}, to be submitted}
\newcommand{\papj}[2]{:~#1, ``#2,'' {\em Astrophys.\ J.}, in press}
\newcommand{\papjl}[2]{:~#1, ``#2,'' {\em Astrophys.\ J.\ Lett.}, in press}
\newcommand{\sapj}[2]{:~#1, ``#2,'' {\em Astrophys.\ J.}, submitted}
\newcommand{\sapjS}[2]{:~#1, ``#2'' {\em Astrophys.\ J.}, submitted}
\newcommand{\papjS}[2]{:~#1, ``#2'' {\em Astrophys.\ J.}, in press}
\newcommand{\tapjl}[2]{:~#1, ``#2,'' {\em Astrophys.\ J.\ Lett.}, to be submitted}
\newcommand{\sapjl}[2]{:~#1, ``#2,'' {\em Astrophys.\ J.\ Lett.}, submitted}
\newcommand{\papjlS}[2]{:~#1, ``#2'' {\em Astrophys.\ J.\ Lett.}, in press}
\newcommand{\sapjlS}[2]{:~#1, ``#2'' {\em Astrophys.\ J.\ Lett.}, submitted}
%\newcommand{\papj}[3]{:~#1, ``#3,'' {\em Astrophys.\ J.\ }{\bf #2}}
\newcommand{\ppapj}[4]{:~#1, ``#3,'' {\em Astrophys.\ J.\ }{\bf #2}, scheduled for the #4 issue}
\newcommand{\ppapjS}[4]{:~#1, ``#3'' {\em Astrophys.\ J.\ }{\bf #2}, scheduled for the #4 issue}
\newcommand{\ppapjl}[4]{:~#1, ``#3,'' {\em Astrophys.\ J.\ Lett.\ }{\bf #2}, scheduled for the #4 issue}
\newcommand{\ppapjlS}[4]{:~#1, ``#3'' {\em Astrophys.\ J.\ Lett.\ }{\bf #2}, scheduled for the #4 issue}
\newcommand{\tjpp}[2]{:~#1, ``#2,'' {\em J.\ Plasma Phys.}, to be submitted}
\newcommand{\sjpp}[2]{:~#1, ``#2,'' {\em J.\ Plasma Phys.}, submitted}
\newcommand{\pjpp}[2]{:~#1, ``#2,'' {\em J.\ Plasma Phys.}, in press}
\newcommand{\djpp}[3]{:~#1, ``#2,'' {\em J.\ Plasma Phys.}, DOI: #3}
\newcommand{\unpub}[2]{:~#1, ``#2,'' unpublished}
\newcommand{\spf}[2]{:~#1, ``#2,'' {\em Phys.\ Fluids}, submitted}
\newcommand{\ppf}[2]{:~#1, ``#2,'' {\em Phys.\ Fluids}, in press}
\newcommand{\ssph}[2]{:~#1, ``#2,'' {\em Solar Phys.}, submitted}
\newcommand{\psph}[2]{:~#1, ``#2,'' {\em Solar Phys.}, in press}
\newcommand{\dsph}[3]{:~#1, ``#2,'' {\em Solar Phys.}, DOI: #3}
\newcommand{\sbif}[2]{:~#1, ``#2,'' {\em Int.\ J.\ Bifurc.\ Chaos}, submitted}
\newcommand{\pbif}[2]{:~#1, ``#2,'' {\em Int.\ J.\ Bifurc.\ Chaos}, in press}
\newcommand{\scsfS}[2]{:~#1, ``#2'' {\em Chaos, Solitons \& Fractals}
(submitted}
\newcommand{\scsf}[2]{:~#1, ``#2,'' {\em Chaos, Solitons \& Fractals}
(submitted}
\newcommand{\pcsfS}[3]{:~#1, ``#2'' {\em Chaos, Solitons \& Fractals}
(#3}
\newcommand{\pcsf}[3]{:~#1, ``#2,'' {\em Chaos, Solitons \& Fractals}
(#3}
\newcommand{\pjfm}[2]{:~#1, ``#2,'' {\em J.\ Fluid Mech.}, in press}
\newcommand{\sjfm}[2]{:~#1, ``#2,'' {\em J.\ Fluid Mech.}, submitted}
\newcommand{\tjfm}[2]{:~#1, ``#2,'' {\em J.\ Fluid Mech.}, to be submitted}
\newcommand{\dmn}[3]{:~#1, ``#2,'' {\em Mon.\ Not.\ Roy.\ Astron.\ Soc.}, DOI: #3}
\newcommand{\pmn}[2]{:~#1, ``#2,'' {\em Mon.\ Not.\ Roy.\ Astron.\ Soc.}, in press}
\newcommand{\smn}[2]{:~#1, ``#2,'' {\em Mon.\ Not.\ Roy.\ Astron.\ Soc.}, submitted}
\newcommand{\smnS}[2]{:~#1, ``#2'' {\em Mon.\ Not.\ Roy.\ Astron.\ Soc.}, submitted}
\newcommand{\tmn}[2]{:~#1, ``#2,'' {\em Mon.\ Not.\ Roy.\ Astron.\ Soc.}, to be submitted}

\newcommand{\km}{\,{\rm km}}

\date{}
%\title{\bf Curriculum vitae and list of publications}
\title{\bf Curriculum vitae}
\author{Axel Brandenburg\\
\small\today\normalsize}\begin{document}
%\maketitle
%\noindent\LARGE{\bf Appendix B. Curriculum vitae and list of publications of Axel Brandenburg}\normalsize\\
%\noindent\LARGE{\bf Curriculum vitae and list of publications \\ of Axel Brandenburg}\normalsize\\
\noindent\LARGE{\bf Curriculum vitae of Axel Brandenburg}\normalsize\\
%\small December 2, 1994\normalsize}\begin{document}\maketitle
\small \today \normalsize
\\

\noindent
Born: 7 April 1959 in Heide, Federal Republic of Germany\\
Nationality: German\\
Marital status: married, 1 child\\
%henkilotunnus on 070459 209P
%dansk person nummer 070459 3015
\vspace{-4mm}

\section*{Address}
% 
%Observatory and Astrophysics Laboratory, University of Helsinki\\
%T\"ahtitorninm\"aki, SF-00130 Helsinki, Finland \\
%Telephone: + 358 - 0 - 191 2948\\
%e-mail: brandenburg@csc.fi
% 
%NORDITA, Blegdamsvej 17, DK 2100 Copenhagen \O, Denmark\\
%Telephone: + 45 - 353 25 228 \\
%FAX: + 45 - 353 89 157 \\
%e-mail: {\sf brandenb@nordita.dk}\\
%\url{http://www.nordita.dk/~brandenb}
% 
%Issac Newton Institute, 20 Clarkson Road, Cambridge CB3 0EH, UK\\
%Telephone: + 44 - 223 33 0555\\
%FAX: + 44 - 223 33 0508\\
%e-mail: brandenb@newton.cam.ac.uk\\
% 
%HAO/NCAR, P.O. Box 3000, Boulder, CO 80307-3000, USA\\
%FAX + 1 (303) 497 1589, Tel + 1 (303) 497 1586\\
%e-mail: brandenb@hao.ucar.edu\\
%School of Mathematics and Statistics\\
%Department of Mathematics\\
%University of Newcastle\\
%Newcastle upon Tyne, NE1 7RU, UK\\
%Tel: +44 191 222-7312\\
%FAX: +44 191 222-8020\\
%email: Axel.Brandenburg@Newcastle.ac.uk\\
%http://antares.ncl.ac.uk/~brandenb
%

% Laboratory for Atmospheric and Space Physics,
% Department of Astrophysical and Planetary Sciences, \& JILA,
% University of Colorado, 3665 Discovery Drive, Boulder, CO 80303, USA\\
% %Axel.Brandenburg@Colorado.edu, phones: (303) 492-9309 and (303) 735-7738\\
% {\sf Axel.Brandenburg@Colorado.edu}, phone: (303) 735-7738\\

\vspace{-2mm}\noindent
Nordita, KTH Royal Institute of Technology and Stockholm University,
AlbaNova University Center,
Hannes Alfv\'ens v\"ag 12, SE - 106 91 Stockholm, Sweden;
%Telephone: +46 8 5537 8707, \quad FAX: +46 8 5537 8404, \quad mobile: +46 7 3270 4376 \\
%\vspace{-2mm}\noindent
e-mail: {\sf brandenb@nordita.org},\\
\quad\url{http://www.nordita.org/~brandenb},
\quad\url{http://orcid.org/0000-0002-7304-021X}

\section*{Education}
%
\begin{description}\itemsep-1mm

\item Docent of Astronomy, University of Helsinki, March 1992

\item Dr. Phil., University of Helsinki, May 1990,
Doctoral dissertation: {\em Challenges for solar dynamo theory: 
$\alpha$-effect, differential rotation and stability},
ISBN 952-90-1697-2
%, ``magna cum laude approbatur''

\item Lic. Phil., University of Helsinki, February 1989,
Licentiate thesis: {\em Kinematic dynamo theory and the solar activity cycle}
%, ``eximia cum laude approbatur''

\item Dipl. Phys., University of Hamburg, January 1986,
Diploma thesis: {\em Hydrodynamics of convective bubbles 
in linear approximation}
%, ``sehr gut''

%\item Abitur, Werner Heisenberg Gymnasium, Heide, June 1978

\end{description}
 
\section*{Employment}
% 
\begin{tabbing}
Jan 2007 -- present:\quad\quad \=Professor of Astrophysics, Stockholm Observatory, NORDITA, Stockholm\\
%May 2018 -- present:\>Adjunct Professor, Carnegie Mellon University \\
Aug 2015 -- Jul. 2018:\>Visiting Faculty, University of Colorado, Boulder (LASP, APS, and JILA)\\
Jan 2000 -- Dec. 2006:\>Professor of Astrophysics, NORDITA, Copenhagen\\
Feb 1996 -- Dec. 2000:\>Professor of Applied Mathematics, University of Newcastle upon Tyne\\
Dec 1994 -- Jan. 1996:\>Nordic Assistant Professor, Nordita, Copenhagen\\
Dec 1992 -- Nov. 1994:\>Postdoctoral Research Fellow, High Altitude Observatory/NCAR, Boulder\\
% SERC Grant GR G59981
%Mar 1992            \>Docent of Astronomy, University of Helsinki\\
Aug 1992 -- Nov. 1992:\>Visiting Fellowship, University of Cambridge\\
Sep 1990 -- Aug. 1992:\>Postdoctoral Research Fellow, Nordita, Copenhagen\\
\end{tabbing}

\section*{Publications}
Below the numbers of publications (published or in print)
and the $h$ indexes (from Web of science, ResearcherID: I-6668-2013),
the Astrophysical Data Service (ADS), and Google Scholar (GS); see also:\\
\url{http://www.nordita.org/~brandenb/pub/node1.html}\\
\\
Number of papers in refereed journals: 459 + 5 submitted\\
Number of invited conference reviews: 43\\
Number of communications to scientific meetings: 86\\
Total number of citations:
% Hamburg or Helsinki or Newcastle or boulder or stockholm or copenhagen
17069, $h$-index 65 (on Web of Science); %WoS (3-jun-23)
20784, $h$-index 70 (ADS); and \\ %(25-Jan-23)
27505, $h$-index: 84 (on Google Scholar)\\ %(12-nov-24)

\section*{Influential papers}

{\it
The second column refers to the paper number in the full list of publications,\\
%\url{http://www.nordita.org/~brandenb/pub/node1.html}}\\
\url{http://norlx65.nordita.org/~brandenb/pub/node1.html}}\\
%Citations are from Web of Science (WoS$^*$),
Citations are from Web of Science (WoS),
Astrophysical Data Service (ADS), and Google Scholar (GS).\\
%$^*$not up-to-date
\vspace{2mm}

\begin{tabular}{llrrr}
      &    & \multicolumn{3}{c}{citations}\\
%paper:& $\;\;$\# & WoS$^*$ & ADS & GS \\
paper:& $\;\;$\# & WoS & ADS & GS~ \\
\hline        
Brandenburg \& Subramanian (2005)              & A.153 &1186 &1381 &1895 \\
Beck, Brandenburg et al.\ (1996)               & A.58  & 801 & 883 &1267 \\
Brandenburg et al.\ (1995)                     & A.44  & 698 & 760 &1095 \\
Brandenburg (2001)                             & A.98  & 439 & 486 & 685 \\
Brandenburg (2005)                             & A.145 & 354 & 341 & 438 \\
Haugen, Brandenburg, \& Dobler (2004)          & A.133 & 269 & 302 & 401 \\
Saar \& Brandenburg (1999)                     & A.90  & 257 & 288 & 381 \\
Brandenburg, Enqvist, \& Olesen (1996)         & A.54  & 231 & 276 & 349 \\
Nordlund, Brandenburg, et al.\ (1992)          & A.22  & 217 & 230 & 299 \\
Brandenburg \& Dobler (2002)                   & A.111 & 194 & 214 & 299 \\
Brandenburg et al.\ (1996)                     & A.52  & 192 & 202 & 291 \\
Dobler, Stix, \& Brandenburg (2006)            & A.159 & 177 & 203 & 289 \\
Christensson, Hindmarsh, \& Brandenburg (2001) & A.104 & 172 & 197 & 247 \\
Brandenburg et al.\ (1989)                     & A.3   & 174 & 184 & 232 \\
Korpi, Brandenburg, et al.\ (1999)             & A.82  & 164 & 180 & 238 \\
Blackman \& Brandenburg (2002)                 & A.115 & 146 & 167 & 213 \\
R\"udiger \& Brandenburg (1995)                & A.41  & 146 & 149 & 199 \\
%Pudritz et al.\ (2007)                & B.25  & --- & 301 & 209 \\
\end{tabular}
\vspace{2mm}

\section*{PhD students}

\begin{tabbing}
Stephen J. Brooks: ~\= 1996--2000\quad \= (Newcastle upon Tyne)\\
Alberto Bigazzi:    \> 1996--2000  \> (Newcastle upon Tyne and L'Aquila, Rome)\\
Maarit J.\ Korpi:   \> 1997--1999  \> (Oulu U)\\
Nils E.\ L. Haugen  \> 2000--2004  \> (Trondheim, NTNU)\\
Tarek A. Yousef     \> 2000--2004  \> (Trondheim, NTNU)\\
Antony J. Mee       \> 2002--2006  \> (Newcastle upon Tyne, co-supervisor)\\
Simon Candelaresi   \> 2009--2012  \> (Stockholm U, Phil. Lic. in Feb. 2011)\\
Fabio Del Sordo     \> 2009--2012  \> (Stockholm U, Phil. Lic. in Feb. 2011)\\
Koen Kemel          \> 2009--2012  \> (Stockholm U, Phil. Lic. in Aug 2011)\\
J\"orn Warnecke     \> 2009--2013  \> (Stockholm U, Phil. Lic. in May 2011)\\
Sarah Jabbari       \> 2012--2016  \> (Stockholm U, Phil. Lic. in May 2014) \\
Illa R. Losada      \> 2013--2019  \> (Stockholm U, Phil. Lic. in Dec 2014) \\
Xiang-Yu Li         \> 2014--2018  \> (Stockholm U, Phil. Lic. in May 2016) \\
Alberto Roper Pol   \> 2017--2020  \> (University of Colorado) \\
Yutong He           \> 2020--2204  \> (Stockholm U, Phil. Lic. in Dec 2022) \\
\end{tabbing}
\vspace{-4mm}

\noindent
Master students: Atefeh Barekat (2013), Nousaba Nasrin Protiti (2023)

\vspace{4mm}
\noindent
Batchelor students: Julia Asplund (2019), Gustav Larsson (2023)

%\begin{itemize}\itemsep -2pt
%\item
%University of Helsinki: Maarit J.\ Korpi$^*$ (1999)
%\item
%University of Newcastle: Stephen J.\ Brooks$^*$ (2000), Antony J.\ Mee (2006)
%\item
%University of L'Aquila: Alberto Bigazzi (2000)
%\item
%University of Trondheim: Nils E.\ L.\ Haugen (2004), Tarek A.\ Yousef (2004)
%\item
%Stockholm University: Simon Candelaresi$^*$, Fabio Del Sordo$^*$,
%Koen Kemel$^*$ (2012), J\"orn Warnecke$^*$ (2013),
%Sarah Jabbari$^*$ (2016$^{**}$), Illa R.\ Losada$^*$ (2016$^{**}$)
%\end{itemize}
%\vspace{-4mm}
%$^*$ I was the official supervisor, $^{**}$ anticipated defense date.
%\vspace{4mm}

%\section*{Licentiate theses}

%\begin{tabbing}
%Simon Candelaresi   \> 2011  \> (Stockholm U)
%Fabio Del Sordo     \> 2011  \> (Stockholm U)
%\end{tabbing}

%\noindent {\em (iii) Supervised post-doctoral fellows}
%\begin{itemize}\itemsep -2pt
%\item
%Marie Curie Fellows: F.J. Sanchez-Salcedo (1999), Boris Dintrans (2001)
%\item
%Project grants: Brigitta von Rekowski (2001), Anne Bardou (2001),
%Wolfgang Dobler (2001),
%Alexander Hubbard (2011), Piyali Chatterjee (2011), Gustavo Guerrero (2011)
%\item
%Nordita fellows in astrophysics: 15 since 2001
%%Anja Andersen, Mattias Christensen, Raphael Plasson, Maarit Korpi,
%%Petri Kapyla, Oleg Kochukhov, Karen Guldbaek, Ebru Devlen,
%%Van Eysden, Oliver Gressel, Mikhail Modestov, Chi-kwan Chan,
%%Niccolo Bucciantini, Eniko Madarassy, Torgny Karlsson
%\end{itemize}

\section*{Teaching experience}

\begin{itemize}

\item[--] {\it Advanced Astrophysical Fluid Dynamics}
(7.5 ECTS) at Stockholm U, postgraduate level (2021)

\item[--] {\it Search for Life in the Universe}
(44 hours) at CU-Boulder, for non-science majors (2017, spring+fall)

\item[--] {\it Fluid Instabilities, Waves, \& Turbulence}
(44 hours) at CU-Boulder, graduate level (2016)

\item[--] {\it Solar \& Space Physics}
(44 hours) at CU-Boulder, upper undergraduate level (2016)

\item[--] {\it Astrophysical Fluid Dynamics}
(7.5 ECTS) at Stockholm U, postgraduate level (2013)

\item[--] {\it Astrophysical Magnetohydrodynamics}
(7.5 ECTS) at Stockholm U, master level (2012)

\item[--] {\it Solar Physics and Magnetohydrodynamics}
(7.5 ECTS) at Stockholm U, postgraduate level (2009)

\item[--] {\it Pencil Code tutorials},
taught in Trieste (Italy, 2009) and Aussois (France, 2009)

\item[--] {\it Solar Physics}
(12 hours) at the IRF Kiruna (2005, 2006, 2007, 2008), postgraduate level

\item[--] {\it Planetary and Stellar Orbits}
(24 hours) at University of Newcastle upon Tyne (1998, 1999, 2000),
second year students

\item[--] {\it Introduction to Astrophysical Fluids}
(24 hours) at University of Newcastle upon Tyne (1997, 1998, 1999),
second year students

\item[--] {\it Fluid Flow and Cosmic Fluids}
(24 hours) at University of Newcastle upon Tyne (1997, 1998, 1999),
third year students

\item[--] {\it Relativistic Fluid Dynamics and Visualization}
(24 hours) at Copenhagen University (1995/1996), shared with {\AA}ke Nordlund,
postgraduate level

\end{itemize}

\section*{Notable recognitions}

\begin{itemize}\itemsep-1pt
\item[2014]
Elected foreign member of the Royal Swedish Academy of Sciences\\
\url{https://www.kva.se/en/news/ny-ledamot-invald-i-akademien-2-3/}

\item[2019]
Honorary professor of Ilia State University (Tbilisi/Georgia)

\item[2019]
Distinguished Fellow of NYUAD (Abu Dhabi)

\item[2022]
Jes\'us Serra Foundation visiting fellow at the Institute of Astrophysics of the Canary Islands

\end{itemize}

\section*{Major grants}

\begin{itemize}

\item
VR project grant, ``Stochastic Gravitational Wave Background from the Early Turbulent Universe''
2019-04234, January 2020 -- December 2022, 4.00 MSEK (430 k\$, as PI)

\item
NSF Astronomy and Astrophysics Research Grants (AAG),
``Collaborative research: A Comprehensive Theoretical Study of Cosmic
Magnetic Fields their Origin, Evolution, and Signatures''
1615100, July 2016 -- June 2019, 224 k\$ (as Co-I/Institutional PI;
PI: Tina Kahniashvili, Carnegie Mellon University)
%13008907

\item
Knut \& Alice Wallenberg Foundation,
``Bottlenecks for the growth of particles suspended in turbulent flows''
January 2015 -- December 2019, 44 MSEK (4.67 M\$, as Co-I)

\item
Research Council of Norway (RCN), FRINATEK research grant
``Particle transport and clustering in turbulent flows''
231444, July 2014 -- June 2017, 7.25 MNOK
(1.18 M\$, as PI)
% http://www.nordita.org/~brandenb/TurboPart/

\item
VR breakthrough research grant, ``Formation of active regions in the Sun''
2012-5797, January 2013 -- December 2016, 4.2 MSEK
(0.63 M\$, as PI)

\item
VR project grant, ``Turbulent dynamo simulation in a spherical shell segment''
621-2011-5076, January 2012 -- December 2014, 1.65 MSEK
(0.25 M\$, as PI)

\item
ERC Advanced Grant, ``Astrophysical Dynamos'' No 227952,\\
February 2009 -- January 2014, 2.22 MEuro
(2.8 M\$, as PI)
% 2009, 10, 11, 12, 13, Jan 14
% 19,600 SEK
% http://www.nordita.org/~brandenb/AstroDyn/

\item
PPARC Research Grant, ``Accretion Discs and Jets'' PPA/G/S/1997/00284,\\
1998 -- 2001, 270 kGBP
(0.42 M\$, as PI)

\end{itemize}

\section*{Fields of research}

%Solar physics, 
Astrophysical fluid dynamics, 
with emphasis on dynamo and turbulence theories;
astrobiology, with emphasis on homochirality.
Particular interests: solar and stellar activity, helioseismology,
convection, differential rotation, galactic turbulence and magnetism,
accretion discs, fractals in turbulence, relativistic hydrodynamics,
early universe, relic gravitational waves, magnetospheric physics.
\\

\section*{Organization of conferences and programs}

\begin{tabbing}
Jan~\= 2024 ~~\= Program on Turbulence in Astrophysical Environments (KITP, Santa Barbara) \\
Aug~\= 2022 ~~\= Program on Magnetic field evolution in low density or strongly stratified plasmas (Stockholm) \\
Aug~\= 2019 ~~\= Program on Gravitational Waves from the Early-Universe (Stockholm) \\
Jun~\= 2018 ~~\= 14th Pencil Code User Meeting (Boulder) \\
Jun~\= 2015 ~~\= Program on Origin, Evolution, and Signatures of Cosmological Magnetic Fields (Stockholm)\\
Oct~\= 2012 ~~\= 12th European Workshop on Astrobiology (Stockholm)\\
Aug~\= 2011 ~~\= Program on Dynamo, Dynamical Systems and Topology (Stockholm)\\
May~\= 2011   \> Program on Predictability + School on Data Assimilation (Stockholm) \\
Feb~\= 2011   \> R\"adlerFest: $\alpha$ effect and beyond (Stockholm) \\
May~\= 2010   \> Program on Turbulent combustion (Stockholm) \\
Sep~\= 2009   \> Program on Solar and Stellar Cycles (Stockholm) \\
Mar~\= 2008   \> Program on Turbulence and Dynamos (Stockholm) \\
Feb~\> 2008   \> Program on the Origins of Homochirality (Stockholm) \\
Nov~\> 2007   \> Joint Nordic and SwAN Astrobiology meeting (Stockholm) \\
Aug~\> 2007   \> 3rd Pencil Code User Meeting (Stockholm) \\
May~\> 2007   \> New Trends in Radiation Hydrodynamics (Stockholm) \\
Jan~\> 2006   \> NorFA Winter School on Astrobiology (Levitunturi, Finnish Lapland)\\
Jul~\> 2005   \> Nordita Master Class in Physics (Hiller\o d)\\
Jan~\> 2005   \> Astrobiology and Origins of Life (Copenhagen)\\
Jan~\> 2005   \> Meeting on Nordic Science Outreach (Copenhagen)\\
Sep~\> 2004   \> Cosmic Ray Dynamics: from Turbulent to Galactic Scale Magnetic Fields (Copenhagen)\\
Aug~\> 2004   \> Astrobiological Problems for Physicists and Biologists (Turku, Finland)\\
Jan~\> 2004   \> Astrobiological Problems for Physicists (Copenhagen)\\
Jul~\> 2002   \> Nordita Master Class in Physics (Hiller\o d)\\
Jul~\> 2001   \> Nordita Master Class in Physics (Hiller\o d)\\
Mar~\> 2001   \> Dynamos in the Laboratory, Computer, and the Sky (Copenhagen)\\
Jul~\> 2000   \> Nordita Master Class in Physics (Copenhagen)\\
Jan~\> 2000   \> Physics of Accretion and Associated Outflows (Copenhagen)\\
May~\> 1997   \> UK-MHD meeting (Newcastle, England)\\
Feb~\> 1996   \> NorFA Winter School on Magnetic fields in Space and Astrophysics (Levitunturi, Finnish Lapland)\\
\end{tabbing}
\vspace{-8mm}
 
\section*{Invited participation in research programs}

\begin{tabbing}
Nov~\= 2022 ~~\= Frontiers in dynamo theory: from the Earth to the stars, 3 weeks (Cambridge) \\
Jun~\> 2019 ~~\> Turbulent Life of Cosmic Baryons, 3 weeks (Aspen) \\
Feb~\> 2011 ~~\> Turbulence Theory, 1 month (Santa Barbara) \\
Jun~\> 2008 ~~\> Dynamo Theory, 1.5 month (Santa Barbara) \\
Nov~\> 2007 ~~\> Star Formation through Cosmic Time, 1 month (Santa Barbara) \\
Sep~\> 2004 ~~\> Magnetohydrodynamics of Stellar Interiors, 3 months (Cambridge) \\
Jun~\> 2002 ~~\> Dynamo Theory, 3 weeks (Aspen) \\
Jan~\> 2002 ~~\> Solar Magnetism and Related Astrophysics, 3 months (Santa Barbara) \\
Apr~\> 2000 ~~\> Astrophysical Turbulence, 3 months (Santa Barbara) \\
Jan~\> 1998 ~~\> Dynamics of Astrophysical Discs, 3 months (Cambridge) \\
Aug~\> 1992 ~~\> Dynamo Theory, 3 months (Cambridge) \\
\end{tabbing}
\vspace{-8mm}

%Thesis examinations
% Vanhanen 1997 Oulu
% Kerr's college's student, Imperial College London
% Campbell's student 1997 Newcastle
% Paul Dellar, 1998 Cambridge (DAMTP)
% Bertil Dorch 1998 Copenhagen
% Erik Aurell's student, 2001 Stockholm
% XXX Copenhagen (2002)
% Hamburg guy, Potsdam (2000)
% Earth dynamo Potsdam (2007)
% Luis Borgonovo, Gamma rays bursts, 2007 Stockholm
% xxx, Oct 2007 Helsinki
% Luca Naso, 2009 Trieste
% Bidya Karak, 2012, Bangalore
% Nishant Singh, 2012, Bangalore
% ??, 2013 Cape Town, South Africa
% Maire (Linkmann), 2018 Glasgow, UK

%\section*{Scholarships}

%Cultural Foundation of Finland, 1988, 1990\\ 
%Deutscher Akademischer Austauschdienst, 1986-1987\\ 
%Finnish ministry of education, 1986-1987\\ 
%Heikki \& Hilma Honkanen Foundation, 1989\\ 
%Societas Scientiarum Fennica (Magnus Ehrnrooth), 1988, 1990\\ 
%Vilho, Yrj\"o \& Kalle V\"ais\"al\"a stipendium, 1988, 1989\\ 

%Cultural Foundation of Finland, 1990\\ 
%Societas Scientiarum Fennica (Magnus Ehrnrooth), 1990\\ 
%Heikki \& Hilma Honkanen Foundation, 1989\\ 
%Vilho, Yrj\"o \& Kalle V\"ais\"al\"a stipendium, 1989\\ 
%Cultural Foundation of Finland, 1988\\ 
%Societas Scientiarum Fennica (Magnus Ehrnrooth), 1988\\ 
%Vilho, Yrj\"o \& Kalle V\"ais\"al\"a stipendium, 1988\\ 
%Finnish ministry of education, 1986-1987\\ 
%Deutscher Akademischer Austauschdienst, 1986-1987\\ 

\section*{Memberships}

Finnish Physical Society (since 1988) \\
International Astronomical Union (since 1990) \\
American Physical Society (since 1996) \\
European Astrobiology Network Association (since 2005) \\
European Physical Society (since 2011) \\
Member of the Royal Swedish Academy of Sciences (Astronomy and Space Science, 2014)\\

\section*{Other academic activities}

I am frequently consulted as a referee for the following journals:
Astrophysical Journal, Astronomy \& Astrophysics, Geophysical and
Astrophysical Fluid Dynamics, Journal of Fluid Mechanics,
Monthly Notices of the Royal Astronomical Society,
Physical Review (PRL, PRD, and PRE), Physics of Plasmas,
Journal of Computational Physics,
Journal of Cosmology \& Astroparticle Physics,
New Journal of Physics.
On the average my load on reviewing papers is 3 per month.

I am also regularly asked to review research proposals
(NSF, PPARC, DFG, SA, ERC, NRC, VR, Hong Kong, Portugal, Austria)
and to examine PhD theses (Finland, Sweden, Denmark, England, Germany,
France, India, South Africa, USA).
I have been an external panel member for the selection of post-docs
(Finnish Academy; suomen akatemia, SA),
major research grants (Deutsche Forschungsgesellschaft, DFG),
and observing time (European Southern Observatories, ESO).

%As external examiner on PhD theses
%1996 Harri Vanhala (Astronomy, Oulu/Finland) Lic
%1997 Bertil Dorch (Astronomy, Copenhagen/Denmark)
%1998 Paul J. Dellar (DAMTP, Cambridge/UK)
%1998 M. Heritage (Imperial Coll., London/UK)
%2001 Katia Ferriere (Observatoy, Toulouse/France) Habil
%2007 Tiera Laitinen (Finn. Met. Inst., Helsinki/Finland)
%2008 Lisa Rosenqvist (Space Phys., Uppsala/Sweden)
%2014 Daniel Carrera (Astronomy, Lund/Sweden) Lic
%2018 Sofie Liljegren (Astronomy, Uppsala/Sweden)
%2020 Lavail (Kochukov student)(Astronomy, Uppsala/Sweden)

\section*{Administrative experience}

\begin{tabbing}
2021--present ~~\= Deputy director of Nordita \\
2010--present ~~\> Editorial Board Member of Astron. Nachr.\\
2010--2015      \> Deputy director of Nordita \\
2008--2015      \> Chairman of the Swedish Astrobiology Network \\
2007--2009      \> Member of the AlbaNova/Nordita colloquium committee\\
2001            \> Director of the Helmholtz Summer School, Potsdam \\
2000--2002      \> Director of the Nordita Master Class\\
\end{tabbing}
\vspace{-8mm}

\section*{Other merits}

Together with Wolfgang Dobler, I initiated the {\sc Pencil Code} in 2001
as a public domain program for solving partial differential equations
on massively parallel supercomputers.
During 2008--2015 it was hosted through the subversion repository on
Google Code (http://pencil-code.googlecode.com), and
since 2015 it is hosted through \url{https://github.com/pencil-code}.
It has been used for currently over 600 scientific publications; see Ref.D.5
in my full list of publications.

\section*{Public Outreach Experience}

\begin{tabbing}
2019 ~~\= Efter big bang: produktionen av gravitationsv{\aa}gor\\
       \> (Guest lecture at on Oct.\ 31, ABF-huset, Sveav\"agen 41, Stockholm)\\
%2016 ~~\= Guided tours through the Laboratory for Atmospheric and Space Sciences\\
2014 ~~\= Article in Fysikaktuellt: S\"okandet efter en ny teori f\"or solfl\"ackar\\
2010 ~~\= Interview ``Cycles of the Sun'' (British Publishers)\\
       \> (\url{http://www.nordita.org/~brandenb/Solar_Activity_10.pdf})\\
2008   \> Podcast {\it Is All Life Left-Handed?}\\
       \> (\url{http://www.astrobio.net/amee/summer_2008/Radio/radio.php})\\
2005   \> Organizer of Meeting on Nordic Science Outreach (Copenhagen) \\
%http://www.nordita.dk/conference/Outreach05/
2005   \> Co-authored outreach articles with Anja Andersen (Kvant and BioZoom)\\
1990   \> Extended interview in Finnish Television (Prisma program, YLE)\\
\end{tabbing}
\vspace{-8mm}

\section*{Language skills}

\noindent
Native: German\\
Fluent: English and Finnish\\
Basic knowledge: Swedish\\

\section*{Hobbies}

Cycling, hiking, swimming.
Participated in the $3\km$ Vansbrosimningen
(\url{https://en.wikipedia.org/wiki/Vansbrosimningen})
% in 2024 and swam the $3\,$km in 1 hour and 10 minutes; see
%(\url{https://results.vansbrosimningen.se/search/414154})
%(start number 8361).
%The average across all ages is 1 hour and 12--18 minutes,
%depending on the variant.
and the $5\km$ G\"ota Kanalsimmet (\url{https://www.gotakanalsimmet.se/}).
%in 2024 and swam the $5\,$km in 2 hours and 20 minutes; see
%\url{https://my4.raceresult.com/296883/RRPublish/data/pdf?name=Online%7CFinal&contest=0&lang=en}.

\vspace{8mm}
%\newpage

\section*{Publications}
% $Id: curri.tex,v 1.2714 2025/01/31 12:43:18 brandenb Exp $

%\newcommand{\relevant}{~**}
\newcommand{\relevant}{}
\newcommand{\important}{*}
%\newcommand{\Brandenburg}{Brandenburg, A.}
%\newcommand{\Brandenburg}{{\bf Brandenburg, A.}}
%\newcommand{\Brandenburg}{\textcolor{blue}{Brandenburg, A.}}
%http://latexcolor.com/
\definecolor{battleshipgrey}{rgb}{0.52, 0.52, 0.51}
\definecolor{upforestgreen}{rgb}{0.0, 0.27, 0.13}
\definecolor{ao(english)}{rgb}{0.0, 0.5, 0.0}
\definecolor{airforceblue}{rgb}{0.36, 0.54, 0.66}
\definecolor{aurometalsaurus}{rgb}{0.43, 0.5, 0.5}
\definecolor{amber(sae/ece)}{rgb}{1.0, 0.49, 0.0}
\definecolor{azure(colorwheel)}{rgb}{0.0, 0.5, 1.0}
\definecolor{britishracinggreen}{rgb}{0.0, 0.26, 0.15}
%\newcommand{\Brandenburg}{\textcolor{battleshipgrey}{\bf Brandenburg, A.}}
%\newcommand{\Brandenburg}{\textcolor{ao(english)}{Brandenburg, A.}} % <==
%\newcommand{\Brandenburg}{\textcolor{aurometalsaurus}{\bf Brandenburg, A.}}
\newcommand{\Brandenburg}{\textcolor{upforestgreen}{Brandenburg, A.}}
%\newcommand{\Brandenburg}{{\bf Brandenburg, A.}}

%\newpage
%
%  r e f
%
%\noindent{\bf Publications}
%\section*{Publications}

%\noindent{\it A. Publications in refereed journals}
\subsection*{A. Publications in refereed journals} %A A A
{\small(Highly quoted papers are denoted by an asterisk)}
\\
\\

\noindent{\it Submitted:}

\begin{itemize}

%Roberts Test
%Solar-MRI

%pumping with memory effect and boundaries, DNS of Roberts II with boundaries (==> Gotska)
%Vasil+24 paper (==> Gotska) %Solar-MRI
%Kandu/Ramkishor radiation: Radiation_Damping
%Tina review
%Tina Andrew Source-CME-GW
%Tanmay 2-fluid
%Eva E+B (striated structures) (==> Gotska)
%Sambit Battery
%forced chiral MHD
%mode-coupling
%Kyrylo: LOFAR-SSfluct
%Higgsless
%Lane-Emden stars --> paper

%fast Roberts II flow with variable phases
%Voyager data
%nl6/tex/notes/projects/master
%tex/rei/incoh
%tex/hydro/Deardorff
%tex/palvi/ionization/idl/notes.tex
%pencil-code/axel/decay/icascade/idl/notes.tex
%tex/savita/gmodes/idl/notes_slope.tex
%kandu: triple helicity
%yutong/rel

%seeding through outflows
%raedler effect
%Dynamo action from galaxy cluster mergers

%drag reduction
%tex/oconnor/flash
%tex/mhd/idealMHDdyn
%stably stratified axisymmetric transport (entropy strat)
%shell model for Hall cascade
%tex/notes/GW_decay_compens --> to model this
%tex/mhd/supersonic
%acoustic bottleneck effect
%anelastic cooling instability
%phototurbulence
%photo-overshoot
%tex/mhd/ADdecay
%tex/nishant/lsm_nolsd
%tex/petri/nsslconv
%tex/simon/bifurc_dynalp
%idl/appl/voyager
%tex/illa/HankelSpot
%DisplacementCurrent
%tex/sarah/RotShearDNSsph
%tex/mhd/hall_vs_Prandtl
% 2.5D Hosking

\item[{468.}~]
\Brandenburg, \& Scannapieco, E.\sapj{2025}
{Magnetically-assisted vorticity production in decaying acoustic turbulence}
(\arxiv{2501.18525}\adsT{2025arXiv250118525B}\httpT{http://norlx65.nordita.org/~brandenb/projects/Acoustic}\ownT{2025/Bran+Scan25})

\item[{467.}~]
Rogachevskii, I., Kleeorin, N., \& \Brandenburg\sapj{2025}
{Theory of the kinetic helicity effect on turbulent diffusion of magnetic and scalar fields}
(\arxiv{2501.13807}\adsT{2025arXiv250113807R}\httpT{http://norlx65.nordita.org/~brandenb/projects/Scalar}\ownT{2025/Roga+Klee+Bran+25})

\item[{466.}~]
\Brandenburg, Yi, L., \& Wu, X.\sjpp{2025}
{Inverse cascade from helical and nonhelical decaying columnar magnetic fields}
(\arxiv{2501.12200}\adsT{2025arXiv250112200B}\httpT{http://norlx65.nordita.org/~brandenb/projects/Roberts-Decay/}\ownT{2025/Bran+Yi+Wu25})

\item[{465.}~]
\Brandenburg, K\"apyl\"a, P. J., Rogachevskii, I., \& Yokoi, N.\sapj{2025}
{Helicity effect on turbulent passive and active scalar diffusivities}
(\arxiv{2501.08879}\adsT{2025arXiv250108879B}\httpT{http://norlx65.nordita.org/~brandenb/projects/Scalar}\ownT{2025/Bran+25})

\item[{464.}~]
\Brandenburg, \& Vishniac, E. T.\sapj{2025}
{Magnetic helicity fluxes in dynamos from rotating inhomogeneous turbulence}
(\arxiv{2412.17402}\adsT{2024arXiv241217402B}\httpT{http://norlx65.nordita.org/~brandenb/projects/Omega-Gradu}\ownT{2025/Bran+Vish25})

\item[{463.}~]
Sharma, R., \Brandenburg, Subramanian, K., \& Vikman, A.\sjcap{2025}
{Lattice simulations of axion-U(1) inflation: gravitational waves, magnetic fields, and black holes}
(\arxiv{2411.04854}\adsT{2024arXiv241104854S}\ownT{2025/Shar+Bran+Subr+Vikm25})

\item[{462.}~]
Neronov, A., Vazza, F., \Brandenburg, Caprini, C.\sana{2025}
{Magnetic fields in a gamma-ray beam as a model of Porphyrion}
(\arxiv{2411.01640}\adsT{2024arXiv241101640N}\ownT{2025/Nero+Vazz+Bran+Capr25})

\end{itemize}

\noindent{\it In press:}
\begin{itemize}

\item[{461.}~]
Vachaspati, T., \& \Brandenburg\pprd{2025}
{Spectra of magnetic fields from electroweak symmetry breaking}
(\arxiv{2412.00641}\adsT{2024arXiv241200641V}\httpT{http://norlx65.nordita.org/~brandenb/projects/EW-B-statistics}\ownT{2025/Vach+Bran25})

\end{itemize}

%\noindent{\it DOI number available:}
%\begin{itemize}

%\end{itemize}

\noindent{\it Published:}
\begin{itemize}

\item[{460.}~]
Dehman, C., \& \Brandenburg\yanaN{2025}{694}{A39}
{Reality of inverse cascading in neutron star crusts}
\arxivB{2408.08819}\adsT{2024arXiv240808819D}\doiT{10.1051/0004-6361/202451904}\ownE{2025/Dehm+Bran25}

\item[{459.}~]
\Brandenburg, \& Banerjee, A.\yjppN{2025}{91}{E5}
{Turbulent magnetic decay controlled by two conserved quantities}
\arxivB{2406.11798}\adsT{2024arXiv240611798B}\doiT{10.1017/S0022377824001508}\httpT{http://norlx65.nordita.org/~brandenb/projects/Two-Conserved}\ownE{2025/Bran+Bane25}

\item[{458.}~]
\Brandenburg, Iarygina, O., Sfakianakis, E. I., \& Sharma, R.\yjcap{2024}{12}{057}
{Magnetogenesis from axion-SU(2) inflation}
\arxivB{2408.17413}\adsT{2024arXiv240817413B}\doiT{10.1088/1475-7516/2024/12/057}\ownE{2024/Bran+Iary+Sfak+Shar24}

\item[{457.}~]
Mtchedlidze, S., Dom\'inguez-Fern\'andez, P., Du, X., Carretti, E., Vazza, F., O'Sullivan, S. P., \Brandenburg, \& Kahniashvili, T.\yapjN{2024}{977}{128}
{Intergalactic medium rotation measure of primordial magnetic fields}
\arxivB{2406.16230}\adsT{2024arXiv240616230M}\doiT{10.3847/1538-4357/ad8dc5}\ownE{2024/Mtch+24}

\item[{456.}~]
Schober, J., Rogachevskii, I., \& \Brandenburg\yprdN{2024}{110}{043515}
{Efficiency of dynamos from the autonomous generation of a chiral asymmetry}
\arxivB{2404.07845}\adsT{2024arXiv240407845S}\doiT{10.1103/PhysRevD.110.043515}\ownE{2024/Scho+Roga+Bran24b}

\item[{455.}~]
\Brandenburg, Neronov, A., \& Vazza, F.\yanaN{2024}{687}{A186}
{Resistively controlled primordial magnetic turbulence decay}
\arxivB{2401.08569}\adsT{2024A\%26A...687A.186B}\doiT{10.1051/0004-6361/202449267}\httpT{http://norlx65.nordita.org/~brandenb/projects/Reconnection-Decay/}\ownT{2024/Bran+24}\adE{2024/Bran+24_ad}

\item[{454.}~]
Iarygina, O., Sfakianakis, E. I., Sharma, R. \& \Brandenburg\yjcap{2024}{04}{018}
{Backreaction of axion-SU(2) dynamics during inflation}
\arxivB{2311.07557}\adsT{2024JCAP...04..018I}\doiT{10.1088/1475-7516/2024/04/018}\httpT{http://norlx65.nordita.org/~brandenb/projects/axion-SU-2-inflation}\ownE{2024/Iary+24}

\item[{453.}~]
\Brandenburg, Clarke, E., Kahniashvili, T., Long, A. J., \& Sun, G.\yprdN{2024}{109}{043534}
{Relic gravitational waves from the chiral plasma instability in the standard cosmological model}
\arxivB{2307.09385}\adsT{2024PhRvD.109d3534B}\doiT{10.1103/PhysRevD.109.043534}\httpT{http://norlx65.nordita.org/~brandenb/projects/GWfromSM/}\ownE{2024/BCKLS24}

\item[{452.}~]
Schober, J., Rogachevskii, I., \& \Brandenburg\yprlN{2024}{132}{065101}
{Chiral anomaly and dynamos from inhomogeneous chemical potential fluctuations}
\arxivB{2307.15118}\adsT{2024PhRvL.132f5101S}\httpT{https://c4science.ch/diffusion/10359/browse/master/2023/Schober_Rogachevskii_Brandenburg_PRL/}\ownE{2024/Scho+Roga+Bran24}

\item[{451.}~]
Sharma, R., Dahl, J., Brandenburg, A., \& Hindmarsh, M.\yjcap{2023}{12}{042}
{Shallow relic gravitational wave spectrum with acoustic peak}
\arxivB{2308.12916}\adsT{2023JCAP...12..042S}\doiT{10.1088/1475-7516/2023/12/042}\httpT{http://norlx65.nordita.org/~brandenb/projects/ShallowGW}\ownE{2023/Shar+Dahl+Bran+Hind23}

\item[{450.}~]
\Brandenburg, Sharma, R., \& Vachaspati, T.\yjppN{2023}{89}{905890606}
{Inverse cascading for initial MHD turbulence spectra between Saffman and Batchelor}
\arxivB{2307.04602}\adsT{2023JPlPh..89f9006B}\doiT{10.1017/S0022377823001253}\httpT{http://norlx65.nordita.org/~brandenb/projects/Cubic-Hosking/}\ownE{2023/Bran+Shar+Vach23}

\item[{449.}~]
Carenza, P., Sharma, R., Marsh, M. C. D., \Brandenburg, M\"uller, E.\yprdN{2023}{108}{103029}
{Magnetohydrodynamics predicts heavy-tailed distributions of axion-photon conversion}
\arxivB{2208.04333}\adsT{2023PhRvD.108j3029C}\doiT{10.1103/PhysRevD.108.103029}\httpT{http://norlx65.nordita.org/~brandenb/projects/ALP_MHD}\ownE{2023/Care+etal23}

\item[{448.}~]
\Brandenburg, Kamada, K., Mukaida, K., Schmitz, K., \& Schober, J.\yprdN{2023}{108}{063529}
{Chiral magnetohydrodynamics with zero total chirality}
\arxivB{2304.06612}\adsT{2023PhRvD.108f3529B}\doiT{10.1103/PhysRevD.108.063529}\httpT{http://norlx65.nordita.org/~brandenb/projects/ZeroTotalChirality/}\ownE{2023/Bran+23}

\item[{447.}~]
\Brandenburg, Elstner, D., Masada, Y., \& Pipin, V.\yjourN{2023}{Spa. Sci. Rev.}{219}{55}
{Turbulent processes and mean-field dynamo}
\arxivB{2303.12425}\adsT{2023SSRv..219...55B}\doiT{10.1007/s11214-023-00999-3}\httpT{http://norlx65.nordita.org/~brandenb/tmp/TurbProcs-MFT/}\ownE{2023/Bran+Elst+Masa+Pipi23}

\item[{446.}~]
\Brandenburg, \& Protiti, N. N.\yjourN{2023}{Entropy}{25}{1270}
{Electromagnetic conversion into kinetic and thermal energies}
\arxivB{2308.00662}\adsT{2023Entrp..25.1270B}\doiT{10.3390/e25091270}\httpT{http://norlx65.nordita.org/~brandenb/projects/EMconversion/}\ownE{2023/Bran+Prot23}

\item[{445.}~]
Mizerski, K. A., Yokoi, N., \& \Brandenburg\yjppN{2023}{89}{905890412}
{Cross-helicity effect on $\alpha$-type dynamo in non-equilibrium turbulence}
\arxivB{2303.01090}\adsT{2023JPlPh..89d9012M}\doiT{10.1017/S0022377823000545}\httpT{http://norlx65.nordita.org/~brandenb/projects/nonEquilibriumDynamo/}\ownE{2023/Mize+Yoko+Bran23}

\item[{444.}~]
Sarin, N., \Brandenburg, \& Haskell, B.\yapjlN{2023}{952}{L21}
{Confronting the neutron star population with inverse cascades}
\arxivB{2305.14347}\adsT{2023ApJ...952L..21S}\doiT{10.3847/2041-8213/ace363}\httpT{http://norlx65.nordita.org/~brandenb/projects/NSpop+InvCasc/}\ownE{2023/Sari+Bran+Hask23}

\item[{443.}~]
\Brandenburg, \& Ntormousi, E.\yjour{2023}{Annu. Rev. Astron. Astrophys.}{61}{561}
{606}{Galactic Dynamos}
\arxivB{2211.03476}\adsT{2023ARA\%26A..61..561B}\doiT{10.1146/annurev-astro-071221-052807}\httpT{http://www.nordita.org/~brandenb/projects/Galactic-Dynamos}\ownE{2023/Bran+Ntor23}

\item[{442.}~]
He, Y., Roper Pol, A., \& \Brandenburg\yjcap{2023}{06}{025}
{Modified propagation of gravitational waves from the early radiation era}
\arxivB{2212.06082}\adsT{2023JCAP...06..025H}\doiT{10.1088/1475-7516/2023/06/025}\httpT{http://www.nordita.org/~yutong/projects/general_modified_GWs}\ownE{2023/He+23}

\item[{441.}~]
\Brandenburg, \& Larsson, G.\yjourN{2023}{Atmosphere}{14}{932}
{Turbulence with magnetic helicity that is absent on average}
\arxivB{2305.08769}\adsT{2023Atmos..14..932B}\doiT{10.3390/atmos14060932}\httpT{http://norlx65.nordita.org/~brandenb/projects/Hosking-Shell/}\ownE{2023/Bran+Lars23}

\item[{440.}~]
\Brandenburg, Kamada, K., \& Schober, J.\yprr{2023}{5}{L022028}
{Decay law of magnetic turbulence with helicity balanced by chiral fermions}
\arxivB{2302.00512}\adsT{2023PhRvR...5b2028B}\doiT{10.1103/PhysRevResearch.5.L022028}\httpT{http://norlx65.nordita.org/~brandenb/projects/DecayWithChiral/}\ownT{2023/Bran+Kama+Scho23}\adE{2023/Bran+Kama+Scho23_ad}

\item[{439.}~]
\Brandenburg\yjppN{2023}{89}{175890101}
{Hosking integral in nonhelical Hall cascade}
\arxivB{2211.14197}\adsT{2023JPlPh..89a1701B}\doiT{10.1017/S0022377823000028}\httpT{http://www.nordita.org/~brandenb/projects/HallNonhel}\ownE{2023/Bran23}

\item[{438.}~]
Mtchedlidze, S., Dom\'inguez-Fern\'andez, P., Du, X., Schmidt, W., \Brandenburg, Niemeyer, J., \& Kahniashvili, T.\yapjN{2023}{944}{100}
{Inflationary and phase-transitional primordial magnetic fields in galaxy clusters}
\arxivB{2210.10183}\adsT{2023ApJ...944..100M}\doiT{10.3847/1538-4357/acb04d}\ownE{2023/Mtch+23}

\item[{437.}~]
\Brandenburg\yjourN{2023}{J. Phys. A: Math. Theor.}{56}{044002}
{Quadratic growth during the COVID-19 pandemic: merging hotspots and reinfections}
\arxivB{2206.15459}\adsT{2023JPhA...56d4002B}\doiT{10.1088/1751-8121/acb743}\httpT{http://norlx51.nordita.org/~brandenb/tmp/merge/}\ownE{2023/Bran23cor}

\item[{436.}~]
\Brandenburg, Rogachevskii, I., \& Schober, J.\ymn{2023}{518}{6367}
{6375}{Dissipative magnetic structures and scales in small-scale dynamos}
\arxivB{2209.08717}\adsT{2023MNRAS.518.6367B}\doiT{10.1093/mnras/stac3555}\httpT{http://www.nordita.org/~brandenb/projects/keta_vs_PrM}\ownT{2023/Bran+Roga+Scho23}\supE{2023/Bran+Roga+Scho23_supp}

\item[{435.}~]
\Brandenburg, Zhou, H., \& Sharma, R.\ymn{2023}{518}{3312}
{3325}{Batchelor, Saffman, and Kazantsev spectra in galactic small-scale dynamos}
\arxivB{2207.09414}\adsT{2023MNRAS.518.3312B}\doiT{10.1093/mnras/stac3217}\httpT{http://norlx65.nordita.org/~brandenb/projects/Kazantsev-Subinertial}\ownE{2023/Bran+Zhou+Shar23}

\item[{434.}~]
Sharma, R., \& \Brandenburg\yprdN{2022}{106}{103536}
{Low frequency tail of gravitational wave spectra from hydromagnetic turbulence}
\arxivB{2206.00055}\adsT{2022PhRvD.106j3536S}\doiT{10.1103/PhysRevD.106.103536}\httpT{http://norlx65.nordita.org/~brandenb/tmp/LowFreqTail/}\ownT{2022/Shar+Bran22}\supE{2022/Shar+Bran22_supp}

\item[{433.}~]
Zhou, H., Sharma, R., \& \Brandenburg\yjppN{2022}{88}{905880602}
{Scaling of the Hosking integral in decaying magnetically-dominated turbulence}
\arxivB{2206.07513}\adsT{2022JPlPh..88f9002Z}\doiT{10.1017/S002237782200109X}\httpT{http://www.nordita.org/~brandenb/projects/Saffman}\ownE{2022/Zhou+Shar+Bran22}

\item[{432.}~]
Sinha, S., Gupta, O., Singh, V., Lekshmi, B., Nandy, D., Mitra, D., Chatterjee, S., Bhattacharya, S., Chatterjee, S., Srivastava, N., \Brandenburg, \& Pal, S.\yapjN{2022}{935}{45}
{A comparative analysis of machine-learning models for solar flare forecasting: Identifying high-performing active region flare indicators}
\arxivB{2204.05910}\adsT{2022ApJ...935...45S}\doiT{10.3847/1538-4357/ac7955}\ownE{2022/Sinh_etal22}

\item[{431.}~]
Li, X.-Y., Mehlig, B., Svensson, G., \Brandenburg, \& Haugen, N. E. L.\yjour{2022}{J. Atmos. Sci.}{79}{1821}
{1835}{Collision fluctuations of lucky droplets with superdroplets}
\arxivB{1810.07475}\adsT{2022JAtS...79.1821L}\doiT{10.1175/JAS-D-20-0371.1}\httpT{http://norlx65.nordita.org/~brandenb/tmp/lucky/}\ownE{2022/Li_etal22}

\item[{430.}~]
K\"apyl\"a, M. J., Rheinhardt, M., \& \Brandenburg\yapjN{2022}{932}{8}
{Compressible test-field method and its application to shear dynamos}
\arxivB{2106.01107}\adsT{2022ApJ...932....8K}\doiT{10.3847/1538-4357/ac5b78}\httpT{http://www.nordita.org/~brandenb/projects/CompressibleTestfield}\ownE{2022/Kapy+Rhei+Bran22}

\item[{429.}~]
Kahniashvili, T., Clarke, E., Stepp, J., \& \Brandenburg\yprlN{2022}{128}{221301}
{Big bang nucleosynthesis limits and relic gravitational wave detection prospects}
\arxivB{2111.09541}\adsT{2022PhRvL.128v1301K}\doiT{10.1103/PhysRevLett.128.221301}\ownT{2022/Kahn_etal22}\supE{2022/Kahn_etal22_supp}

\item[{428.}~]
\Brandenburg, \& Ntormousi, E.\ymn{2022}{513}{2136}
{2151}{Dynamo effect in unstirred self-gravitating turbulence}
\arxivB{2112.03838}\adsT{2022MNRAS.513.2136B}\doiT{10.1093/mnras/stac982}\ownE{2022/Bran+Ntor22}

\item[{427.}~]
Mtchedlidze, S., Dom\'inguez-Fern\'andez, P., Du, X., \Brandenburg, Kahniashvili, T., O'Sullivan, S., Schmidt, W., \& Br\"uggen, M.\yapjN{2022}{929}{127}
{Evolution of primordial magnetic fields during large-scale structure formation}
\arxivB{2109.13520}\adsT{2022ApJ...929..127M}\doiT{10.3847/1538-4357/ac5960}\ownE{2022/Mtch+22}

\item[{426.}~]
Roper Pol, A., Mandal, A., \Brandenburg, \& Kahniashvili, T.\yjcap{2022}{04}{019}
{Polarization of gravitational waves from helical MHD turbulent sources}
\arxivB{2107.05356}\adsT{2022JCAP...04..019R}\doiT{10.1088/1475-7516/2022/04/019}\ownE{2022/RopP+Mand+Bran+Kahn22}

\item[{425.}~]
Schober, J., Rogachevskii, I., \& \Brandenburg\yprlN{2022}{128}{065002}
{Production of a chiral magnetic anomaly with emerging turbulence and mean-field dynamo action}
\arxivB{2107.12945}\adsT{2022PhRvL.128f5002S}\doiT{10.1103/PhysRevLett.128.065002}\ownE{2022/Scho+Roga+Bran22a}

\item[{424.}~]
Schober, J., Rogachevskii, I., \& \Brandenburg\yprdN{2022}{105}{043507}
{Dynamo instabilities in plasmas with inhomogeneous chiral chemical potential}
\arxivB{2107.13028}\adsT{2022PhRvD.105d3507S}\doiT{10.1103/PhysRevD.105.043507}\ownE{2022/Scho+Roga+Bran22b}

\item[{423.}~]
Haugen, N. E. L., \Brandenburg, Sandin, C., \& Mattsson, L.\yjfmN{2022}{934}{A37}
{Spectral characterisation of inertial particle clustering in turbulence}
\arxivB{2105.01539}\adsT{2022JFM...934A..37H}\doiT{10.1017/jfm.2021.1143}\httpT{http://norlx65.nordita.org/~brandenb/projects/isoth_expwave}\ownE{2022/Haug+Bran+Sand+Matt22}

\item[{422.}~]
\Brandenburg, He, Y., \& Sharma, R.\yapjN{2021}{922}{192}
{Simulations of helical inflationary magnetogenesis and gravitational waves}
\arxivB{2107.12333}\adsT{2021ApJ...922..192B}\doiT{10.3847/1538-4357/ac20d9}\httpT{http://www.nordita.org/~brandenb/projects/HelicalInflationaryGW}\ownE{2021/Bran+He+Shar21}

\item[{421.}~]
\Brandenburg, \& Sharma, R.\yapjN{2021}{920}{26}
{Simulating relic gravitational waves from inflationary magnetogenesis}
\arxivB{2106.03857}\adsT{2021ApJ...920...26B}\doiT{10.3847/1538-4357/ac1599}\httpT{http://www.nordita.org/~brandenb/projects/InflationaryMagnetoGW}\ownE{2021/Bran+Shar21}

\item[{420.}~]
\Brandenburg, \& Das, U.\ypfN{2021}{33}{095125}
{Turbulent radiative diffusion and turbulent Newtonian cooling}
\arxivB{2010.07046}\adsT{2021PhFl...33i5125B}\doiT{10.1063/5.0065485 }\httpT{http://www.nordita.org/~brandenb/projects/TurbNewtonianCooling}\ownE{2021/Bran+Das21}

\item[{419.}~]
\Brandenburg, Clarke, E., He, Y., \& Kahniashvili, T.\yprdNS{2021}{104}{043513}
{Can we observe the QCD phase transition-generated gravitational waves through pulsar timing arrays?}
\arxivB{2102.12428}\adsT{2021PhRvD.104d3513B}\doiT{10.1103/PhysRevD.104.043513}\httpT{http://www.nordita.org/~brandenb/projects/GWfromQCD}\ownE{2021/Bran+Clar+He+Kahn21}

\item[{418.}~]
He, Y., \Brandenburg, \& Sinha, A.\yjourN{2021}{J. Cosmol. Astropart. Phys.}{07}{015}
{Spectrum of turbulence-sourced gravitational waves as a constraint on graviton mass}
\arxivB{2104.03192}\adsT{2021JCAP...07..015H}\httpT{http://norlx65.nordita.org/~brandenb/tmp/GravitonGW}\ownE{2021/He+Bran+Sinh21}

\item[{417.}~]
\Brandenburg, Gogoberidze, G., Kahniashvili, T., Mandal, S., \& Roper~Pol, A., \& Shenoy, N.\yjourN{2021}{Class. Quantum Grav.}{38}{145002}
{The scalar, vector, and tensor modes in gravitational wave turbulence simulations}
\arxivB{2103.01140}\adsT{2021CQGra..38n5002B}\doiT{10.1088/1361-6382/ac011c}\ownE{2021/BGKMRPS21}

\item[{416.}~]
\Brandenburg, He, Y., Kahniashvili, T., Rheinhardt, M., \& Schober, J.\yapjN{2021}{911}{110}
{Gravitational waves from the chiral magnetic effect}
\arxivB{2101.08178}\adsT{2021ApJ...911..110B}\doiT{10.3847/1538-4357/abe4d7}\httpT{http://www.nordita.org/~brandenb/projects/GWfromCME}\ownE{2021/Bran_etal21}

\item[{415.}~]
Blanco, N., Stafford, K., Lavoie, M.-C., Brandenburg, A., G{\'o}rna, M. W., \& Merski, M.\yjourN{2021}{Epidemiology and Infection}{149}{e80}
{A simple model for the total number of SARS-CoV-2 infections on a national level}
\arxivB{2007.02712}\adsT{2020arXiv200702712B}\doiT{10.1017/S0950268821000649}\ownE{2021/Blan+21}
%https://www.hindawi.com/journals/jmath/2022/7715078/
%https://www.sciencedirect.com/science/article/pii/S2213398421001615?via%3Dihub
%https://onlinelibrary.wiley.com/doi/10.1002/jmv.27173

\item[{414.}~]
Jakab, P., \& \Brandenburg\yanaN{2021}{647}{A18}
{The effect of a dynamo-generated field on the Parker wind}
\arxivB{2006.02971}\adsT{2021A\%26A...647A..18J}\doiT{10.1051/0004-6361/202038564}\httpT{http://www.nordita.org/~brandenb/projects/StellarWind/}\ownE{2021/Jaka+Bran21}
%SOLAR arXiv on 2021-03-04

\item[{413.}~]
Kahniashvili, T., \Brandenburg, Gogoberidze, G., Mandal, S., \& Roper~Pol, A.\yprr{2021}{3}{013193}
{Circular polarization of gravitational waves from early-universe helical turbulence}
\arxivB{2011.05556}\adsT{2021PhRvR...3a3193K}\doiT{10.1103/PhysRevResearch.3.013193}\httpT{http://www.nordita.org/~brandenb/projects/CircPol}\ownE{2021/Kahn_etal21}

\item[{412.}~]
Pencil Code Collaboration: \Brandenburg, Johansen, A., Bourdin, P. A., Dobler, W., Lyra, W., Rheinhardt, M., Bingert, S., Haugen, N. E. L., Mee, A., Gent, F., Babkovskaia, N., Yang, C.-C., Heinemann, T., Dintrans, B., Mitra, D., Candelaresi, S., Warnecke, J., K\"apyl\"a, P. J., Schreiber, A., Chatterjee, P., K\"apyl\"a, M. J., Li, X.-Y., Kr\"uger, J., Aarnes, J. R., Sarson, G. R., Oishi, J. S., Schober, J., Plasson, R., Sandin, C., Karchniwy, E., Rodrigues, L. F. S., Hubbard, A., Guerrero, G., Snodin, A., Losada, I. R., Pekkil\"a, J., \& Qian, C.\yjourN{2021}{J. Open Source Softw.}{6}{2807}
{The Pencil Code, a modular MPI code for partial differential equations and particles: multipurpose and multiuser-maintained}
\arxivB{2009.08231}\adsT{2021JOSS....6.2807C}\doiT{10.21105/joss.02807}\httpT{http://norlx51.nordita.org/~brandenb/pencil-code/paper}\ownE{2021/PCcollab21}

\item[{411.}~]
K\"apyl\"a, M. J., \'Alvarez Vizoso, J., Rheinhardt, M., \Brandenburg, \& Singh, N. K.\yapjN{2020}{905}{179}
{On the existence of shear--current effects in magnetized burgulence}
\arxivB{2006.05661}\adsT{2020ApJ...905..179K}\doiT{10.3847/1538-4357/abc1e8}\ownE{2020/MKapy_etal20}

\item[{410.}~]
Roper Pol, A., Mandal, S., \Brandenburg, Kahniashvili, T., \& Kosowsky, A.\yprdN{2020}{102}{083512}
{Numerical Simulations of Gravitational Waves from Early-Universe Turbulence}
\arxivB{1903.08585}\adsT{2020PhRvD.102h3512R}\doiT{10.1103/PhysRevD.102.083512}\httpT{http://norlx51.nordita.org/~brandenb/tmp/GW/}\ownE{2020/RopP_etal20b}

\item[{409.}~]
\Brandenburg\yjour{2020}{Infectious Disease Modelling}{5}{681}
{690}{Piecewise quadratic growth during the 2019 novel coronavirus epidemic}
\arxivB{2002.03638}\adsT{2020arXiv200203638B}\doiT{10.1016/j.idm.2020.08.014}\httpT{http://norlx51.nordita.org/~brandenb/tmp/2019-nCov/}\ownE{2020/Bran20cor}

\item[{408.}~]
\Brandenburg\yapjN{2020}{901}{18}
{Hall cascade with fractional magnetic helicity in neutron star crusts}
\arxivB{2006.12984}\adsT{2020ApJ...901...18B}\doiT{10.3847/1538-4357/abad92}\httpT{http://www.nordita.org/~brandenb/projects/HallCascade}\ownE{2020/Bran20}

\item[{407.}~]
Prabhu, A., \Brandenburg, K\"apyl\"a, M. J., \& Lagg, A.\yanaN{2020}{641}{A46}
{Helicity proxies from linear polarisation of solar active regions}
\arxivB{2001.10884}\adsT{2020A\%26A...641A..46P}\doiT{10.1051/0004-6361/202037614}\ownE{2020/Prab_etal20}
%SOLAR arXiv on 2020-09-12

\item[{406.}~]
Asplund, J., Johannesson, G., \& \Brandenburg\yapjN{2020}{898}{124}
{On the measurement of handedness in {\em Fermi} Large Area Telescope data}
\arxivB{2005.13065}\adsT{2020ApJ...898..124A}\doiT{10.3847/1538-4357/ab9744}\ownE{2020/Aspl+Joha+Bran20}

\item[{405.}~]
\Brandenburg, Durrer, R., Huang, Y., Kahniashvili, T., Mandal, S., \& Mukohyama S.\yprdN{2020}{102}{02353}
{Primordial magnetic helicity evolution with a homogeneous magnetic field from inflation}
\arxivB{2005.06449}\adsT{2020PhRvD.102b3536B}\doiT{10.1103/PhysRevD.102.023536}\httpT{http://norlx51.nordita.org/~brandenb/tmp/Mukohyama/}\ownE{2020/Bran_etal20Muk}

\item[{404.}~]
\Brandenburg, \& Furuya, R. S.\ymn{2020}{496}{4749}
{4759}{Application of a helicity proxy to edge-on galaxies}
\arxivB{2003.07284}\adsT{2020MNRAS.496.4749B}\doiT{10.1093/mnras/staa1795}\httpT{http://www.nordita.org/~brandenb/projects/EBpol_EdgeOn/}\ownE{2020/Bran+Furu20}

\item[{403.}~]
Pusztai, I., Juno, J., \Brandenburg, TenBarge, J. M., Hakim, A., Francisquez, M., \& Sundstr\"{o}m, A.\yprlN{2020}{124}{255102}
{Dynamo in weakly collisional nonmagnetized plasmas impeded by Landau damping of magnetic fields}
\arxivB{2001.11929}\adsT{2020PhRvL.124y5102P}\doiT{10.1103/PhysRevLett.124.255102}\httpT{http://www.nordita.org/~brandenb/projects/kineticDynamo/}\ownE{2020/Pusz_etal20}

\item[{402.}~]
\Brandenburg, \& Br\"uggen, M.\yapjlN{2020}{896}{L14}
{Hemispheric handedness in the Galactic synchrotron polarization foreground}
\arxivB{2003.14178}\adsT{2020ApJ...896L..14B}\doiT{10.3847/2041-8213/ab974a}\httpT{http://www.nordita.org/~brandenb/projects/GalHemiHandedness/}\ownE{2020/Bran+Brug20}

\item[{401.}~]
K\"apyl\"a, P. J., Rheinhardt, M., \Brandenburg, \& K\"apyl\"a, M. J.\yanaN{2020}{636}{A93}
{Turbulent viscosity and effective magnetic Prandtl number from simulations of isotropically forced turbulence}
\arxivB{1901.00787}\adsT{2020A\%26A...636A..93K}\doiT{10.1051/0004-6361/201935012}\ownE{2020/KRBK20}
%SOLAR arXiv on 2020-11-06

\item[{400.}~]
\Brandenburg, \& Boldyrev, S.\yapjN{2020}{892}{80}
{The turbulent stress spectrum in the inertial and subinertial ranges}
\arxivB{1912.07499}\adsT{2020ApJ...892...80B}\doiT{10.3847/1538-4357/ab77bd}\ownE{2020/Bran+Bold20}
%SOLAR arXiv on 2020-04-17

\item[{399.}~]
\Brandenburg, \& Chen, L.\yjppN{2020}{86}{905860110}
{The nature of mean-field generation in three classes of optimal dynamos}
\arxivB{1911.01712}\adsT{2020JPlPh..86a9010B}\doiT{10.1017/S0022377820000082}\httpT{http://norlx51.nordita.org/~brandenb/tmp/long_chen/}\ownE{2020/Bran+Chen20}
%SOLAR arXiv in April 2020

\item[{398.}~]
\Brandenburg, \& Scannapieco, E.\yapjN{2020}{889}{55}
{Magnetic helicity dissipation and production in an ideal MHD code}
\arxivB{1910.06074}\adsT{2020ApJ...889...55B}\doiT{10.3847/1538-4357/ab5e7f}\httpT{http://www.nordita.org/~brandenb/projects/Helicity_in_IdealMHDCode/}\ownE{2020/Bran+Scan20}
%SOLAR arXiv on 2020-03-18

\item[{397.}~]
Li, X.-Y., \Brandenburg, Svensson, G., Haugen, N. E. L., Mehlig, B., \& Rogachevskii, I.\yjour{2020}{J. Atmosph. Sci.}{77}{337}
{353}{Condensational and collisional growth of cloud droplets in a turbulent environment}
\arxivB{1807.11859}\adsT{2020JAtS...77..337L}\doiT{10.1175/JAS-D-19-0107.1}\ownE{2020/Li_etal20}

\item[{396.}~]
Singh, N. K., Raichur, H., K\"apyl\"a, M. J., Rheinhardt, M., \Brandenburg, \& K\"apyl\"a, P. J.\ygafd{2020}{114}{196}
{212}{{\em f}-mode strengthening from a localized bipolar subsurface magnetic field}
\arxivB{1808.08904}\adsT{2020GApFD.114..196S}\doiT{10.1080/03091929.2019.1653461}\httpT{http://norlx51.nordita.org/~brandenb/tmp/modes_pc/}\ownE{2020/Sing_etal20}
%SOLAR arXiv on 2020-05-23

\item[{395.}~]
\Brandenburg, \& Das, U.\ygafd{2020}{114}{162}
{195}{The time step constraint in radiation hydrodynamics}
\arxivB{1901.06385}\adsT{2020GApFD.114..162B}\doiT{10.1080/03091929.2019.1676894}\httpT{http://norlx51.nordita.org/~brandenb/tmp/tstep/}\ownE{2020/Bran+Das20}
%SOLAR arXiv on 2020-01-18

\item[{394.}~]
Roper Pol, A., \Brandenburg, Kahniashvili, T., Kosowsky, A., \& Mandal, S.\ygafd{2020}{114}{130}
{161}{The timestep constraint in solving the gravitational wave equations sourced by hydromagnetic turbulence}
\arxivB{1807.05479}\adsT{2020GApFD.114..130R}\httpT{http://norlx51.nordita.org/~brandenb/tmp/PencilGW/}\doiT{10.1080/03091929.2019.1653460}\ownE{2020/RopP_etal20}

\item[{393.}~]
Schober, J., \Brandenburg, \& Rogachevskii, I.\ygafd{2020}{114}{106}
{129}{Chiral fermion asymmetry in high-energy plasma simulations}
\arxivB{1808.06624}\adsT{2020GApFD.114..106S}\httpT{http://norlx51.nordita.org/~brandenb/tmp/ChiralMHD_PencilCode/}\doiT{10.1080/03091929.2019.1591393}\ownE{2020/SBR20}

\item[{392.}~]
Qian, C., Wang, C., Liu, J., \Brandenburg, Haugen, N. E. L., \& Liberman, M.\ygafd{2020}{114}{58}
{76}{Convergence properties of detonation simulations}
\arxivB{1902.03816}\adsT{2020GApFD.114...58Q}\doiT{10.1080/03091929.2019.1668382}\httpT{http://norlx51.nordita.org/~brandenb/tmp/detonation/}\ownE{2020/Qian_etal20}

\item[{391.}~]
K\"apyl\"a, P. J., Gent, F. A., Olspert, N., K\"apyl\"a, M. J., \& \Brandenburg\ygafd{2020}{114}{8}
{34}{Sensitivity to luminosity, centrifugal force, and boundary conditions in spherical shell convection}
\arxivB{1807.09309}\adsT{2020GApFD.114....8K}\httpT{http://norlx51.nordita.org/~brandenb/tmp/pencil-conv/}\doiT{10.1080/03091929.2019.1571586}\ownE{2020/Kapy_etal20}
%SOLAR arXiv on 2020-08-04

\item[{390.}~]
\Brandenburg\yapjN{2019}{883}{119}
{A global two-scale helicity proxy from $\pi$-ambiguous solar magnetic fields}
\arxivB{1906.03877}\adsT{2019ApJ...883..119B}\doiT{10.3847/1538-4357/ab3ec0}\httpT{http://norlx51.nordita.org/~brandenb/tmp/EBglobal/}\ownE{2019/Bra19Glo}
%http://www.nordita.org/~brandenb/Own_Papers/2019/Bra19Glo.pdf
%SOLAR arXiv on 2019-10-16

\item[{389.}~]
Gosain, S., \& \Brandenburg\yapjN{2019}{882}{80}
{Spectral magnetic helicity of solar active regions between 2006 and 2017}
\arxivB{1902.11273}\adsT{2019ApJ...882...80G}\httpT{http://www.nordita.org/~brandenb/projects/Hinode/}\ownE{2019/Gosa+Bran19}
%SOLAR arXiv on 2019-12-07

\item[{388.}~]
\Brandenburg\yoleb{2019}{49}{49}
{60}{The limited roles of autocatalysis and enantiomeric cross-inhibition in achieving homochirality in dilute systems}
\arxivB{1903.07855}\adsT{2019OLEB...49...49B}\doiT{10.1007/s11084-019-09579-4}\ownE{2019/Bran19ab}

\item[{387.}~]
\Brandenburg, \& Rempel, M.\yapjN{2019}{879}{57}
{Reversed dynamo at small scales and large magnetic Prandtl number}
\arxivB{1903.11869}\adsT{2019ApJ...879...57B}\doiT{10.3847/1538-4357/ab24bd}\ownE{2019/Bran+Remp19}
%SOLAR arXiv on 2019-08-25

\item[{386.}~]
\Brandenburg\ymn{2019}{487}{2673}
{2684}{Ambipolar diffusion in large Prandtl number turbulence}
\arxivB{1903.08976}\adsT{2019MNRAS.487.2673B}\doiT{10.1093/mnras/stz1509}\httpT{http://norlx51.nordita.org/~brandenb/tmp/AD/}\ownE{2019/Bran19AD}

\item[{385.}~]
K\"apyl\"a, P. J., Viviani, M., K\"apyl\"a, M. J., \Brandenburg, \& Spada, F.\ygafd{2019}{113}{149}
{183}{Effects of a subadiabatic layer on convection and dynamos in spherical wedge simulations}
\arxivB{1803.05898}\adsT{2019GApFD.113..149K}\doiT{10.1080/03091929.2019.1571584}\ownE{2019/KVKBS19}
%SOLAR

\item[{384.}~]
Schober, J., \Brandenburg, Rogachevskii, I., \& Kleeorin, N.\ygafd{2019}{113}{107}
{130}{Energetics of turbulence generated by chiral MHD dynamos}
\arxivB{1803.06350}\adsT{2019GApFD.113..107S}\doiT{10.1080/03091929.2018.1515313}\ownE{2019/SBRK19}
%AB: Hale ok

\item[{383.}~]
\Brandenburg, Kahniashvili, T., Mandal, S., Roper Pol, A., Tevzadze, A. G., \& Vachaspati, T.\yprf{2019}{4}{024608}
{Dynamo effect in decaying helical turbulence}
\arxivB{1710.01628}\adsT{2019PhRvF...4b4608B}\doiT{10.1103/PhysRevFluids.4.024608}\ownT{2019/Bran_etal19}\adE{2019/Bran_etal19_ad}
%AB: Hale ok

\item[{382.}~]
Li, X.-Y., Svensson, G., \Brandenburg, \& Haugen, N. E. L.\yjour{2019}{Atmosph. Chem. Phys.}{19}{639}
{648}{Cloud droplet growth due to supersaturation fluctuations in stratiform clouds}
\arxivB{1806.10529}\adsT{2019ACP....19..639L}\httpT{http://norlx51.nordita.org/~brandenb/tmp/supersat/}\doiT{10.5194/acp-2018-644}\ownE{2019/Li_etal19}

\item[{381.}~]
Bracco, A., Candelaresi, S., Del Sordo, F., \& \Brandenburg\yanaN{2019}{621}{A97}
{Is there a left-handed magnetic field in the solar neighborhood? Exploring helical magnetic fields in the interstellar medium through dust polarization power spectra}
\arxivB{1807.10188}\adsT{2019A\%26A...621A..97B}\doiT{10.1051/0004-6361/201833961}\ownE{2019/Brac_etal19}

\item[{380.}~]
\Brandenburg, Bracco, A., Kahniashvili, T., Mandal, S., Roper Pol, A., Petrie, G. J. D., \& Singh, N. K.\yapjN{2019}{870}{87}
{{\em E} and {\em B} polarizations from inhomogeneous and solar surface turbulence}
\arxivB{1807.11457}\adsT{2019ApJ...870...87B}\doiT{10.3847/1538-4357/aaf383}\ownE{2019/Bran_etal19EB}
%http://www.nordita.org/~brandenb/Own_Papers/2019/Bran_etal19EB.pdf
%SOLAR arXiv on 2019-04-25

\item[{379.}~]
Losada, I. R., Warnecke, J., \Brandenburg, Kleeorin, N., \& Rogachevskii, I.\yanaN{2019}{621}{A61}
{Magnetic bipoles in rotating turbulence with coronal envelope}
\arxivB{1803.04446}\adsT{2019A\%26A...621A..61L}\doiT{10.1051/0004-6361/201833018}\ownE{2019/LWBKR19}
%SOLAR

\item[{378.}~]
\Brandenburg\yan{2018}{339}{631}
{640}{Magnetic helicity and fluxes in an inhomogeneous alpha squared dynamo}
\arxivB{1901.07552}\adsT{2018AN....339..631B}\doiT{10.1002/asna.201913604}\ownE{2018/Bran18AN}
%SOLAR arXiv on 2019-03-02

\item[{377.}~]
\Brandenburg, \& Oughton, S.\yan{2018}{339}{641}
{646}{Cross-helically forced and decaying hydromagnetic turbulence}
\arxivB{1901.05875}\adsT{2018AN....339..641B}\doiT{10.1002/asna.201913602}\ownE{2018/Bran+Ough18}

\item[{376.}~]
Bourdin, Ph.-A., \& \Brandenburg\yapjN{2018}{869}{3}
{Magnetic helicity from multipolar regions on the solar surface}
\arxivB{1804.04160}\adsT{2018ApJ...869....3B}\doiT{10.3847/1538-4357/aae97f}\ownE{2018/Bour+Bran18}
%SOLAR arXiv on 2019-06-13

\item[{375.}~]
Bourdin, Ph.-A., Singh, N. K., \& \Brandenburg\yapjN{2018}{869}{2}
{Magnetic helicity reversal in the corona at small plasma beta}
\arxivB{1804.04153}\adsT{2018ApJ...869....2B}\doiT{10.3847/1538-4357/aae97a}\ownE{2018/Bour_etal18}
%SOLAR

\item[{374.}~]
Rogachevskii, I., Kleeorin, N., \& \Brandenburg\yjourN{2018}{J. Plasma Phys.}{84}{735840502}
{Compressibility effects in turbulent MHD and passive scalar transport: mean-field theory}
\arxivB{1801.01804}\adsT{2018JPlPh..84e7302R}\doiT{10.1017/S0022377818000983}\ownE{2018/Roga+Klee+Bran18}
%NORDITA-2017-140

\item[{373.}~]
Li, X.-Y., \Brandenburg, Svensson, G., Haugen, N. E. L., Mehlig, B., \& Rogachevskii, I.\yjour{2018}{J. Atmosph. Sci.}{75}{3469}
{3487}{Effect of turbulence on collisional growth of cloud droplets}
\arxivB{1711.10062}\adsT{2018JAtS...75.3469L}\doiT{10.1175/JAS-D-18-0081.1}\ownE{2018/Li_etal18}
%AB: Hale ok-

\item[{372.}~]
Viviani, M., Warnecke, J., K\"apyl\"a, M. J., K\"apyl\"a, P. J., Olspert, N., Cole-Kodikara, E. M., Lehtinen, J. J., \& \Brandenburg\yanaN{2018}{616}{A160}
{Transition from axi- to nonaxisymmetric dynamo modes in spherical convection models of solar-like stars}
\arxivB{1710.10222}\adsT{2018A\%26A...616A.160V}\doiT{10.1051/0004-6361/201732191}\ownE{2018/Vivi_etal18}
%AB: Hale ok
%SOLAR

\item[{371.}~]
\Brandenburg\yjourN{2018}{J. Plasma Phys.}{84}{735840404}
{Advances in mean-field dynamo theory and applications to astrophysical turbulence}
\arxivB{1801.05384}\adsT{2018JPlPh..84d7304B}\doiT{10.1017/S0022377818000806}\ownE{2018/Bran18}
%solar preprints arxive submitted on Sep 5, 2018
%AB: Hale ok
%SOLAR

\item[{370.}~]
Singh, N. K., K\"apyl\"a, M. J., \Brandenburg, K\"apyl\"a, P. J., Lagg, A., \& Virtanen, I.\yapjN{2018}{863}{182}
{Bihelical spectrum of solar magnetic helicity and its evolution}
\arxivB{1804.04994}\adsT{2018ApJ...863..182S}\doiT{10.3847/1538-4357/aad0f2}\ownE{2018/Sing_etal18}
%AB: Hale ok
%SOLAR

\item[{369.}~]
\Brandenburg, Durrer, R., Kahniashvili, T., Mandal, S., \& Yin, W. W.\yjourN{2018}{J. Cosmol. Astropart. Phys.}{08}{034}
{Statistical properties of scale-invariant helical magnetic fields and applications to cosmology}
\arxivB{1804.01177}\adsT{2018JCAP...08..034B}\doiT{10.1088/1475-7516/2018/08/034}\ownE{2018/BDKMY18}

\item[{368.}~]
Zhang, H., \& \Brandenburg\yapjlN{2018}{862}{L17}
{Solar kinetic energy and cross helicity spectra}
\arxivB{1804.10321}\adsT{2018ApJ...862L..17Z}\doiT{10.3847/2041-8213/aad337}\ownE{2018/Zhan+Bran18}
%solar preprints arxive submitted on Aug 3, 2018
%AB: Hale ok
%SOLAR

\item[{367.}~]
\Brandenburg, Haugen, N. E. L., Li, X.-Y., \& Subramanian, K.\ymn{2018}{479}{2827}
{2833}{Varying the forcing scale in low Prandtl number dynamos}
\arxivB{1805.01249}\adsT{2018MNRAS.479.2827B}\doiT{10.1093/mnras/sty1570}\ownE{2018/Bran_Haug_Li_Subr18}
%AB: Hale ok
%SOLAR

\item[{366.}~]
K\"apyl\"a, P. J., K\"apyl\"a, M. J., \& \Brandenburg\yan{2018}{339}{127}
{133}{Small-scale dynamos in simulations of stratified turbulent convection}
\arxivB{1802.09607}\adsT{2018AN....339..127K}\doiT{10.1002/asna.201813477}\ownE{2018/Kapy+Kapy+Bran18}
%AB: Hale ok
%SOLAR

\item[{365.}~]
\Brandenburg, \& Chatterjee, P.\yan{2018}{339}{118}
{126}{Strong nonlocality variations in a spherical mean-field dynamo}
\arxivB{1802.04231}\adsT{2018AN....339..118B}\doiT{10.1002/asna.201813472}\httpT{http://www.nordita.org/~brandenb/projects/spherical-geom/}\ownE{2018/Bran+Chat18}
%AB: Hale ok
%SOLAR

\item[{364.}~]
Schober, J., Rogachevskii, I., \Brandenburg, Boyarsky, A., Fr\"ohlich, J., Ruchayskiy, O., \& Kleeorin, N.\yapjN{2018}{858}{124}
{Laminar and turbulent dynamos in chiral magnetohydrodynamics. II. Simulations}
\arxivB{1711.09733}\adsT{2018ApJ...858..124S}\doiT{10.3847/1538-4357/aaba75}\ownE{2018/Scho_etal18}
%AB: Hale ok

\item[{363.}~]
Bushby, P. J., K\"apyl\"a, P. J., Masada, Y., \Brandenburg, Favier, B., Guervilly, C., \& K\"apyl\"a, M. J.\yanaN{2018}{612}{A97}
{Large-scale dynamos in rapidly rotating plane layer convection}
\arxivB{1710.03174}\adsT{2018A\%26A...612A..97B}\doiT{10.1051/0004-6361/201732066}\ownE{2018/Bushby_etal18}
%AB: Hale ok
%SOLAR

\item[{362.}~]
\Brandenburg, \& Giampapa, M. S.\yapjlNS{2018}{855}{L22}
{Enhanced stellar activity for slow antisolar differential rotation?}
\arxivB{1802.08689}\adsT{2018ApJ...855L..22B}\doiT{10.3847/2041-8213/aab20a}\ownE{2018/Bran+Giam18}
%http://www.nordita.org/~brandenb/Own_Papers/2018/Bran+Giam18.pdf
%solar preprints arxive submitted on Apr 15, 2018
%AB: Hale ok
%SOLAR

\item[{361.}~]
Perri, B., \& \Brandenburg\yanaN{2018}{609}{A99}
{Spontaneous flux concentrations from the negative effective magnetic pressure instability beneath a radiative stellar surface}
\arxivB{1701.03018}\adsT{2018A\%26A...609A..99P}\doiT{10.1051/0004-6361/201730421}\ownE{2018/Perr+Bran18}
%solar preprints arxive submitted on Feb 3, 2018
%SOLAR

\item[{360.}~]
Warnecke, J., Rheinhardt, M., K\"apyl\"a, P. J., K\"apyl\"a, M. J., \& \Brandenburg\yanaN{2018}{609}{A51}
{Turbulent transport coefficients in spherical wedge dynamo simulations of solar-like stars}
\arxivB{1601.03730}\adsT{2018A\%26A...609A..51W}\doiT{10.1051/0004-6361/201628136}\ownE{2018/Warn_etal18}
%solar preprints arxive submitted on Mar 12, 2018
%SOLAR

\item[{359.}~]
\Brandenburg, Kahniashvili, T., Mandal, S., Roper Pol, A., Tevzadze, A. G., \& Vachaspati, T.\yprdN{2017}{96}{123528}
{Evolution of hydromagnetic turbulence from the electroweak phase transition}
\arxivB{1711.03804}\adsT{2017PhRvD..96l3528B}\doiT{10.1103/PhysRevD.96.123528}\ownE{2017/Bran_etal17b}
%Early Universe

\item[{358.}~]
Kahniashvili, T., \Brandenburg, Durrer, R., Tevzadze, A. G., \& Yin, W.\yjourN{2017}{J. Cosmol. Astropart. Phys.}{12}{002}
{Scale-invariant helical magnetic field evolution and the duration of inflation}
\arxivB{1610.03139}\adsT{2017JCAP...12..002K}\doiT{10.1088/1475-7516/2017/12/002}\ownE{2017/Kahn_etal17}
%Early Universe

\item[{357.}~]
Singh, N. K., Rogachevskii, I., \& \Brandenburg\yapjlN{2017}{850}{L8}
{Enhancement of small-scale turbulent dynamo by large-scale shear}
\arxivB{1610.07215}\adsT{2017ApJ...850L...8S}\doiT{10.3847/2041-8213/aa96a1}\ownE{2017/Sing+Roga+Bran17}
%Dynamo Theory
%SOLAR

\item[{356.}~]
\Brandenburg, Schober, J., \& Rogachevskii, I.\yan{2017}{338}{790}
{793}{The contribution of kinetic helicity to turbulent magnetic diffusivity}
\arxivB{1706.03421}\adsT{2017AN....338..790B}\doiT{10.1002/asna.201713384}\ownE{2017/Bran+Scho+Roga17}
%Dynamo Theory

\item[{355.}~]
Rogachevskii, I., Ruchayskiy, O., Boyarsky, A., Fr\"{o}hlich, J., Kleeorin, N., \Brandenburg, \& Schober, J.\yapjN{2017}{846}{153}
{Laminar and turbulent dynamos in chiral magnetohydrodynamics. I. Theory}
\arxivB{1705.00378}\adsT{2017ApJ...846..153R}\doiT{10.3847/1538-4357/aa886b}\ownE{2017/Roga_etal17}
%Dynamo Theory

\item[{354.}~]
Cameron, R. H., Dikpati, M., \& \Brandenburg\yjour{2017}{Spa. Sci. Rev.}{210}{367}
{395}{The global solar dynamo}
\arxivB{1602.01754}\dadsT{2017SSRv..210..367C}\doiT{10.1007/s11214-015-0230-3}\ownE{2017/Came+Dikp+Bran17}
%AB: address OK, Boulder
%Solar/stellar Physics
%SOLAR

\item[{353.}~]
K\"apyl\"a, P. J., Rheinhardt, M., \Brandenburg, Arlt, R., K\"apyl\"a, M. J., Lagg, A., Olspert, N., \& Warnecke, J.\yapjlN{2017}{845}{L23}
{Extended subadiabatic layer in simulations of overshooting convection}
\arxivB{1703.06845}\adsT{2017ApJ...845L..23K}\doiT{10.3847/2041-8213/aa83ab}\ownE{2017/Kapy_etal17b}
%Solar/stellar Physics
%SOLAR

\item[{352.}~]
\Brandenburg, Schober, J., Rogachevskii, I., Kahniashvili, T., Boyarsky, A., Fr\"ohlich, J., Ruchayskiy, O., \& Kleeorin, N.\yapjlN{2017}{845}{L21}
{The turbulent chiral magnetic cascade in the early universe}
\arxivB{1707.03385}\adsT{2017ApJ...845L..21B}\doiT{10.3847/2041-8213/aa855d}\ownE{2017/Bran_etal17}
%Early Universe

\item[{351.}~]
\Brandenburg, Ashurova, M. B., \& Jabbari, S.\yapjlN{2017}{845}{L15}
{Compensating Faraday depolarization by magnetic helicity in the solar corona}
\arxivB{1706.09540}\adsT{2017ApJ...845L..15B}\doiT{10.3847/2041-8213/aa844b}\ownE{2017/Bran+Ashu+Jabb17}
%Solar/stellar Physics
%SOLAR

\item[{350.}~]
\Brandenburg, Mathur, S., \& Metcalfe, T. S.\yapjN{2017}{845}{79}
{Evolution of coexisting long and short period stellar activity cycles}
\arxivB{1704.09009}\adsT{2017ApJ...845...79B}\doiT{10.3847/1538-4357/aa7cfa}\ownE{2017/Bran+Math+Metc17}
%http://www.nordita.org/~brandenb/Own_Papers/2017/Bran+Math+Metc17.pdf
%Solar/stellar Physics
%SOLAR

\item[{349.}~]
Li, X.-Y., \Brandenburg, Haugen, N. E. L., \& Svensson, G.\yjour{2017}{J. Adv. Model. Earth Syst.}{9}{1116}
{1137}{Eulerian and Lagrangian approaches to multidimensional condensation and collection}
\arxivB{1604.08169}\adsT{2017JAMES...9.1116L}\doiT{10.1002/2017MS000930}\httpT{http://norlx51.nordita.org/~brandenb/tmp/SwarmSmolu_numerics}\ownE{2017/Li_etal17}
%Turbulence

\item[{348.}~]
Jabbari, S., \Brandenburg, Kleeorin, N., \& Rogachevskii, I.\ymn{2017}{467}{2753}
{2765}{Sharp magnetic structures from dynamos with density stratification}
\arxivB{1607.08897}\adsT{2017MNRAS.467.2753J}\doiT{10.1093/mnras/stx148}\ownE{2017/Jabb_etal17}
%AB: have added computing both PDC and CU
%Turbulence
%SOLAR

\item[{347.}~]
K\"apyl\"a, P. J., K\"apyl\"a, M. J., Olspert, N., Warnecke, J., \& \Brandenburg\yanaN{2017}{599}{A4}
{Convection-driven spherical shell dynamos at varying Prandtl numbers}
\arxivB{1605.05885}\adsT{2017A\%26A...599A...4K}\doiT{10.1051/0004-6361/201628973}\ownE{2017/Kapy_etal17}
%Solar/stellar Physics
%SOLAR

\item[{346.}~]
\Brandenburg\yanaN{2017}{598}{A117}
{Analytic solution of an oscillatory migratory alpha squared stellar dynamo}
\arxivB{1611.02671}\adsT{2017A\%26A...598A.117B}\doiT{10.1051/0004-6361/201630033}\ownE{2017/Bran17}
%Solar/stellar Physics
%SOLAR

\item[{345.}~]
\Brandenburg, Petrie, G. J. D., \& Singh, N. K.\yapjN{2017}{836}{21}
{Two-scale analysis of solar magnetic helicity}
\arxivB{1610.05410}\adsT{2017ApJ...836...21B}\doiT{10.3847/1538-4357/836/1/21}\ownE{2017/Bran+Petr+Sing17}
%solar preprints arxive submitted on Dec 12, 2017
%Solar/stellar Physics
%SOLAR

\item[{344.}~]
\Brandenburg, \& Kahniashvili, T.\yprlN{2017}{118}{055102}
{Classes of hydrodynamic and magnetohydrodynamic turbulent decay}
\arxivB{1607.01360}\adsT{2017PhRvL.118e5102B}\doiT{10.1103/PhysRevLett.118.055102}\ownT{2017/Bran+Kahn17}\supE{2017/Bran+Kahn17_supp}
%Turbulence

\item[{343.}~]
\Brandenburg, Rogachevskii, I., \& Kleeorin, N.\yjourN{2016}{New J. Phys.}{18}{125011}
{Magnetic concentrations in stratified turbulence: the negative effective magnetic pressure instability}
\arxivB{1610.03459}\adsT{2016NJPh...18l5011B}\doiT{10.1088/1367-2630/aa513e}\ownE{2016/BRK16}
%SOLAR

\item[{342.}~]
Warnecke, J., K\"apyl\"a, P. J., K\"apyl\"a, M. J., \& \Brandenburg\yanaN{2016}{596}{A115}
{Influence of a coronal envelope as a free boundary to global convective dynamo simulations}
\arxivB{1503.05251}\adsT{2016A\%26A...596A.115W}\doiT{10.1051/0004-6361/201526131}\ownE{2016/Warn_Kapy_Kapy_Bran16}
%SOLAR

\item[{341.}~]
Singh, N. K., Raichur, H., \& \Brandenburg\yapjN{2016}{832}{120}
{High-wavenumber solar {\em f}-mode strengthening prior to active region formation}
\arxivB{1601.00629}\adsT{2016ApJ...832..120S}\doiT{10.3847/0004-637X/832/2/120}\ownE{2016/Sing+Raic+Bran16}
%solar preprints arxive submitted on Oct 21, 2017
%FRPA ok
%SOLAR

\item[{340.}~]
\Brandenburg\yapjN{2016}{832}{6}
{Stellar mixing length theory with entropy rain}
\arxivB{1504.03189}\adsT{2016ApJ...832....6B}\doiT{10.3847/0004-637X/832/1/6}\ownE{2016/B16}
%solar preprints arxive of Dec 24, 2016
%SOLAR

\item[{339.}~]
Cole, E., \Brandenburg, K\"apyl\"a, P. J., \& K\"apyl\"a, M. J.\yanaN{2016}{593}{A134}
{Robustness of oscillatory alpha squared dynamos in spherical wedges}
\arxivB{1601.05246}\adsT{2016A\%26A...593A.134C}\doiT{10.1051/0004-6361/201628165}\ownE{2016/Cole_etal16}
%FRPA ok
%SOLAR

\item[{338.}~]
Kahniashvili, T., \Brandenburg, \& Tevzadze, A. G.\ypsN{2016}{91}{104008}
{The evolution of primordial magnetic fields since their generation}
\arxivB{1507.00510}\adsT{2016PhyS...91j4008K}\doiT{10.1088/0031-8949/91/10/104008}\ownE{2016/Kahn+Bran+Tevz16}
%FRPA ok

\item[{337.}~]
Bhat, P., Subramanian, K., \& \Brandenburg\ymn{2016}{461}{240}
{247}{A unified large/small-scale dynamo in helical turbulence}
\arxivB{1508.02706}\adsT{2016MNRAS.461..240B}\doiT{10.1093/mnras/stw1257}\ownE{2016/Bhat_etal16}
%FRPA ok
%SOLAR

\item[{336.}~]
Jabbari, S., \Brandenburg, Mitra, D., Kleeorin, N., \& Rogachevskii, I.\ymn{2016}{459}{4046}
{4056}{Turbulent reconnection of magnetic bipoles in stratified turbulence}
\arxivB{1601.08167}\adsT{2016MNRAS.459.4046J}\doiT{10.1093/mnras/stw888}\ownE{2016/Jabb_etal16}
%FRPA ok
%SOLAR

\item[{335.}~]
Warnecke, J., Losada, I. R., \Brandenburg, Kleeorin, N., \& Rogachevskii, I.\yanaN{2016}{589}{A125}
{Bipolar region formation in stratified two-layer turbulence}
\arxivB{1502.03799}\adsT{2016A\%26A...589A.125W}\doiT{10.1051/0004-6361/201525880}\ownE{2016/Warn_etal16a}
%solar preprints arxive of Oct 31, 2016
%FRPA ok
%SOLAR

\item[{334.}~]
K\"apyl\"a, M. J., K\"apyl\"a, P. J., Olspert, N., \Brandenburg, Warnecke, J., Karak, B. B., \& Pelt, J.\yanaN{2016}{589}{A56}
{Multiple dynamo modes as a mechanism for long-term solar activity variations}
\arxivB{1507.05417}\adsT{2016A\%26A...589A..56K}\doiT{10.1051/0004-6361/201527002}\httpT{http://www.mps.mpg.de/the-sun-s-hidden-magnetic-field-during-grand-minimum}\ownE{2016/MKapy_etal16}
%FRINA/added
%FRPA ok
%SOLAR

\item[{333.}~]
K\"apyl\"a, P. J., \Brandenburg, Kleeorin, N., K\"apyl\"a, M. J., \& Rogachevskii, I.\yanaN{2016}{588}{A150}
{Magnetic flux concentrations from turbulent stratified convection}
\arxivB{1511.03718}\adsT{2016A\%26A...588A.150K}\doiT{10.1051/0004-6361/201527731}\ownE{2016/Kapy_etal16}
%solar preprints arxive of June 30, 2016
%FRPA ok
%SOLAR

\item[{332.}~]
Yokoi, N., \& \Brandenburg\ypreN{2016}{93}{033125}
{Large-scale flow generation by inhomogeneous helicity}
\arxivB{1511.08983}\adsT{2016PhRvE..93c3125Y}\doiT{10.1103/PhysRevE.93.033125}\ownE{2016/Yoko+Bran16}
%noFRINA
%FRPA ok

\item[{331.}~]
Zhang, H., \Brandenburg, \& Sokoloff, D. D.\yapjN{2016}{819}{146}
{Evolution of magnetic helicity and energy spectra of solar active regions}
\arxivB{1503.00846}\adsT{2016ApJ...819..146Z}\doiT{10.3847/0004-637X/819/2/146}\ownE{2016/Zhang_etal16}
%solar preprints arxive of March 13, 2016
%AB: address OK
%FRPA ok
%SOLAR

\item[{330.}~]
\Brandenburg\yjourN{2016}{Physics}{9}{26}
{A new twist in simulating solar flares}
\arxivB{1603.01917}\adsT{2016PhyOJ...9...26B}\doiT{10.1103/Physics.9.26}\ownE{2016/Bran16}
%solar preprints arxive of March 13, 2016
%AB: address OK, Boulder
%FRINA 15/ added
%SOLAR

\item[{329.}~]
Bhat, P., \& \Brandenburg\yanaN{2016}{587}{A90}
{Hydraulic effects in a radiative atmosphere with ionization}
\arxivB{1411.6610}\adsT{2016A\%26A...587A..90B}\doiT{10.1051/0004-6361/201425396}\ownE{2016/Bhat+Bran16}
%solar preprints arxive of May 20, 2016
%AB: address OK, Boulder
%noFRINA
%SOLAR

\item[{328.}~]
Karak, B. B., \& \Brandenburg\yapjSN{2016}{816}{28}
{Is the small-scale magnetic field correlated with the dynamo cycle?}
\arxivB{1505.06632}\adsT{2016ApJ...816...28K}\doiT{10.3847/0004-637X/816/1/28}\ownE{2016/Kara+Bran16}
%AB: address OK, Boulder
%FRINA/added
%FRPA ok
%SOLAR

\item[{327.}~]
Miesch, M., Matthaeus, W., \Brandenburg, Petrosyan, A., Pouquet, A., Cambon, C., Jenko, F., Uzdensky, D., Stone, J., Tobias, S., Toomre, J., \& Velli, M.\yjour{2015}{Spa. Sci. Rev.}{194}{97}
{137}{Large-eddy simulations of magnetohydrodynamic turbulence in space and astrophysics}
\arxivB{1505.01808}\adsT{2015SSRv..194...97M}\doiT{10.1007/s11214-015-0190-7}\ownE{2015/Miesch_etal15}
%FRINA 15

\item[{326.}~]
Andrievsky, A., \Brandenburg, Noullez, A., \& Zheligovsky, V.\yapjN{2015}{811}{135}
{Negative magnetic eddy diffusivities from the test-field method and multiscale stability theory}
\arxivB{1501.04465}\adsT{2015ApJ...811..135A}\doiT{10.1088/0004-637X/811/2/135}\ownE{2015/Andr+Bran+Noul+Zhel15}

\item[{325.}~]
Jabbari, S., \Brandenburg, Kleeorin, N., Mitra, D., \& Rogachevskii, I.\yapjN{2015}{805}{166}
{Bipolar magnetic spots from dynamos in stratified spherical shell turbulence}
\arxivB{1411.4912}\adsT{2015ApJ...805..166J}\doiT{10.1088/0004-637X/805/2/166}\ownE{2015/Jabb_etal15}
%FRINA 15

\item[{324.}~]
Karak, B. B., Kitchatinov, L. L., \& \Brandenburg\yapjN{2015}{803}{95}
{Hysteresis between distinct modes of turbulent dynamos}
\arxivB{1411.0485}\adsT{2015ApJ...803...95K}\doiT{10.1051/0004-6361/201424521}\ownE{2015/Kara+Kit+Bran15}
%FRINA 14

\item[{323.}~]
Karak, B. B., K\"apyl\"a, M. J., K\"apyl\"a, P. J., \Brandenburg, Olspert, N., \& Pelt, J.\yanaN{2015}{576}{A26}
{Magnetically controlled stellar differential rotation near the transition from solar to anti-solar profiles}
\arxivB{1407.0984}\adsT{2015A\%26A...576A..26K}\doiT{10.1051/0004-6361/201424521}\ownE{2015/Kara_etal15}
%FRINA 13

\item[{322.}~]
\Brandenburg, Kahniashvili, T., \& Tevzadze, A. G.\yprlN{2015}{114}{075001}
{Nonhelical inverse transfer of a decaying turbulent magnetic field}
\arxivB{1404.2238}\adsT{2015PhRvL.114g5001B}\doiT{10.1103/PhysRevLett.114.075001}\ownT{2015/Bran+Kahn+Tevz15}\supE{2015/Bran+Kahn+Tevz15_supp}
%FRINA 12 (entered)

\item[{321.}~]
Singh, N. K., \Brandenburg, Chitre, S. M., \& Rheinhardt, M.\ymn{2015}{447}{3708}
{3722}{Properties of $p$- and $f$-modes in hydromagnetic turbulence}
\arxivB{1404.3246}\adsT{2015MNRAS.447.3708S}\doiT{10.1093/mnras/stu2540}\ownE{2015/Sing+Bran+Chit+Rhei15}
%FRINA 11

\item[{320.}~]
\Brandenburg, Hubbard, A., \& K\"apyl\"a, P. J.\yan{2015}{336}{91}
{96}{Dynamical quenching with non-local alpha and downward pumping}
\arxivB{1412.0997}\adsT{2015AN....336...91B}\doiT{10.1002/asna.201412141}\ownE{2015/Bran+Hubb+Kapy15}
%noFRINA

\item[{319.}~]
%listed in ERC report
Barekat, A., \& \Brandenburg\yanaN{2014}{571}{A68}
{Near-polytropic stellar simulations with a radiative surface}
\arxivB{1308.1660}\adsT{2014A\%26A...571A..68B}\doiT{10.1051/0004-6361/201322461}\ownE{2014/Bare+Bran14}
%solar preprints arxive of 2014-12-13
%noFRINA

\item[{318.}~]
Warnecke, J., K\"apyl\"a, P. J., K\"apyl\"a, M. J., \& \Brandenburg\yapjlN{2014}{796}{L12}
{On the cause of solar-like equatorward migration in global convective dynamo simulations}
\arxivB{1409.3213}\adsT{2014ApJ...796L..12W}\doiT{10.1088/2041-8205/796/1/L12}\ownE{2014/Warn_etal14}
%noFRINA

\item[{317.}~]
Subramanian, K., \& \Brandenburg\ymn{2014}{445}{2930}
{2940}{Traces of large-scale dynamo action in the kinematic stage}
\arxivB{1408.4416}\adsT{2014MNRAS.445.2930S}\doiT{10.1093/mnras/stu1954}\ownE{2014/Subm+Bran14}
%FRINA 10 (in 2014)

\item[{316.}~]
Singh, N. K., \Brandenburg, \& Rheinhardt, M.\yapjlNS{2014}{795}{L8}
{Fanning out of the solar $f$-mode in presence of nonuniform magnetic fields?}
\arxivB{1407.0356}\adsT{2014ApJ...795L...8S}\doiT{10.1088/2041-8205/795/1/L8}\ownE{2014/Sing+Bran+Rhei14}
%FRINA 9

\item[{315.}~]
K\"apyl\"a, P. J., K\"apyl\"a, M. J., \& \Brandenburg\yanaN{2014}{570}{A43}
{Confirmation of bistable stellar differential rotation profiles}
\arxivB{1401.2981}\adsT{2014A\%26A...570A..43K}\doiT{10.1051/0004-6361/201423412}\ownE{2014/Kapy+Kapy+Bran14}
%noFRINA

\item[{314.}~]
Karak, B. B., Rheinhardt, M., \Brandenburg, K\"apyl\"a, P. J., \& K\"apyl\"a, M. J.\yapjN{2014}{795}{16}
{Quenching and anisotropy of hydromagnetic turbulent transport}
\arxivB{1406.4521}\adsT{2014ApJ...795...16K}\doiT{10.1088/0004-637X/795/1/16}\ownE{2014/Kara_etal14}
%FRINA 8

\item[{313.}~]
Mitra, D., \Brandenburg, Kleeorin, N., Rogachevskii, I.\ymn{2014}{445}{761}
{769}{Intense bipolar structures from stratified helical dynamos}
\arxivB{1404.3194}\adsT{2014MNRAS.445..761M}\doiT{10.1093/mnras/stu1755}\ownE{2014/Mitr+Bran+Klee+Roga14}
%solar preprints arxive of January 16, 2016
%FRINA 7

\item[{312.}~]
Jabbari, S., \Brandenburg, Losada, I. R., Kleeorin, N., \& Rogachevskii, I.\yanaN{2014}{568}{A112}
{Magnetic flux concentrations from dynamo-generated fields}
\arxivB{1401.6107}\adsT{2014A\%26A...568A.112J}\doiT{10.1051/0004-6361/201423499}\ownE{2014/Jabb_etal14}
%solar preprints arxive of July 27, 2015
%FRINA 6

\item[{311.}~]
Candelaresi, S., Hillier, A., Maehara, H., \Brandenburg, \& Shibata, K.\yapjN{2014}{792}{67}
{Superflare occurrence and energies on G-, K- and M-type dwarfs}
\arxivB{1405.1453}\adsT{2014ApJ...792...67C}\doiT{10.1088/0004-637X/792/1/67}\ownE{2014/Cand_etal14}
%http://www.nordita.org/~brandenb/Own_Papers/2014/Cand_etal14.pdf
%solar preprints arxive of Sept 25, 2014
%FRINA 5

\item[{310.}~]
Modestov, M., Bychkov, V., Brodin, G., Marklund, M., \& \Brandenburg\yppN{2014}{21}{072126}
{Evolution of magnetic field generated by the Kelvin-Helmholtz instability}
\arxivB{1402.2761}\adsT{2014PhPl...21g2126M}\doiT{10.1063/1.4891340}\ownE{2014/Mode_etal14}
%FRINA 4

\item[{309.}~]
V\"ais\"al\"a, M. S., \Brandenburg, Mitra, D., K\"apyl\"a, P. J., \& Mantere, M. J.\yanaN{2014}{567}{A139}
{Quantifying the effect of turbulent magnetic diffusion on the growth rate of the magneto-rotational instability}
\arxivB{1310.3157}\adsT{2014A\%26A...567A.139V}\doiT{10.1051/0004-6361/201322837}\ownE{2014/Vaisala_etal14}
%noFRINA

\item[{308.}~]
\Brandenburg\yapjN{2014}{791}{12}
{Magnetic Prandtl number dependence of the kinetic-to-magnetic dissipation ratio}
\arxivB{1404.6964}\adsT{2014ApJ...791...12B}\doiT{10.1088/0004-637X/791/1/12}\ownE{2014/B14}
%http://www.nordita.org/~brandenb/Own_Papers/2014/B14.pdf
%solar preprints arxive of July 30, 2014
%FRINA 3

%listed in ERC report III
\item[{307.}~]
Mitra, D., \Brandenburg, Dasgupta, D., Niklasson, E., \& Ram, A.\ypreN{2014}{89}{042919}
{Particle energization through time-periodic helical magnetic fields}
\arxivB{1306.0151}\adsT{2014PhRvE..89d2919M}\doiT{10.1103/PhysRevE.89.042919}\ownE{2014/Mitr+Bran+Dasg+Nilk+Ram14}
%noFRINA

\item[{306.}~]
Rheinhardt, M., Devlen, E., R\"adler, K.-H., \& \Brandenburg\ymn{2014}{441}{116}
{126}{Mean-field dynamo action from delayed transport}
\arxivB{1401.5026}\adsT{2014MNRAS.441..116R}\doiT{10.1093/mnras/stu438}\ownE{2014/Rhei+Devl+Radl+Bran14}
%Submitted to solar preprints arxive on May 29, 2014
%FRINA 2

\item[{305.}~]
\Brandenburg, \& Stepanov, R.\yapjN{2014}{786}{91}
{Faraday signature of magnetic helicity from reduced depolarization}
\arxivB{1401.4102}\adsT{2014ApJ...786...91B}\doiT{10.1088/0004-637X/786/2/91}\ownE{2014/Bran+Step14}
%FRINA 1

%listed in ERC report
\item[{304.}~]
Losada, I. R., \Brandenburg, Kleeorin, N., \& Rogachevskii, I.\yanaN{2014}{564}{A2}
{Magnetic flux concentrations in a polytropic atmosphere}
\arxivB{1307.4945}\adsT{2014A\%26A...564A...2L}\doiT{10.1051/0004-6361/201322315}\ownE{2014/Losa_etal14}
%Submitted to solar preprints arxive on May 3, 2014

\item[{303.}~]
Zhang, H., \Brandenburg, \& Sokoloff, D. D.\yapjlN{2014}{784}{L45}
{Magnetic helicity and energy spectra of a solar active region}
\arxivB{1311.2432}\adsT{2014ApJ...784L..45Z}\doiT{10.1088/2041-8205/784/2/L45}\ownE{2014/Zhang_etal14}
%Submitted to solar preprints arxive on March 23, 2014

%listed in ERC report II
\item[{302.}~]
R\"udiger, G., \& \Brandenburg\ypreN{2014}{89}{033009}
{The alpha effect in a turbulent liquid-metal plane Couette flow}
\arxivB{1201.0652}\adsT{2014PhRvE..89c3009R}\doiT{10.1103/PhysRevE.89.033009}\ownE{2014/Rue+Bran14}

%listed in ERC report
\item[{301.}~]
\Brandenburg, Gressel, O., Jabbari, S., Kleeorin, N., \& Rogachevskii, I.\yanaN{2014}{562}{A53}
{Mean-field and direct numerical simulations of magnetic flux concentrations from vertical field}
\arxivB{1309.3547}\adsT{2014A\%26A...562A..53B}\doiT{10.1051/0004-6361/201322681}\ownE{2014/BGJKR14}
%Submitted to solar preprints arxive on July 5, 2014

\item[{300.}~]
Cole, E., K\"apyl\"a, P. J., Mantere, M. J., \& \Brandenburg\yapjlN{2014}{780}{L22}
{An azimuthal dynamo wave in spherical shell convection}
\arxivB{1309.6802}\adsT{2014ApJ...780L..22C}\doiT{10.1088/2041-8205/780/2/L22}\ownE{2014/Cole_etal14}

\item[{299.}~]
Warnecke, J., K\"apyl\"a, P. J., Mantere, M. J., \& \Brandenburg\yapjN{2013}{778}{141}
{Spoke-like differential rotation in a convective dynamo with a coronal envelope}
\arxivB{1301.2248}\adsT{2013ApJ...778..141W}\doiT{10.1088/0004-637X/778/2/141}\ownE{2013/Warn_etal13b}
%submitted to solar physics archive (7 Dec 2013)

\item[{298.}~]
K\"apyl\"a, P. J., Mantere, M. J., Cole, E., Warnecke, J., \& \Brandenburg\yapjN{2013}{778}{41}
{Effects of enhanced stratification on equatorward dynamo wave propagation}
\arxivB{1301.2595}\adsT{2013ApJ...778...41K}\doiT{10.1088/0004-637X/778/1/41}\ownE{2013/Kapy_etal13}
%submitted to solar physics archive (2 Feb 2014)

\item[{297.}~]
%listed in ERC report under Period III
\Brandenburg, \& Lazarian, A.\yjour{2013}{Spa.\ Sci.\ Rev.}{178}{163}
{200}{Astrophysical hydromagnetic turbulence}
\arxivB{1307.5496}\adsT{2013SSRv..178..163B}\doiT{10.1007/s11214-013-0009-3}\ownE{2013/Bran+Laza13}

%listed in ERC report under Period III
\item[{296.}~]
Bykov, A. M., \Brandenburg, Malkov, M. A., \& Osipov, S. M.\yjour{2013}{Spa.\ Sci.\ Rev.}{178}{201}
{232}{Microphysics of cosmic ray driven plasma instabilities}
\arxivB{1304.7081}\adsT{2013SSRv..178..201B}\doiT{10.1007/s11214-013-9988-3}\ownE{2013/Bykov_etal13}

%listed in ERC report
\item[{295.}~]
Warnecke, J., Losada, I. R., \Brandenburg, Kleeorin, N., \& Rogachevskii, I.\yapjlN{2013}{777}{L37}
{Bipolar magnetic structures driven by stratified turbulence with a coronal envelope}
\arxivB{1308.1080}\adsT{2013ApJ...777L..37W}\doiT{10.1088/2041-8205/777/2/L37}\ownE{2013/Warn_etal13a}
%submitted to solar physics archive (15 Nov 2013)

%The following papers have been mentioned in the 2013 computing report
%listed in ERC report under Period III
\item[{294.}~]
\Brandenburg, Kleeorin, N., \& Rogachevskii, I.\yapjlN{2013}{776}{L23}
{Self-assembly of shallow magnetic spots through strongly stratified turbulence}
\arxivB{1306.4915}\adsT{2013ApJ...776L..23B}\doiT{10.1088/2041-8205/776/2/L23}\ownE{2013/BKR13}
%is on solar physics archive (3 Oct 2013)

%listed in ERC report under Period III
\item[{293.}~]
Rempel, E. L., Chian, A. C.-L., \Brandenburg, Mu\~noz, P. R., \& Shadden, S. C.\yjfm{2013}{729}{309}
{329}{Coherent structures and the saturation of a nonlinear dynamo}
\arxivB{1210.6637}\adsT{2013JFM...729..309R}\doiT{10.1017/jfm.2013.290}\ownE{2013/Remp_etal13}

%listed in ERC report
\item[{292.}~]
Kemel, K., \Brandenburg, Kleeorin, N., Mitra, D., \& Rogachevskii, I.\ysph{2013}{287}{293}
{313}{Active region formation through the negative effective magnetic pressure instability}
\arxivB{1203.1232}\adsT{2013SoPh..287..293K}\doiT{10.1007/s11207-012-0031-8}\ownE{2013/Kem+Bran+Klee+Mitr+Roga13}
%submitted to solar physics archive (15 Sept 2013)

%listed in ERC report
\item[{291.}~]
Jabbari, S., \Brandenburg, Kleeorin, N., Mitra, D., \& Rogachevskii, I.\yanaN{2013}{556}{A106}
{Surface flux concentrations in a spherical alpha squared dynamo}
\arxivB{1302.5841}\adsT{2013A\%26A...556A.106J}\doiT{10.1051/0004-6361/201321353}\ownE{2013/Jabb_etal13}

%listed in ERC report
\item[{290.}~]
Losada, I. R., \Brandenburg, Kleeorin, N., \& Rogachevskii, I.\yanaN{2013}{556}{A83}
{Competition of rotation and stratification in flux concentrations}
\arxivB{1212.4077}\adsT{2013A\%26A...556A..83L}\doiT{10.1051/0004-6361/201220939}\ownE{2013/Losa_etal13}

%listed in ERC report III
\item[{289.}~]
Mitra, D., Wettlaufer, J. S., \& \Brandenburg\yapjSN{2013}{773}{120}
{Can planetesimals form by collisional fusion?}
\arxivB{1306.3672}\adsT{2013ApJ...773..120M}\doiT{10.1088/0004-637X/773/2/120}\ownE{2013/Mitr+Wett+Bran13}

%listed in ERC report III
\item[{288.}~]
Kemel, K., \Brandenburg, Kleeorin, N., \& Rogachevskii, I.\ypsN{2013}{T155}{014027}
{Non-uniformity effects in the negative effective magnetic pressure instability}
\arxivB{1208.0517}\adsT{2013PhST..155a4027K}\doiT{10.1088/0031-8949/2013/T155/014027}\ownE{2013/Kem+Bran+Klee+Roga13}

%listed in ERC report III
\item[{287.}~]
Svedin, A., Cu\'ellar, M. C., \& \Brandenburg\ymn{2013}{433}{2278}
{2285}{Data assimilation for stratified convection}
\arxivB{1207.7314}\adsT{2013MNRAS.433.2278S}\doiT{10.1093/mnras/stt891}\ownE{2013/Sved_etal13}

%listed in ERC report III
\item[{286.}~]
Devlen, E., \Brandenburg, \& Mitra, D.\ymn{2013}{432}{1651}
{1657}{A mean field dynamo from negative eddy diffusivity}
\arxivB{1212.2626}\adsT{2013MNRAS.432.1651D}\doiT{10.1093/mnras/stt590}\ownE{2013/Devl+Bran+Mitr13}

%listed in ERC report under Period III
\item[{285.}~]
Kahniashvili, T., Tevzadze, A. G., \Brandenburg, \& Neronov, A.\yprdN{2013}{87}{083007}
{Evolution of primordial magnetic fields from phase transitions}
\arxivB{1212.0596}\adsT{2013PhRvD..87h3007K}\doiT{10.1103/PhysRevD.87.083007}\ownE{2013/Kahn_etal13}

%listed in ERC report under Period III
\item[{284.}~]
Candelaresi, S., \& \Brandenburg\ypreNS{2013}{87}{043104}
{Kinetic helicity needed to drive large-scale dynamos}
\arxivB{1208.4529}\adsT{2013PhRvE..87d3104C}\doiT{10.1103/PhysRevE.87.043104}\ownE{2013/Cand+Bran13}

%listed in ERC report under Period II
\item[{283.}~]
K\"apyl\"a, P. J., Mantere, M. J., \& \Brandenburg\ygafd{2013}{107}{244}
{257}{Oscillatory large-scale dynamos from Cartesian convection simulations}
\arxivB{1111.6894}\adsT{2013GApFD.107..244K}\doiT{10.1080/03091929.2012.715158}\ownE{2013/Kapy+Mant+Bran13}

%listed in ERC report under Period II
\item[{282.}~]
\Brandenburg, \& R\"adler, K.-H.\ygafd{2013}{107}{207}
{217}{Yoshizawa's cross-helicity effect and its quenching}
\arxivB{1112.1237}\adsT{2013GApFD.107..207B}\doiT{10.1080/03091929.2012.681307}\ownE{2013/Bran+Radl13}

%listed in ERC report under Period III
\item[{281.}~]
Del Sordo, F., Guerrero, G., \& \Brandenburg\ymn{2013}{429}{1686}
{1694}{Turbulent dynamo with advective magnetic helicity flux}
\arxivB{1205.3502}\adsT{2013MNRAS.429.1686D}\doiT{10.1093/mnras/sts398}\ownE{2013/DSor+Guer+Bran13}

%listed in ERC report under Period III
\item[{280.}~]
\Brandenburg, Gressel, O., K\"apyl\"a, P. J., Kleeorin, N., Mantere, M. J., Rogachevskii, I.\yapjN{2013}{762}{127}
{New scaling for the alpha effect in slowly rotating turbulence}
\arxivB{1208.5004}\adsT{2013ApJ...762..127B}\doiT{10.1088/0004-637X/762/2/127}\ownE{2013/BGKKMR13}

%listed in ERC report
\item[{279.}~]
Losada, I. R., \Brandenburg, Kleeorin, N., Mitra, D., \& Rogachevskii, I.\yanaN{2012}{548}{A49}
{Rotational effects on the negative magnetic pressure instability}
\arxivB{1207.5392}\adsT{2012A\%26A...548A..49L}\doiT{10.1051/0004-6361/201220078}\ownE{2012/Losa_etal12}
%is on solar physics archive

%listed in ERC report under Period III
\item[{278.}~]
Kahniashvili, T., \Brandenburg, Campanelli, L., Ratra, B., \& Tevzadze, A. G.\yprdN{2012}{86}{103005}
{Evolution of inflation-generated magnetic field through phase transitions}
\arxivB{1206.2428}\adsT{2012PhRvD..86j3005K}\doiT{10.1103/PhysRevD.86.103005}\ownE{2012/Kahn_etal12}

%listed in ERC report under Period III
\item[{277.}~]
Tevzadze, A. G., Kisslinger, L., \Brandenburg, \& Kahniashvili, T.\yapjN{2012}{759}{54}
{Magnetic fields from QCD phase transitions}
\arxivB{1207.0751}\adsT{2012ApJ...759...54T}\doiT{10.1088/0004-637X/759/1/54}\ownE{2012/Tevz_etal12}

%listed in ERC report under Period II
\item[{276.}~]
Warnecke, J., K\"apyl\"a, P. J., Mantere, M. J., \& \Brandenburg\ysph{2012}{280}{299}
{319}{Ejections of magnetic structures above a spherical wedge driven by a convective dynamo with differential rotation}
\arxivB{1112.0505}\adsT{2012SoPh..280..299W}\doiT{10.1007/s11207-012-0108-4}\ownE{2012/Warn+Kapy+Mant+Bran12}
%is on solar physics archive

%listed in ERC report under Period II
\item[{275.}~]
Kemel, K., \Brandenburg, Kleeorin, N., Mitra, D., \& Rogachevskii, I.\ysph{2012}{280}{321}
{333}{Spontaneous formation of magnetic flux concentrations in stratified turbulence}
\arxivB{1112.0279}\adsT{2012SoPh..280..321K}\doiT{10.1007/s11207-012-9949-0}\ownE{2012/Kem+Bran+Klee+Mitr+Roga12}

%listed in ERC report under Period II
\item[{274.}~]
Haugen, N. E. L., Kleeorin, N., Rogachevskii, I., \& \Brandenburg\ypfN{2012}{24}{075106}
{Detection of the phenomenon of turbulent thermal diffusion in numerical simulations}
\arxivB{1101.4188}\adsT{2012PhFl...24g5106H}\doiT{10.1063/1.4733450}\ownE{2012/HKRB12}

%listed in ERC report under Period III
\item[{273.}~]
\Brandenburg, Sokoloff, D., \& Subramanian, K.\yjour{2012}{Spa.\ Sci.\ Rev.}{169}{123}
{157}{Current status of turbulent dynamo theory: From large-scale to small-scale dynamos}
\arxivB{1203.6195}\adsT{2012SSRv..169..123B}\doiT{10.1007/s11214-012-9909-x}\ownE{2012/Bran+Sok+Sub12}

%listed in ERC report under Period III
\item[{272.}~]
Warnecke, J., \Brandenburg, \& Mitra, D.\yjourN{2012}{J. Spa.\ Weather Spa.\ Clim.}{2}{A11}
{Magnetic twist: a source and property of space weather}
\arxivB{1203.0959}\adsT{2012JSWSC...2A..11W}\doiT{10.1051/swsc/2012011}\ownE{2012/Warn+Bran+Mitra12}
%is on solar physics archive

%listed in ERC report under Period III
\item[{271.}~]
K\"apyl\"a, P. J., Mantere, M. J., \& \Brandenburg\yapjlN{2012}{755}{L22}
{Cyclic magnetic activity due to turbulent convection in spherical wedge geometry}
\arxivB{1205.4719}\adsT{2012ApJ...755L..22K}\doiT{10.1088/2041-8205/755/1/L22}\ownE{2012/Kapy+Mant+Bran12}
%is on solar physics archive

%listed in ERC report under Period III
\item[{270.}~]
Bonanno, A., \Brandenburg, Del Sordo, F., \& Mitra, D.\ypreN{2012}{86}{016313}
{Breakdown of chiral symmetry during saturation of the Tayler instability}
\arxivB{1204.0081}\adsT{2012PhRvE..86a6313B}\doiT{10.1103/PhysRevE.86.016313}\ownE{2012/Bona+Bran+DSor+Mit12}

%listed in ERC report under Period II
\item[{269.}~]
Snellman, J. E., Rheinhardt, M., K\"apyl\"a, P. J., Mantere, M. J., \& \Brandenburg\ypsN{2012}{86}{018406}
{Mean-field closure parameters for passive scalar turbulence}
\arxivB{1112.4777}\adsT{2012PhyS...86a8406S}\doiT{10.1088/0031-8949/86/01/018406}\ownE{2012/Snell_etal12b}

%listed in ERC report under Period II
\item[{268.}~]
Rempel, E. L., Chian, A. C.-L., \& \Brandenburg\ypsN{2012}{86}{018405}
{Lagrangian chaos in an ABC-forced nonlinear dynamo}
\arxivB{1201.4324}\adsT{2012PhyS...86a8405R}\doiT{10.1088/0031-8949/86/01/018405}\ownE{2012/Remp+Chian+Bran12}

%listed in ERC report under Period III
\item[{267.}~]
Rogachevskii, I., Kleeorin, N., \Brandenburg, \& Eichler, D.\yapjN{2012}{753}{6}
{Cosmic-ray current-driven turbulence and mean-field dynamo effect}
\arxivB{1204.4246}\adsT{2012ApJ...753....6R}\doiT{10.1088/0004-637X/753/1/6}\ownE{2012/Roga+Klee+Bran+Eich12}

% ---------------- entered as unpublished -------------------------
\item[{266.}~]
K\"apyl\"a, P. J., \Brandenburg, Kleeorin, N., Mantere, M. J., \& Rogachevskii, I.\ymn{2012}{422}{2465}
{2473}{Negative effective magnetic pressure in turbulent convection}
\arxivB{1104.4541}\adsT{2012MNRAS.422.2465K}\doiT{10.1111/j.1365-2966.2012.20801.x}\ownE{2012/Kapy_etal12}

\item[{265.}~]
\Brandenburg, Kemel, K., Kleeorin, N., \& Rogachevskii, I.\yapjN{2012}{749}{179}
{The negative effective magnetic pressure in stratified forced turbulence}
\arxivB{1005.5700}\adsT{2012ApJ...749..179B}\doiT{10.1088/0004-637X/749/2/179}\ownE{2012/BKKR12}

\item[{264.}~]
Kitchatinov, L. L., \& \Brandenburg\yan{2012}{333}{230}
{236}{Transport of angular momentum and chemical species by anisotropic mixing in stellar radiative interiors}
\arxivB{1201.2484}\adsT{2012AN....333..230K}\doiT{10.1002/asna.201211648}\ownE{2012/Kit+Bran12}

\item[{263.}~]
Chan, C. K., Mitra, D., \& \Brandenburg\ypreN{2012}{85}{036315}
{Dynamics of saturated energy condensation in two-dimensional turbulence}
\arxivB{1109.6937}\adsT{2012PhRvE..85c6315C}\doiT{10.1103/PhysRevE.85.036315}\ownE{2012/Chan+Mitr+Bran12}

\item[{262.}~]
Dosopoulou, F., Del Sordo, F., Tsagas, C.\ G., \& \Brandenburg\yprdN{2012}{85}{063514}
{Vorticity production and survival in viscous and magnetized cosmologies}
\arxivB{1112.6164}\adsT{2012PhRvD..85f3514D}\doiT{10.1103/PhysRevD.85.063514}\ownE{2012/Doso_etal12}

\item[{261.}~]
\Brandenburg, \& Petrosyan, A.\yan{2012}{333}{195}
{201}{Reynolds number dependence of kinetic helicity decay in linearly forced turbulence}
\arxivB{1012.1464}\adsT{2012AN....333..195B}\doiT{10.1002/asna.201211654}\ownE{2012/Bran+Petr12}

\item[{260.}~]
Hubbard, A., \& \Brandenburg\yapjN{2012}{748}{51}
{Catastrophic quenching in alpha--Omega dynamos revisited}
\arxivB{1107.0238}\adsT{2012ApJ...748...51H}\doiT{10.1088/0004-637X/748/1/51}\ownE{2012/Hubb+Bran12}

% ---------------- ERC reporting cut -------------------------
\item[{259.}~]
\Brandenburg, R\"adler, K.-H., \& Kemel, K.\yanaN{2012}{539}{A35}
{Mean-field transport in stratified and/or rotating turbulence}
\arxivB{1108.2264}\adsT{2012A\%26A...539A..35B}\adsT{2012A\%26A...545C...1B}\doiT{10.1051/0004-6361/201117871}\ownT{2012/Bran+Radl+Kem12}\ownE{2012/Bran+Radl+Kem12_corr}

\item[{258.}~]
Mitra, D., \& \Brandenburg\ymn{2012}{420}{2170}
{2177}{Scaling and intermittency in incoherent alpha--shear dynamo}
\arxivB{1107.2419}\adsT{2012MNRAS.420.2170M}\doiT{10.1111/j.1365-2966.2011.20190.x}\ownE{2012/Mit+Bran12}

\item[{257.}~]
Kemel, K., \Brandenburg, Kleeorin, N., \& Rogachevskii, I.\yan{2012}{333}{95}
{100}{Properties of the negative effective magnetic pressure instability}
\arxivB{1107.2752}\adsT{2012AN....333...95K}\doiT{10.1002/asna.201111638}\ownE{2012/KBKR12}

\item[{256.}~]
Guerrero, G., Rheinhardt, M., \Brandenburg, \& Dikpati, M.\ymn{2012}{420}{L1}
{L5}{Plasma flow versus magnetic feature-tracking speeds in the Sun}
\arxivB{1107.4801}\adsT{2012MNRAS.420L...1G}\doiT{10.1111/j.1745-3933.2011.01167.x}\ownE{2012/Guer_etal12}

\item[{255.}~]
Rheinhardt, M., \& \Brandenburg\yan{2012}{333}{71}
{77}{Modeling spatio-temporal nonlocality in mean-field dynamos}
\arxivB{1110.2891}\adsT{2012AN....333...71R}\doiT{10.1002/asna.201111625}\httpT{http://norlx51.nordita.org/~brandenb/tmp/nonlocal_pde}\ownE{2012/Rhei+Bran12}

\item[{254.}~]
Snellman, J. E., \Brandenburg, K\"apyl\"a, P. J., \& Mantere, M. J.\yan{2012}{333}{78}
{83}{Verification of Reynolds stress parameterizations from simulations}
\arxivB{1109.4857}\adsT{2012AN....333...78S}\doiT{10.1002/asna.201111617}\ownE{2012/Snell_etal12}

\item[{253.}~]
K\"apyl\"a, P. J., Mantere, M. J., \& \Brandenburg\yan{2011}{332}{883}
{890}{Effects of stratification in spherical shell convection}
\arxivB{1109.4625}\adsT{2011AN....332..883K}\doiT{10.1002/asna.201111619}\ownE{2011/Kapy+Mant+Bran11}
%X

\item[{252.}~]
Kemel, K., \Brandenburg, \& Ji, H.\ypreN{2011}{84}{056407}
{A model of driven and decaying magnetic turbulence in a cylinder}
\arxivB{1106.1129}\adsT{2011PhRvE..84e6407K}\doiT{10.1103/PhysRevE.84.056407}\ownE{2011/Kem+Bran+Ji11}
%X

\item[{251.}~]
Rogachevskii, I., Kleeorin, N., K\"apyl\"a, P. J., \& \Brandenburg\ypreN{2011}{84}{056314}
{Pumping velocity in homogeneous helical turbulence with shear}
\arxivB{1105.5785}\adsT{2011PhRvE..84e6314R}\doiT{10.1103/PhysRevE.84.056314}\ownE{2011/Roga_etal11}
%X

\item[{250.}~]
Plasson, R., \Brandenburg, Jullien, L., \& Bersini, H.\yjour{2011}{Artif. Life}{17}{219}
{236}{Autocatalysis: at the root of self-replication}
\doiB{10.1162/artl_a_00033}\ownE{2011/Plasson_etal11b}
%DIVA

\item[{249.}~]
Hubbard, A., Rheinhardt, M. \& \Brandenburg\yanaN{2011}{535}{A48}
{The fratricide of alpha--Omega dynamos by their alpha squared siblings}
\arxivB{1102.2617}\adsT{2011A\%26A...535A..48H}\doiT{10.1051/0004-6361/201116705}\ownE{2011/Hubb+Rhei+Bran11}
%X

\item[{248.}~]
R\"adler, K.-H., \Brandenburg, Del Sordo, F., \& Rheinhardt, M.\ypreN{2011}{84}{4}
{Mean-field diffusivities in passive scalar and magnetic transport
in irrotational flows}
\arxivB{1104.1613}\adsT{2011PhRvE..84d6321R}\doiT{10.1103/PhysRevE.84.046321}\ownE{2011/Radl_etal11}
%X

\item[{247.}~]
\Brandenburg\yapjN{2011}{741}{92}
{Nonlinear small-scale dynamos at low magnetic Prandtl numbers}
\arxivB{1106.5777}\adsT{2011ApJ...741...92B}\doiT{10.1088/0004-637X/741/2/92}\ownE{2011/Bran11}
%is on solar physics archive

\item[{246.}~]
\Brandenburg, Kemel, K., Kleeorin, N., Mitra, D., \& Rogachevskii, I.\yapjlN{2011}{740}{L50}
{Detection of negative effective magnetic pressure instability in turbulence simulations}
\arxivB{1109.1270}\adsT{2011ApJ...740L..50B}\doiT{10.1088/2041-8205/740/2/L50}\httpT{http://www.nordita.org/~brandenb/projects/NEMPI/}\ownE{2011/Bran_etal11}
%is on solar physics archive

\item[{245.}~]
Chatterjee, P., Mitra, D., Rheinhardt, M., \& \Brandenburg\yanaN{2011}{534}{A46}
{Alpha effect due to buoyancy instability of a magnetic layer}
\arxivB{1011.1218}\adsT{2011A\%26A...534A..46C}\doiT{10.1051/0004-6361/201016108}\ownE{2011/Chat+Mitra+Rhei+Bran11}
%X

\item[{244.}~]
Warnecke, J., \Brandenburg, \& Mitra, D.\yanaN{2011}{534}{A11}
{Dynamo-driven plasmoid ejections above a spherical surface}
\arxivB{1104.0664}\adsT{2011A\%26A...534A..11W}\doiT{10.1051/0004-6361/201117023}\ownE{2011/Warn+Bran+Mitra11}
%is on solar physics archive

\item[{243.}~]
Chatterjee, P., Mitra, D., \Brandenburg, \& Rheinhardt, M.\ypreN{2011}{84}{025403R}
{Spontaneous chiral symmetry breaking by hydromagnetic buoyancy}
\arxivB{1011.1251}\adsT{2011PhRvE..84b5403C}\doiT{10.1103/PhysRevE.84.025403}\ownE{2011/Chat+Mitra+Bran+Rhei11}
%X

\item[{242.}~]
Bejarano, C., Gomez, D. O., \& \Brandenburg\yapjN{2011}{737}{62}
{Shear-driven instabilities in Hall-magnetohydrodynamic plasmas}
\arxivB{1012.5284}\adsT{2011ApJ...737...62B}\doiT{10.1088/0004-637X/737/2/62}\ownE{2011/Beja+Gomez+Bran11}
%X

\item[{241.}~]
Candelaresi, S., \& \Brandenburg\ypreN{2011}{84}{016406}
{Decay of helical and non-helical magnetic knots}
\arxivB{1103.3518}\adsT{2011PhRvE..84a6406C}\doiT{10.1103/PhysRevE.84.016406}\ownE{2011/Cand+Bran11}
%DIVA

\item[{240.}~]
Plasson, R., \Brandenburg, Jullien, L., \& Bersini, H.\yjour{2011}{J. Phys. Chem. A}{115}{8073}
{8085}{Autocatalyses}
\arxivB{1006.2634}\doiT{10.1021/jp110079p}\ownE{2011/Plasson_etal11}
%DIVA

\item[{239.}~]
\Brandenburg\yjour{2011}{Pramana J. Phys.}{77}{67}
{76}{Chandrasekhar-Kendall functions in astrophysical dynamos}
\arxivB{1103.4976}\adsT{2011Prama..77...67B}\doiT{http://dx.doi.org/10.1007/s12043-011-0112-5}\ownE{2011/Bran11_chandra}
%DIVA

\item[{238.}~]
K\"apyl\"a, P. J., Mantere, M. J., Guerrero, G., \Brandenburg, \& Chatterjee, P.\yanaN{2011}{531}{A162}
{Reynolds stress and heat flux in spherical shell convection}
\arxivB{1010.1250}\adsT{2011A\%26A...531A.162K}\doiT{10.1051/0004-6361/201015884}\ownE{2011/Kapy_etal11}
%DIVA

\item[{237.}~]
Rempel, E. L., Chian, A. C.-L., \& \Brandenburg\yapjN{2011}{735}{L9}
{Lagrangian coherent structures in a nonlinear dynamo}
\arxivB{1011.6327}\adsT{2011ApJ...735L...9R}\doiT{10.1088/2041-8205/735/1/L9}\ownE{2011/Remp+Chian+Bran11}
%DIVA

\item[{236.}~]
\Brandenburg, Subramanian, K., Balogh, A., \& Goldstein, M. L.\yapjN{2011}{734}{9}
{Scale dependence of magnetic helicity in the solar wind}
\arxivB{1101.1709}\adsT{2011ApJ...734....9B}\doiT{10.1088/0004-637X/734/1/9}\ownE{2011/BSBG11}
%is on solar physics archive
%DIVA

\item[{235.}~]
Del Sordo, F., \& \Brandenburg\yanaN{2011}{528}{A145}
{Vorticity production through rotation, shear, and baroclinicity}
\arxivB{1008.5281}\adsT{2011A\%26A...528A.145D}\doiT{10.1051/0004-6361/201015661}\ownE{2011/DSor+Bran11}
%DIVA

\item[{234.}~]
\Brandenburg, \& Nordlund, \AA.\yjourN{2011}{Rep.\ Prog.\ Phys.}{74}{046901}
{Astrophysical turbulence modeling}
\arxivB{0912.1340}\adsT{2011RPPh...74d6901B}\doiT{10.1088/0034-4885/74/4/046901}\ownE{2011/Bran+Nord11}
%DIVA

\item[{233.}~]
R\"udiger, G., Kitchatinov, L. L., \& \Brandenburg\ysph{2011}{269}{3}
{12}{Cross helicity and turbulent magnetic diffusivity in the solar convection zone}
\arxivB{1004.4881}\adsT{2011SoPh..269....3R}\doiT{10.1007/s11207-010-9683-4}\ownE{2011/Rue+Kit+Bra11}
%X

\item[{232.}~]
Mitra, D., Moss, D., Tavakol, R., \& \Brandenburg\yanaN{2011}{526}{A138}
{Alleviating alpha quenching by solar wind and meridional flow}
\arxivB{1008.4226}\adsT{2011A\%26A...526A.138M}\doiT{10.1051/0004-6361/201015637}\ownE{2011/Mitra_etal11}
%DIVA

\item[{231.}~]
\Brandenburg, Haugen, N. E. L., \& Babkovskaia, N.\ypreN{2011}{83}{016304}
{Turbulent front speed in the Fisher equation: dependence on Damk\"ohler number}
\arxivB{1008.5145}\adsT{2011PhRvE..83a6304B}\doiT{10.1103/PhysRevE.83.016304}\ownE{2011/Bran+Hau+Babk11}
%DIVA

\item[{230.}~]
Candelaresi, S., Hubbard, A., \Brandenburg, \& Mitra, D.\yppN{2011}{18}{012903}
{Magnetic helicity transport in the advective gauge family}
\arxivB{1010.6177}\adsT{2011PhPl...18a2903C}\doiT{10.1063/1.3533656}\ownE{2011/Cand+Hubb+Bran+Mitra11}
%DIVA

\item[{229.}~]
Hubbard, A., \& \Brandenburg\yapjN{2011}{727}{11}
{Magnetic helicity flux in the presence of shear}
\arxivB{1006.3549}\adsT{2011ApJ...727...11H}\doiT{10.1088/0004-637X/727/1/11}\ownE{2011/Hubb+Bran11}
%DIVA

\item[{228.}~]
\Brandenburg\yan{2011}{332}{51}
{56}{Dissipation in dynamos at low and high magnetic Prandtl numbers}
\arxivB{1010.4805}\adsT{2011AN....332...51B}\doiT{10.1002/asna.201011478}\ownE{2011/B11}
%DIVA

\item[{227.}~]
Chatterjee, P., Guerrero, G., \& \Brandenburg\yanaN{2011}{525}{A5}
{Magnetic helicity fluxes in interface and flux transport dynamos}
\arxivB{1005.5335}\adsT{2011A\%26A...525A...5C}\doiT{10.1051/0004-6361/201015073}\ownE{2011/Chat+Guer+Bran11}
%DIVA

\item[{226.}~]
Babkovskaia, N., Haugen, N. E. L., \Brandenburg\yjour{2011}{J.\ Comp.\ Phys.}{230}{1}
{12}{A high-order public domain code for direct numerical simulations of turbulent combustion}
\arxivB{1005.5301}\adsT{2011JCoPh.230....1B}\doiT{10.1016/j.jcp.2010.08.028}\ownE{2011/Babk+Haug+Bran11}
%DIVA

\item[{225.}~]
\Brandenburg, Chatterjee, P., Del Sordo, F., Hubbard, A., K\"apyl\"a, P. J., \& Rheinhardt, M.\ypsN{2010}{T142}{014028}
{Turbulent transport in hydromagnetic flows}
\arxivB{1004.5380}\adsT{2010PhST..142a4028B}\doiT{10.1088/0031-8949/2010/T142/014028}\ownE{2010/Bran_etal10}

\item[{224.}~]
Guerrero, G., Chatterjee, P., \& \Brandenburg,\ymn{2010}{409}{1619}
{1630}{Shear-driven and diffusive helicity fluxes in alpha--Omega dynamos}
\arxivB{1005.4818}\adsT{2010MNRAS.409.1619G}\doiT{10.1111/j.1365-2966.2010.17408.x}\ownE{2010/Guer+Chat+Bran10}

\item[{223.}~]
Chatterjee, P., \Brandenburg, \& Guerrero, G.\ygafdS{2010}{104}{591}
{599}{Can catastrophic quenching be alleviated by separating shear and alpha effect?}
\arxivB{1005.5708}\adsT{2010GApFD.104..591C}\doiT{10.1080/03091929.2010.504185}\ownE{2010/Chat+Bran+Guer10}

\item[{222.}~]
Hubbard, A., \& \Brandenburg\ygafd{2010}{104}{577}
{590}{Magnetic helicity fluxes in an alpha squared dynamo embedded in a halo}
\arxivB{1004.4591}\adsT{2010GApFD.104..577H}\doiT{10.1080/03091929.2010.506438}\ownE{2010/Hubb+Bran10}

\item[{221.}~]
Warnecke, J., \& \Brandenburg\yanaN{2010}{523}{A19}
{Surface appearance of dynamo-generated large-scale fields}
\arxivB{1002.3620}\adsT{2010A\%26A...523A..19W}\doiT{10.1051/0004-6361/201014287}\ownE{2010/Warn+Bran10}

\item[{220.}~]
Rheinhardt, M., \& \Brandenburg\yanaN{2010}{520}{A28}
{Test-field method for mean-field coefficients with MHD background}
\arxivB{1004.0689}\adsT{2010A\%26A...520A..28R}\doiT{10.1051/0004-6361/201014700}\ownE{2010/Rhei+Bran10}

\item[{219.}~]
K\"apyl\"a, P. J., Korpi, M. J., \& \Brandenburg\yanaN{2010}{518}{A22}
{Open and closed boundaries in large-scale convective dynamos}
\arxivB{0911.4120}\adsT{2010A\%26A...518A..22K}\doiT{10.1051/0004-6361/200913722}\ownE{2010/Kapy+Korp+Bran10b}

\item[{218.}~]
K\"apyl\"a, P. J., \Brandenburg, Korpi, M. J., Snellman, J. E., \&
Narayan, R.\yapj{2010}{719}{67}
{76}{Angular momentum transport in convectively unstable shear flows}
\arxivB{1003.0900}\adsT{2010ApJ...719...67K}\doiT{10.1088/0004-637X/719/1/67}\ownE{2010/Kapy_etal10b}

\item[{217.}~]
Mitra, D., Tavakol, R., K\"apyl\"a, P. J., \& \Brandenburg\yapjl{2010}{719}{L1}
{L4}{Oscillatory migrating magnetic fields in helical turbulence
in spherical domains}
\arxivB{0901.2364}\adsT{2010ApJ...719L...1M}\doiT{10.1088/2041-8205/719/1/L1}\ownE{2010/Mitra_etal10b}

\item[{216.}~]
Madarassy, E. J. M., \& \Brandenburg\ypreN{2010}{82}{016304}
{Calibrating passive scalar transport in shear-flow turbulence}
\arxivB{0906.3314}\adsT{2010PhRvE..82a6304M}\doiT{10.1103/PhysRevE.82.016304}\ownE{2010/Mada+Bran10}

\item[{215.}~]
Kahniashvili, T., \Brandenburg, Tevzadze, A. G., \& Ratra, B.\yprdN{2010}{81}{123002}
{Numerical simulations of the decay of primordial magnetic turbulence}
\arxivB{1004.3084}\adsT{2010PhRvD..81l3002K}\doiT{10.1103/PhysRevD.81.123002}\ownE{2010/Kahn_etal10}
%Jref

\item[{214.}~]
Del Sordo, F., Candelaresi, S., \& \Brandenburg\ypreN{2010}{81}{036401}
{Magnetic field decay of three interlocked flux rings with zero linking number}
\arxivB{0910.3948}\adsT{2010PhRvE..81c6401D}\doiT{10.1103/PhysRevE.81.036401}\ownE{2010/DSor+Cand+Bran10}

\item[{213.}~]
K\"apyl\"a, P. J., Korpi, M. J., \& \Brandenburg\ymn{2010}{402}{1458}
{1466}{The alpha effect in rotating convection with sinusoidal shear}
\arxivB{0908.2423}\adsT{2010MNRAS.402.1458K}\doiT{10.1111/j.1365-2966.2009.16004.x}\ownE{2010/Kapy+Korp+Bran10}

\item[{212.}~]
Plasson, R., \& \Brandenburg\yoleb{2010}{40}{93}
{110}{Homochirality and the need for energy}
\arxivB{0908.0658}\adsT{2010OLEB...40...93P}\doiT{10.1007/s11084-009-9181-6}\ownE{2010/Plass+Bran10}

\item[{211.}~]
\Brandenburg\ymn{2010}{401}{347}
{354}{Magnetic field evolution in simulations with Euler potentials}
\arxivB{0907.1906}\adsT{2010MNRAS.401..347B}\doiT{10.1111/j.1365-2966.2009.15640.x}\ownE{2010/Bran10_euler}

\item[{210.}~]
Mitra, D., Candelaresi, S., Chatterjee, P., Tavakol, R., \& \Brandenburg\yan{2010}{331}{130}
{135}{Equatorial magnetic helicity flux in simulations with different gauges}
\arxivB{0911.0969}\adsT{2010AN....331..130M}\doiT{10.1002/asna.200911308}\ownE{2010/Mitra_etal10}

\item[{209.}~]
R\"adler, K.-H., \& \Brandenburg\yan{2010}{331}{14}
{21}{Mean electromotive force proportional to mean flow in mhd turbulence}
\arxivB{0910.0071}\adsT{2010AN....331...14R}\doiT{10.1002/asna.200911313}\ownE{2010/Radl+Bran10}

\item[{208.}~]
\Brandenburg, Kleeorin, N., \& Rogachevskii, I.\yan{2010}{331}{5}
{13}{Large-scale magnetic flux concentrations from turbulent stresses}
\arxivB{0910.1835}\adsT{2010AN....331....5B}\doiT{10.1002/asna.200911311}\ownE{2010/Bran+Klee+Roga10}

\item[{207.}~]
K\"apyl\"a, P. J., Korpi, M. J., \Brandenburg, Mitra, D., \& Tavakol, R.\yan{2010}{331}{73}
{81}{Convective dynamos in spherical wedge geometry}
\arxivB{0909.1330}\adsT{2010AN....331...73K}\doiT{10.1002/asna.200911252}\ownE{2010/Kapy_etal10}

\item[{206.}~]
Hubbard, A., \& \Brandenburg\yapj{2009}{706}{712}
{726}{Memory effects in turbulent transport}
\arxivB{0811.2561}\adsT{2009ApJ...706..712H}\doiT{10.1088/0004-637X/706/1/712}\ownE{2009/Hubb+Bran09}

\item[{205.}~]
\Brandenburg\yjourN{2009}{Plasma Phys. Control. Fusion}{51}{124043}
{The critical role of magnetic helicity in astrophysical dynamos}
\arxivB{0909.4377}\adsT{2009PPCF...51l4043B}\doiT{10.1088/0741-3335/51/12/124043}\ownE{2009/Bran09_sofia}

\item[{204.}~]
Sur, S., \& \Brandenburg\ymn{2009}{399}{273}
{280}{The role of the Yoshizawa effect in the Archontis dynamo}
\arxivB{0902.2394}\adsT{2009MNRAS.399..273S}\ownE{2009/Sur+Bran09}
%w/ Astronomy

\item[{203.}~]
Hubbard, A., Del Sordo, F., K\"apyl\"a, P. J., \& \Brandenburg\ymn{2009}{398}{1891}
{1899}{The alpha effect with imposed and dynamo-generated magnetic fields}
\arxivB{0904.2773}\adsT{2009MNRAS.398.1891H}\ownE{2009/Hubb+DSor+Kapy+Bran09}
%w/ Astronomy

\item[{202.}~]
\Brandenburg, Candelaresi, S., \& Chatterjee, P.\ymn{2009}{398}{1414}
{1422}{Small-scale magnetic helicity losses from a mean-field dynamo}
\arxivB{0905.0242}\adsT{2009MNRAS.398.1414B}\doiT{10.1111/j.1365-2966.2009.15188.x}\ownE{2009/Bran+Cand+Chat09}
%w/ Astronomy

\item[{201.}~]
Vermersch, V., \& \Brandenburg\yan{2009}{330}{797}
{806}{Shear-driven magnetic buoyancy oscillations}
\arxivB{0909.0324}\adsT{2009AN....330..797V}\ownE{2009/Verm+Bran09}
%w/ Astronomy

\item[\important {200.}~]
K\"apyl\"a, P. J., Korpi, M. J., \& \Brandenburg\yana{2009}{500}{633}
{646}{Alpha effect and turbulent diffusion from convection}
\arxivB{0812.1792}\adsT{2009A\%26A...500..633K}\ownE{2009/Kapy+Korp+Bran09b}
%w/o Astronomy

\item[{199.}~]
K\"apyl\"a, P. J., \& \Brandenburg\yapj{2009}{699}{1059}
{1166}{Turbulent dynamos with shear and fractional helicity}
\arxivB{0810.2298}\adsT{2009ApJ...699.1059K}\ownE{2009/Kapy+Bran09}
%w/o Astronomy

\item[{198.}~]
\Brandenburg\yapj{2009}{697}{1206}
{1213}{Large-scale dynamos at low magnetic Prandtl numbers}
\arxivB{0808.0961}\adsT{2009ApJ...697.1206B}\doiT{10.1088/0004-637X/697/2/1206}\ownE{2009/B09}

\item[{197.}~]
K\"apyl\"a, P. J., Korpi, M. J., \& \Brandenburg\yapj{2009}{697}{1153}
{1163}{Large-scale dynamos in rigidly rotating turbulent convection}
\arxivB{0812.3958}\adsT{2009ApJ...697.1153K}\ownE{2009/Kapy+Korp+Bran09a}

\item[{196.}~]
Mitra, D., Tavakol, R., \Brandenburg, \& Moss, D.\yapj{2009}{697}{923}
{933}{Turbulent dynamos in spherical shell segments of varying geometrical extent}
\arxivB{0812.3106}\adsT{2009ApJ...697..923M}\ownE{2009/Mit+Tav+Bra+Mos09}

\item[{195.}~]
\Brandenburg, Svedin, A., \& Vasil, G. M.\ymn{2009}{395}{1599}
{1606}{Turbulent diffusion with rotation or magnetic fields}
\arxivB{0901.2112}\adsT{2009MNRAS.395.1599B}\doiT{10.1111/j.1365-2966.2009.14646.x}\ownE{2009/Bran+Svedin+Vasil09}

\item[{194.}~]
\Brandenburg\yjour{2009}{Spa.\ Sci.\ Rev.}{144}{87}
{104}{Advances in theory and simulations of large-scale dynamos}
\arxivB{0901.0329}\adsT{2009SSRv..144...87B}\doiT{10.1007/s11214-009-9490-0}\ownE{2009/B09_bern}

\item[{193.}~]
Mitra, D., K\"apyl\"a, P. J., Tavakol, R., \& \Brandenburg\yana{2009}{495}{1}
{8}{Alpha effect and diffusivity in helical turbulence with shear}
\arxivB{0806.1608}\adsT{2009A\%26A...495....1M}\ownE{2009/Mitra_etal09a}

\item[{192.}~]
R\"adler, K.-H., \& \Brandenburg\ymn{2009}{393}{113}
{125}{Mean-field effects in the Galloway-Proctor flow}
\arxivB{0809.0851}\adsT{2009MNRAS.393..113R}\ownE{2009/Radl+Bran09}

\item[{191.}~]
Liljestr\"om, A. J., Korpi, M. J., K\"apyl\"a, P. J., \Brandenburg, \&
Lyra, W.\yan{2009}{330}{92}
{99}{Turbulent stresses as a function of shear rate in a local disk model}
\arxivB{0811.2341}\adsT{2009AN....330...92L}\ownE{2009/Liljestrom_etal09}

\item[{190.}~]
K\"apyl\"a, P. J., Mitra, D., \& \Brandenburg\ypreN{2009}{79}{016302}
{Numerical study of large-scale vorticity generation in shear-flow turbulence}
\arxivB{0810.0833}\adsT{2009PhRvE..79a6302K}\ownE{2009/Kapy+Mitra+Bran09}

\item[{189.}~]
Tilgner, A., \& \Brandenburg\ymn{2008}{391}{1477}
{1481}{A growing dynamo from a saturated Roberts flow dynamo}
\arxivB{0808.2141}\adsT{2008MNRAS.391.1477T}\ownE{2008/Tilg+Bran08}

\item[\important \relevant {188.}~]
K\"apyl\"a, P. J., Korpi, M. J., \& \Brandenburg\yana{2008}{491}{353}
{362}{Large-scale dynamos in turbulent convection with shear}
\arxivB{0806.0375}\adsT{2008A\%26A...491..353K}\ownE{2008/Kapy+Korp+Bran08}

\item[{187.}~]
\Brandenburg, R\"adler, K.-H., Rheinhardt, M., \& Subramanian, K.\yapjl{2008}{687}{L49}
{L52}{Magnetic quenching of alpha and diffusivity tensors in helical turbulence}
\arxivB{0805.1287}\adsT{2008ApJ...687L..49B}\ownE{2008/BRRS08}

\item[{186.}~]
K\"apyl\"a, P. J., \& \Brandenburg\yana{2008}{488}{9}
{23}{Lambda effect from forced turbulence simulations}
\arxivB{0806.3751}\adsT{2008A\%26A...488....9K}\ownE{2008/Kapy+Bran08}

\item[{185.}~]
\Brandenburg\yan{2008}{329}{725}
{731}{The dual role of shear in large-scale dynamos}
\arxivB{0808.0959}\adsT{2008AN....329..725B}\ownE{2008/Bran08_dual}

\item[{184.}~]
\Brandenburg\ypsN{2008}{T130}{014016}
{Turbulent protostellar discs}
\arxivB{0808.0960}\adsT{2008PhST..130a4016B}\ownE{2008/Bran08_disc}

\item[{183.}~]
Jouve, L., Brun, A. S., Arlt, R., \Brandenburg, Dikpati, M.,
Bonanno, A., K\"apyl\"a, P. J., Moss, D., Rempel, M., Gilman, P.,
Korpi, M. J., \& Kosovichev, A. G.\yana{2008}{483}{949}
{960}{A solar mean field dynamo benchmark}
\adsB{2008A\%26A...483..949J}\ownE{2008/Jouve_etal08}

\item[{182.}~]
Babkovskaia, N., \Brandenburg, \& Poutanen, J.\ymn{2008}{386}{1038}
{1044}{Boundary layer on the surface of a neutron star}
\arxivB{0802.1663}\adsT{2008MNRAS.386.1038B}\httpT{http://www.nordita.org/~brandenb/projects/nstar/}\ownE{2008/Babk+Bran+Pout08}

\item[{181.}~]
Dib, S., \Brandenburg, Kim, J., Gopinathan, M.,
\& Andre, P.\yapj{2008}{678}{L105}
{L108}{Core mass function: the role of gravity}
\arxivB{0801.2257}\adsT{2008ApJ...678L.105D}\ownE{2008/Dib_etal08}

\item[{180.}~]
\Brandenburg, R\"adler, K.-H., \& Schrinner, M.\yana{2008}{482}{739}
{746}{Scale dependence of alpha effect and turbulent diffusivity}
\arxivB{0801.1320}\adsT{2008A\%26A...482..739B}\ownE{2008/Bran+Radl+Schr08}

\item[{179.}~]
\Brandenburg, \& Spiegel, E. A.\yan{2008}{329}{351}
{358}{Modeling a Maunder minimum}
\arxivB{0801.2156}\adsT{2008AN....329..351B}\ownE{2008/Bran+Spie08}

\item[{178.}~]
Sur, S., \Brandenburg, \& Subramanian, K.\ymn{2008}{385}{L15}
{L19}{Kinematic alpha effect in isotropic turbulence simulations}
\arxivB{0711.3789}\adsT{2008MNRAS.385L..15S}\ownE{2008/Sur_etal08}

\item[\important \relevant {177.}~]
\Brandenburg, R\"adler, K.-H., Rheinhardt, M., \& K\"apyl\"a, P. J.\yapj{2008}{676}{740}
{751}{Magnetic diffusivity tensor and dynamo effects in rotating
and shearing turbulence}
\arxivB{0710.4059}\adsT{2008ApJ...676..740B}\ownE{2008/Bran+Radl+Rhei+Kap08}

\item[{176.}~]
R\"adler, K.-H., \& \Brandenburg\ypreN{2008}{77}{026405}
{Alpha--effect dynamos with zero kinetic helicity}
\arxivB{0801.0602}\adsT{2008PhRvE..77b6405R}\ownE{2008/Radl+Bran08}

\item[{175.}~]
\Brandenburg, K\"apyl\"a, P. J., Mitra, D., Moss, D., \& Tavakol, R.\yan{2007}{328}{1118}
{1121}{The helicity constraint in spherical shell dynamos}
\arxivB{0711.3616}\adsT{2007AN....328.1118B}\ownE{2007/Bran+Kapy+Mitra+Moss+Tava07}

\item[{174.}~]
K\"apyl\"a, P. J., \& \Brandenburg\yan{2007}{328}{1006}
{1008}{Turbulent viscosity and $\Lambda$-effect
from numerical turbulence models}
\arxivB{0710.5632}\adsT{2007AN....328.1006K}\ownE{2007/Kapy+Bran07}

\item[{173.}]
\Brandenburg, Lehto, H. J., \& Lehto, K. M.\yab{2007}{7}{725}
{732}{Homochirality in an early peptide world}
\qbioB{0610051}\adsT{2007AsBio...7..725B}\ownE{2007/Bran+Lehtos07}

\item[{172.}]
\Brandenburg, \& K\"apyl\"a, P. J.\yjour{2007}{New J.\ Phys.}{9}{305, 1}
{24}{Magnetic helicity effects in astrophysical and laboratory dynamos}
\arxivB{0705.3507}\adsT{2007NJPh....9..305B}\ownE{2007/Bran+Kapy07}

\item[{171.}]
\Brandenburg, \& Subramanian, K.\yan{2007}{328}{507}
{512}{Simulations of the anisotropic kinetic and magnetic alpha effects}
\arxivB{0705.3508}\adsT{2007AN....328..507B}\ownE{2007/Bran+Sub07}

\item[{170.}]
Sur, S., Subramanian, K., \& \Brandenburg\ymn{2007}{376}{1238}
{1250}{Kinetic and magnetic alpha effects in non-linear dynamo theory}
\astrophB{0701001}\adsT{2007MNRAS.376.1238S}\ownE{2007/Sur_etal07}

\item[{169.}]
\Brandenburg, Korpi, M. J., \& Mee, A. J.\yapj{2007}{654}{945}
{954}{Thermal instability in shearing and periodic turbulence}
\astrophB{0604244}\adsT{2007ApJ...654..945B}\doiT{10.1086/509143}\httpT{http://www.nordita.org/~brandenb/projects/galti/}\ownE{2007/Bran+Korp+Mee07}

\item[{168.}]
Snodin, A. P., \Brandenburg, Mee, A. J., \& Shukurov, A.\ymn{2006}{373}{643}
{652}{Simulating field-aligned diffusion of a cosmic ray gas}
\astrophB{0507176}\adsT{2006MNRAS.373..643S}\ownE{2006/Snodin_etal06}

\item[\relevant {167.}]
Subramanian, K., \& \Brandenburg\yapj{2006}{648}{L71}
{L74}{Magnetic helicity density and its flux in weakly inhomogeneous turbulence}
\astrophB{0509392}\adsT{2006ApJ...648L..71S}\ownE{2006/Sub+Bran06}
%is on solar physics archive

\item[{166.}]
Haugen, N. E. L., \& \Brandenburg\ypfN{2006}{18}{075106}
{Hydrodynamic and hydromagnetic energy spectra from large eddy simulations}
\astrophB{0412666}\adsT{2006PhFl...18g5106H}\doiT{10.1063/1.2222399}\httpT{http://www.nordita.org/~brandenb/projects/HydroMHDspec_from_LES/}\ownE{2006/Hau+Bran06}

\item[{165.}]
Gustafsson, M., \Brandenburg, Lemaire, J. L., \& Field, D.\yana{2006}{454}{815}
{825}{The nature of turbulence in OMC1 at the star forming scale:
observations and simulations}
\astrophB{0512214}\adsT{2006A\%26A...454..815G}\httpT{http://www.nordita.org/~brandenb/projects/maiken/}\ownE{2006/Gustafsson_etal06}

\item[{164.}]
Mee, A. J., \& \Brandenburg\ymn{2006}{370}{415}
{419}{Turbulence from localized random expansion waves}
\astrophB{0602057}\adsT{2006MNRAS.370..415M}\ownE{2006/Mee+Bran06}

\item[{163.}]
\Brandenburg, \& Dintrans, B.\yana{2006}{450}{437}
{444}{Nonaxisymmetric stability in the shearing sheet approximation}
\astrophB{0111313}\adsT{2006A\%26A...450..437B}\ownE{2006/Bran+Din06}

\item[{162.}]
\Brandenburg\yan{2006}{327}{123}
{130}{Magnetic helicity in primordial and dynamo scenarios of galaxies}
\astrophB{0601496}\adsT{2006AN....327..461B}\ownE{2006/B06_bologna}

\item[{161.}]
Shukurov, A., Sokoloff, D., Subramanian, K., \& \Brandenburg\yana{2006}{448}{L33}
{L36}{Galactic dynamo and helicity losses through fountain flow}
\astrophB{0512592}\adsT{2006A\%26A...448L..33S}\ownE{2006/Shuk+Sok+Sub+Bran06}

\item[{160.}]
Heinemann, T., Dobler, W., Nordlund, \AA., \& \Brandenburg\yana{2006}{448}{731}
{737}{Radiative transfer in decomposed domains}
\astrophB{0503510}\adsT{2006A\%26A...448..731H}\ownE{2006/Heinemann_etal06}
%is on solar physics archive

\item[\important {159.}]
Dobler, W., Stix, M., \& \Brandenburg\yapj{2006}{638}{336}
{347}{Convection and magnetic field generation in fully convective spheres}
\astrophB{0410645}\adsT{2006ApJ...638..336D}\ownE{2006/Dobl+Stix+Bran06}

\item[{158.}]
von Rekowski, B., \& \Brandenburg\yan{2006}{327}{53}
{71}{Stellar dynamo driven wind braking versus disc coupling}
\astrophB{0504053}\adsT{2006AN....327...53V}\ownE{2006/vReko+Bran06}

\item[{157.}]
Nilsson, M., \Brandenburg, Andersen, A. C., \&
H\"ofner, S.\yija{2005}{4}{233}
{239}{Unidirectional polymerization leading to homochirality in the RNA world}
\qbioB{0505041}\adsT{2005IJAsB...4..233N}\ownE{2005/Nil+Bran+And+Hof05}

\item[{156.}]
\Brandenburg, Andersen, A. C., \& Nilsson, M.\yoleb{2005}{35}{507}
{521}{Dissociation in a polymerization model of homochirality}
\qbioB{0502008}\adsT{2005OLEB...35..507B}\ownE{2005/Bran+And+Nil05}

\item[{155.}]
\Brandenburg\yan{2005}{326}{787}
{797}{Turbulence and its parameterization in accretion discs}
\astrophB{0510015}\adsT{2005AN....326..787B}\ownE{2005/B05_QPO}

\item[{154.}]
Multam\"aki, T., \& \Brandenburg\yija{2005}{4}{75}
{80}{Spatial dynamics of homochiralization}
\qbioB{0505040}\adsT{2005IJAsB...4...75M}\ownE{2005/Mult+Bran05}

\item[\important \relevant {153.}]
\Brandenburg, \& Subramanian, K.\yjour{2005}{Phys.\ Rep.}{417}{1}
{209}{Astrophysical magnetic fields and nonlinear dynamo theory}
\astrophB{0405052}\adsT{2005PhR...417....1B}\doiT{10.1016/j.physrep.2005.06.005}\ownE{2005/Bran+Sub05}
%is on solar physics archive

\item[{152.}]
\Brandenburg, \& Subramanian, K.\yana{2005}{439}{835}
{843}{Minimal tau approximation and simulations of the alpha effect}
\astrophB{0504222}\adsT{2005A\%26A...439..835B}\ownE{2005/Bran+Sub05a}

\item[{151.}]
\Brandenburg, Chan, K. L., Nordlund, \AA., \& Stein, R. F.\yan{2005}{326}{681}
{692}{Effect of the radiative background flux in convection}
\astrophB{0508404}\adsT{2005AN....326..681B}\ownE{2005/Bran+Chan+Nor+Stein05}
%is on solar physics archive

\item[{150.}]
\Brandenburg, Andersen, A. C., H\"ofner, S., \& Nilsson, M.\yoleb{2005}{35}{225}
{241}{Homochiral growth through enantiomeric cross-inhibition}
\qbioB{0401036}\adsT{2005OLEB...35..225B}\ownE{2005/Bran+And+Hof+Nil05}

\item[{149.}]
Dintrans, B., \Brandenburg, Nordlund, \AA.,
\& Stein, R. F.\yana{2005}{438}{365}
{376}{Spectrum and amplitudes of internal gravity waves excited by penetrative
convection in solar-type stars}
\astrophB{0502138}\adsT{2005A\%26A...438..365D}\ownE{2005/Dintrans_etal05}

\item[\relevant {148.}]
\Brandenburg, \& Subramanian, K.\yan{2005}{326}{400}
{408}{Strong mean field dynamos require supercritical helicity fluxes}
\astrophB{0505457}\adsT{2005AN....326..400B}\doiT{10.1002/asna.200510362}\ownE{2005/Bran+Sub05b}
%is on solar physics archive

\item[{147.}]
Christensson, M., Hindmarsh, M., \& \Brandenburg\yan{2005}{326}{393}
{399}{Scaling laws in decaying helical 3D magnetohydrodynamic turbulence}
\astrophB{0209119}\adsT{2005AN....326..393C}\ownE{2005/Christensson_etal05}

\item[{146.}]
Schekochihin, A. A., Haugen, N. E. L., \Brandenburg, Cowley, S. C., Maron, J. L., \& McWilliams, J. C.\yapj{2005}{625}{L115}
{L118}{The onset of small scale dynamo at small magnetic Prandtl numbers}
\astrophB{0412594}\adsT{2005ApJ...625L.115S}\ownE{2005/Scheko_etal05}

\item[\important \relevant {145.}]
\Brandenburg\yapj{2005}{625}{539}
{547}{The case for a distributed solar dynamo shaped by near-surface shear}
\astrophB{0502275}\adsT{2005ApJ...625..539B}\doiT{10.1086/429584}\ownE{2005/B05}
%is on solar physics archive

\item[{144.}]
\Brandenburg, Haugen, N. E. L., K\"apyl\"a, P. J., \& Sandin, C.\yan{2005}{326}{174}
{185}{The problem of small and large scale fields in the solar dynamo}
\astrophB{0412364}\adsT{2005AN....326..174B}\ownE{2005/Bran+Hau+Kap+San05}

\item[143.]
\Brandenburg, \& Multam\"aki, T.\yijaS{2004}{3}{209}
{219}{How long can left and right handed life forms coexist?}
\qbioB{0407008}\adsT{2004IJAsB...3..209B}\ownE{2004/Bran+Mult04}

\item[\relevant {142.}]
Subramanian, K., \& \Brandenburg\yprlN{2004}{93}{205001}
{Nonlinear current helicity fluxes in turbulent dynamos and alpha quenching}
\astrophB{0408020}\adsT{2004PhRvL..93t5001S}\ownE{2004/Sub+Bran04}

\item[{141.}]
Pearson, B. R., Yousef, T. A., Haugen, N. E. L., \Brandenburg, \&
Krogstad, P. \AA.\ypreN{2004}{70}{056301}
{Delayed correlation between turbulent energy dissipation and injection}
\physicsB{0404114}\adsT{2004PhRvE..70e6301P}\ownE{2004/pearson_etal04}

\item[\relevant {140.}]
\Brandenburg, \& Sandin, C.\yana{2004}{427}{13}
{21}{Catastrophic alpha quenching alleviated by helicity flux and shear}
\astrophB{0401267}\adsT{2004A\%26A...427...13B}\ownE{2004/Bran+Sand04}
%is on solar physics archive

\item[139.]
von Rekowski, B., \Brandenburg, Dobler, W.,
\& Shukurov, A.\yass{2004}{292}{493}
{500}{Outflows from dynamo active protostellar accretion discs}
\astrophB{0306603}\adsT{2004Ap\%26SS.292..493V}\ownE{2004/vReko_etal04}

\item[138.]
Sarson, G. R., Shukurov, A., Nordlund, \AA., Gudiksen, B., \& \Brandenburg\yass{2004}{292}{267}
{272}{Self-regulating supernovae heating in interstellar medium simulations}
\astrophB{0307013}\adsT{2004Ap\%26SS.292..267S}\ownE{2004/Sars_etal04}

\item[137.]
Haugen, N. E. L., \Brandenburg, \& Dobler, W.\yass{2004}{292}{53}
{60}{High-resolution simulations of nonhelical MHD turbulence}
\astrophB{0306453}\adsT{2004Ap\%26SS.292...53H}\ownE{2004/Hau+Bran+Dobl04b}

\item[{136.}]
Haugen, N. E. L., \& \Brandenburg\ypreN{2004}{70}{036408}
{Suppression of small scale dynamo action by an imposed magnetic field}
\astrophB{0402281}\adsT{2004PhRvE..70c6408H}\ownE{2004/Hau+Bran04b}

\item[{135.}]
Haugen, N. E. L., \Brandenburg, \& Mee, A. J.\ymn{2004}{353}{947}
{952}{Mach number dependence of the onset of dynamo action}
\astrophB{0405453}\adsT{2004MNRAS.353..947H}\ownE{2004/Hau+Bran+Mee04}

\item[{134.}]
Haugen, N. E. L., \& \Brandenburg\ypreN{2004}{70}{026405}
{Inertial range scaling in numerical turbulence with hyperviscosity}
\astrophB{0402301}\adsT{2004PhRvE..70b6405H}\ownE{2004/Hau+Bran04a}

\item[\important 133.]
Haugen, N. E. L., \Brandenburg, \& Dobler, W.\ypreN{2004}{70}{016308}
{Simulations of nonhelical hydromagnetic turbulence}
\astrophB{0307059}\adsT{2004PhRvE..70a6308H}\doiT{10.1103/PhysRevE.70.016308}\ownE{2004/Hau+Bran+Dobl04}

\item[{132.}]
Dintrans, B., \& \Brandenburg\yana{2004}{421}{775}
{782}{Identification of gravity waves in hydrodynamical simulations}
\astrophB{0311094}\adsT{2004A\%26A...421..775D}\ownE{2004/Dintrans+Bran04}

\item[131.]
\Brandenburg, \& Matthaeus, W. H.\ypreN{2004}{69}{056407}
{Magnetic helicity evolution in a periodic domain with imposed field}
\astrophB{0305373}\adsT{2004PhRvE..69e6407B}\ownE{2004/Bran+Matth04}
%is on solar physics archive

\item[130.]
Yousef, T. A., Haugen, N. E. L., \& \Brandenburg\ypreN{2004}{69}{056303}
{Self-similar scaling in decaying numerical turbulence}
\astrophB{0312505}\adsT{2004PhRvE..69e6303Y}\ownE{2004/You+Hau+Bran04}

\item[129.]
von Rekowski, B., \& \Brandenburg\yana{2004}{420}{17}
{32}{Outflows and accretion in a star--disc system with stellar magnetosphere
and disc dynamo}
\astrophB{0307201}\adsT{2004A\%26A...420...17V}\ownE{2004/vReko+Bran04}

\item[128.]
Johansen, A., Andersen, A. C., \& \Brandenburg\yana{2004}{417}{361}
{371}{Simulations of dust-trapping vortices in protoplanetary discs}
\astrophB{0310059}\adsT{2004A\%26A...417..361J}\ownE{2004/Joh+And+Bran04}
% 0 citations (Jun 2004)

\item[127.]
\Brandenburg, K\"apyl\"a, P. J., \& Mohammed, A.\ypf{2004}{16}{1020}
{1027}{Non-Fickian diffusion and tau-approximation from numerical turbulence}
\astrophB{0306521}\adsT{2004PhFl...16.1020B}\ownE{2004/Bran+Kapy+Moha04}
% 1 citations (Jun 2004)

\item[126.]
Shukurov, A., Sarson, G. S., Nordlund, \AA., Gudiksen, B., \& \Brandenburg\yass{2004}{289}{319}
{322}{The effects of spiral arms on the multi-phase ISM}
\astrophB{0212260}\adsT{2004Ap\%26SS.289..319S}\ownE{2004/Shuk+Sars+Nord+Gudi+Bran04}
% 0 citations (Jun 2004)

\item[125.]
\Brandenburg, Blackman, E. G., \& Sarson, G. R.\yjour{2003}{Adv. Spa. Sci.}{32}{1835}
{1844}{How magnetic helicity ejection helps large scale dynamos}
\astrophB{0305374}\adsT{2003AdSpR..32.1835B}\ownE{2003/Bran+Black+Sars03}
%is on solar physics archive
% 0 citations (Jun 2004)

\item[124.]
Yousef, T. A., \Brandenburg, \& R\"udiger, G.\yana{2003}{411}{321}
{327}{Turbulent magnetic Prandtl number and magnetic diffusivity quenching from simulations}
\astrophB{0302425}\adsT{2003A\%26A...411..321Y}\ownE{2003/You+Bran+Rue03}
% 1 citations (Jun 2004)

\item[123.]
Haugen, N. E. L., \Brandenburg, \& Dobler, W.\yapjlS{2003}{597}{L141}
{L144}{Is nonhelical hydromagnetic turbulence peaked at small scales?}
\astrophB{0303372}\adsT{2003ApJ...597L.141H}\doiT{10.1086/380189}\ownE{2003/Hau+Bran+Dobl03}
% 6 citations (Jun 2004)

\item[122.]
Yousef, T. A., \& \Brandenburg\yana{2003}{407}{7}
{12}{Relaxation of writhe and twist of a bi-helical magnetic field}
\astrophB{0303148}\adsT{2003A\%26A...407....7Y}\ownE{2003/Yousef+Bran03}
% 2 citations (Jun 2004)
%is on solar physics archive
%< The lower curves, denoted by (i), are for $\nu=2\times\eta=10^{-3}$
%> The lower curves, denoted by (i), are for $\nu=\eta=2\times10^{-3}$

\item[121.]
Dobler, W., Haugen, N. E. L., Yousef, T. A., \& \Brandenburg\ypreN{2003}{68}{026304}
{Bottleneck effect in three-dimensional turbulence simulations}
\astrophB{0303324}\adsT{2003PhRvE..68b6304D}\ownE{2003/Dobl+Hau+You+Bran03}
% 3 citations (Jun 2004)

\item[120.]
R\"adler, K.-H., \& \Brandenburg\ypreN{2003}{67}{026401}
{Contributions to the theory of a two-scale homogeneous
dynamo experiment}
\physicsB{0208023}\adsT{2003PhRvE..67b6401R}\ownE{2003/Radl+Bran03}
% 0 citations (Jun 2004)

\item[\relevant 119.]
Blackman, E. G., \& \Brandenburg\yapjl{2003}{584}{L99}
{L102}{Doubly helical coronal ejections from dynamos
and their role in sustaining the solar cycle}
\astrophB{0212010}\adsT{2003ApJ...584L..99B}\ownE{2003/Black+Bran03}
%is on solar physics archive
% 5 citations (Jun 2004)

\item[118.]
von Rekowski, B., \Brandenburg, Dobler, W., \&,
Shukurov, A.\yana{2003}{398}{825}
{844}{Structured outflow from a dynamo active accretion disc}
\astrophB{0003174}\adsT{2003A\%26A...398..825V}\ownE{2003/vReko_etal03}
% 3 citations (Feb 2004)
% 4 citations (Jun 2004)

\item[117.]
Dintrans, B., \Brandenburg, Nordlund, \AA.,
\& Stein, R. F.\yass{2003}{284}{237}
{240}{Stochastic excitation of gravity waves by overshooting
convection in solar-type stars}
\astrophB{0403093}\adsT{2003Ap\%26SS.284..237D}\ownE{2003/Dintrans_etal03}
% 0 citations (Jun 2004)

\item[116.]
Ossendrijver, M., Stix, M., \Brandenburg, \&
R\"udiger, G.\yana{2002}{394}{735}
{745}{Magnetoconvection and dynamo coefficients: II.
Field direction dependent pumping of magnetic field}
\astrophB{0202299}\adsT{2002A\%26A...394..735O}\ownE{2002/Ossen_etal02}
% 6 citations (Jun 2004)

\item[\relevant 115.]
Blackman, E. G., \& \Brandenburg\yapj{2002}{579}{359}
{373}{Dynamic nonlinearity in large scale dynamos with shear}
\astrophB{0204497}\adsT{2002ApJ...579..359B}\ownE{2002/BB02}
% 3 citations (Mar 2003)
% 14 citations (Feb 2004)
% 17 citations (Jun 2004)

\item[114.]
Saar, S. H., \& \Brandenburg\yan{2002}{323}{357}
{360}{A new look at dynamo cycle amplitudes}
\astrophB{0207392}\adsT{2002AN....323..357S}\ownE{2002/Saar+Bran02}
% 0 citations (Jun 2004)

\item[113.]
\Brandenburg \& Dobler, W.\yan{2002}{323}{411}
{416}{Solar and stellar dynamos -- latest developments}
\astrophB{0207393}\adsT{2002AN....323..411B}\ownE{2002/Bran+Dobl_an02}
%http://www.aip.de/AN/movies
% 3 citations (Jun 2004)

\item[112.]
\Brandenburg, \& Sokoloff, D.\ygafd{2002}{96}{319}
{344}{Local and nonlocal magnetic diffusion and alpha-effect
tensors in shear flow turbulence}
\astrophB{0111568}\adsT{2002GApFD..96..319B}\ownE{2002/Bran+Soko02}
% 4 citations (Feb 2004)
% 5 citations (Jun 2004)

\item[111.]
\Brandenburg, \& Dobler, W.\yjour{2002}{Comp. Phys. Comm.}{147}{471}
{475}{Hydromagnetic turbulence in computer simulations}
\astrophB{0111569}\adsT{2002CoPhC.147..471B}\ownE{2002/Bran+Dobl02}
% 1 citations (Jun 2004)

\item[110.]
Fogedby, H. C., \& \Brandenburg\ypreN{2002}{66}{016604}
{Solitons in the noisy Burgers equation}
\condmatB{0105100}\adsT{2002PhRvE..66a6604F}\ownE{2002/Fogedby+Bran02}
% 2 citations (Feb 2004)
% 2 citations (Jun 2004)

\item[109.]
\Brandenburg, Dobler, W., \& Subramanian, K.\yan{2002}{323}{99}
{122}{Magnetic helicity in stellar dynamos: new numerical experiments}
\astrophB{0111567}\adsT{2002AN....323...99B}\ownE{2002/Bran+Dobl+Sub02}
% 6 citations (Mar 2003)
% 13 citations (Feb 2004)
% 16+1 citations (Jun 2004)

\item[108.]
Dobler, W., Shukurov, A., \& \Brandenburg\ypreN{2002}{65}{036311}
{Nonlinear states of the screw dynamo}
\astrophB{0105484}\adsT{2002PhRvE..65c6311D}\ownE{2002/Dobler+Shuk+Bran02}
% 1 citations (Feb 2003)
% 4 citations (Jun 2004)

\item[107.]
\Brandenburg, \& Sarson, G. R.\yprl{2002}{88}{055003, 1}
{4}{The effect of hyperdiffusivity on turbulent dynamos with helicity}
\astrophB{0110171}\adsT{2002PhRvL..88e5003B}\ownE{2002/Bran+Sars02}
% 2 citations (Feb 2003)
% 6 citations (Feb 2004)
% 6 citations (Jun 2004)

\item[106.]
Arlt, R., \& \Brandenburg\yana{2001}{380}{359}
{372}{Search for non-helical disc dynamos in simulations}
\astrophB{0106557}\adsT{2001A\%26A...380..359A}\ownE{2001/Arlt+Bran01}
% 3 citations (Feb 2003)
% 5 citations (Jun 2004)

\item[105.]
\Brandenburg, \& von Rekowski, B.\yana{2001}{379}{1153}
{1160}{Astrophysical significance of the anisotropic kinetic alpha effect}
\astrophB{0106280}\adsT{2001A\%26A...379.1153B}\ownE{2001/Bran+vReko01}
% 1 citations (Feb 2003)
% 2 citations (Feb 2004)
% 3 citations (Jun 2004)

\item[104.]
Christensson, M., Hindmarsh, M., \& \Brandenburg\ypreN{2001}{64}{056405}
{Inverse cascade in decaying 3D magnetohydrodynamic turbulence}
\astrophB{0011321}\adsT{2001PhRvE..64e6405C}\ownE{2001/Christensson_etal01}
% 6 citations (Feb 2004)
% 10 citations (Jun 2004)

\item[103.]
Ossendrijver, M., Stix, M., \& \Brandenburg\yana{2001}{376}{713}
{726}{Magnetoconvection and dynamo coefficients: dependence of the
alpha effect on rotation and magnetic field}
\astrophB{0108274}\adsT{2001A\%26A...376..713O}\ownE{2001/Ossen_etal01}
% 11 citations (Jun 2004)

\item[102.]
\Brandenburg, Bigazzi, A., \& Subramanian, K.\ymn{2001}{325}{685}
{692}{The helicity constraint in turbulent dynamos with shear}
\astrophB{0011081}\adsT{2001MNRAS.325..685B}\ownE{2001/BBS01}
% 7 citations (Feb 2004)
% 7+1 citations (Jun 2004)

\item[101.]
\Brandenburg\yjour{2001}{Science}{292}{2440}
{2441}{Magnetic mysteries}
\adsB{2001Sci...292.2440B}\ownT{2001/Bran01_sci}\htmlE{2440}
%(\url{http://www.sciencemag.org/cgi/content/summary/292/5526/2440},
% 0 citations (Jun 2004)
% 1 citations (May 2005)

\item[100.]
\Brandenburg, \& Hazlehurst, J.\yana{2001}{370}{1092}
{1102}{Evolution of highly buoyant thermals in a stratified layer}
\astrophB{0008099}\adsT{2001A\%26A...370.1092B}\ownE{2001/Bran+Hazle01}
% 1 citations (Feb 2003)
% 1 citations (Feb 2004)
% 2 citations (Jun 2004)

\item[99.]
Bardou, A., von Rekowski, B., Dobler, W., \Brandenburg, \&
Shukurov, A.\yana{2001}{370}{635}
{648}{The effects of vertical outflow on disk dynamos}
\astrophB{0011545}\adsT{2001A\%26A...370..635B}\ownE{2001/Bardou_etal01}
% 1 citations (Jun 2004)

\item[\important \relevant 98.]
\Brandenburg\yapj{2001}{550}{824}
{840}{The inverse cascade and nonlinear alpha effect in simulations
of isotropic helical hydromagnetic turbulence}
\astrophB{0006186}\adsT{2001ApJ...550..824B}\ownE{2001/B01}
% Errata: normalization: 1/sqrt(2), not 2; plus sign in Eq(21).
% 5 citations (Oct 2001)
% 9 citations (Jan 2002)
% 19 citations (Jul 2002)
% 30 citations (Oct 2002)
% 33 (20) citations (Dec 2002)
% 36 citations (Feb 2003) 14 self-citations
% 39 citations (Mar 2003)
% 41 citations (May 2003)
% 43 citations (Aug 2003)
% 46 citations (Sep 2003)
% 60 citations (Feb 2004)
% 67+10 citations (Jun 2004)
% 87+ citations (May 2005)

\item[97.]
\Brandenburg, \& Dobler, W.\yana{2001}{369}{329}
{338}{Large scale dynamos with helicity loss through boundaries}
\astrophB{0012472}\adsT{2001A\%26A...369..329B}\ownE{2001/Bran+Dobl01}
% 15 citations (Feb 2004)
% 15 citations (Jun 2004)

\item[96.]
S\'{a}nchez-Salcedo, F. J., \& \Brandenburg\ymn{2001}{322}{67}
{78}{Dynamical friction of bodies orbiting in a gaseous sphere}
\astrophB{0010003}\adsT{2001MNRAS.322...67S}\ownE{2001/Salc+Bran01}
% 5 citations (Feb 2004)
% 6 citations (Jun 2004)
% 8 citations (May 2005)

\item[95.]
\Brandenburg, \& Subramanian, K.\yana{2000}{361}{L33}
{L36}{Large scale dynamos with ambipolar diffusion nonlinearity}
\astrophB{0007450}\adsT{2000A\%26A...361L..33B}\ownE{2000/Bran+Sub00}
% 16 citations (Feb 2004)
% 16 citations (Jun 2004)
% 20 citations (May 2005)

\item[94.]
Torkelsson, U., Ogilvie, G. I., \Brandenburg, Pringle, J. E.,
Nordlund, \AA., \& Stein, R. F.\ymn{2000}{318}{47}
{57}{The response of a turbulent accretion disc to an imposed epicyclic
shearing motion}
\astrophB{0005199}\adsT{2000MNRAS.318...47T}\ownE{2000/Torkel_etal00}
% 5 citations (Jun 2004)

\item[93.]
Urpin, V., \& \Brandenburg\ymn{2000}{316}{684}
{688}{Non-linear magnetic diffusivity in mean-field electrodynamics}
\adsB{2000MNRAS.316..684U}\ownE{2000/Urp+Bran00}
% 2 citations (Jun 2004)

\item[92.]
\Brandenburg\yjour{2000}{Phil.\ Trans.\ Roy.\ Soc.\ Lond.\ A}{358}{759}
{776}{Dynamo-generated turbulence and outflows from accretion discs}
\adsB{2000RSPTA.358..759B}\ownE{2000/Bran_philtrans00}
% 5 citations (Jun 2004)

\item[91.]
Miesch, M. S., \Brandenburg, \& Zweibel, E. G.\ypr{2000}{E61}{457}
{467}{Nonlocal transport of passive scalars in turbulent penetrative convection}
\adsB{2000PhRvE..61..457M}\ownE{2000/Miesch+Bran+Zwei00}
% 5 citations (Jun 2004)

\item[90.]
Saar, S. H., \& \Brandenburg\yapj{1999}{524}{295}
{310}{Time evolution of the magnetic activity cycle period. II.
Results for an expanded stellar sample}
\adsB{1999ApJ...524..295S}\ownE{1999/Saar+Bran99}
% 31 citations (Jun 2004)

\item[89.]
Korpi, M. J., \Brandenburg, Shukurov, A., \& Tuominen, I.\yana{1999}{350}{230}
{239}{Evolution of a superbubble in a turbulent, multi-phased and
magnetized ISM}
\adsB{1999A\%26A...350..230K}\ownE{1999/Korpi_etal99b}
% 14 citations (Jun 2004)

\item[88.]
S\'{a}nchez-Salcedo, F. J., \& \Brandenburg\yapjl{1999}{522}{L35}
{L38}{Deceleration by dynamical friction in a gaseous medium}
\adsB{1999ApJ...522L..35S}\ownE{1999/Salc+Bran99}
% 3 citations (Jun 2004)

\item[87.]
R\"udiger, G., \Brandenburg, \& Pipin, V. V.\yan{1999}{320}{135}
{140}{A helicity proxy from horizontal solar flow patterns}
\adsB{1999AN....320..135R}\ownE{1999/Rue+Bran+Pip99}
% 0 citations (Jun 2004)

\item[86.]
Kerr, R. M., \& \Brandenburg\yprl{1999}{83}{1155}
{1158}{Evidence for a singularity in ideal magnetohydrodynamics:
implications for fast reconnection}
\physicsB{9812017}\adsT{1999PhRvL..83.1155K}\ownE{1999/Kerr+Bran99}
%~physics/9812017
% 2 citations (Sept 2000)
% 11 citations (Jun 2004)

\item[85.]
Moss, D., \& \Brandenburg\yana{1999}{346}{1009}
{1010}{Comment on `The sunspot as a self-excited dynamo'}
\adsB{1999A\%26A...346.1009M}\ownE{1999/Moss+Bran99}
% 2 citations (Jun 2004)

\item[84.]
Urpin, V., \& \Brandenburg\yana{1999}{345}{1054}
{1058}{Magnetic drift processes in differentially rotating turbulence}
\adsB{1999A\%26A...345.1054U}\ownE{1999/Urp+Bran99}
% 2 citations (Jun 2004)

\item[83.]
Covas, E., Tavakol, R., Tworkowski, A., \Brandenburg,
Brooke, J., \& Moss, D.\yana{1999}{345}{669}
{679}{The influence of geometry and topology on axisymmetric mean field
dynamos}
\astrophB{9811079}\adsT{1999A\%26A...345..669C}\ownE{1999/Covas_etal99}
% 4 citations (Jun 2004)

\item[\important 82.]
Korpi, M. J., \Brandenburg, Shukurov, A., Tuominen, I.,
\& Nordlund, \AA.\yapjl{1999}{514}{L99}
{L102}{A supernova regulated interstellar medium: simulations of the
turbulent multiphase medium}
\adsB{1999ApJ...514L..99K}\ownE{1999/Korpi_etal99}
%  4 citations (Oct 2000)
% 26 citations (Feb 2004)
% 29 citations (Jun 2004)

\item[81.]
Bigazzi, A., \Brandenburg, \& Moss, D.\ypp{1999}{6}{72}
{80}{Vortex tube models for turbulent dynamo action}
\adsB{1999PhPl....6...72B}\ownE{1999/Big+Bran+Moss99}
% 0 citations (Jun 2004)

\item[80.]
S\'{a}nchez-Salcedo, F. J., \& \Brandenburg,
Shukurov, A.\yjour{1998}{Astron.\ Spac.\ Sci.}{263}{87}
{90}{Turbulence and magnetic fields in clusters of galaxies}
\adsB{1998Ap\%26SS.263...87S}\ownE{1998/Sanch+Bran+Shuk98}
% 1 citations (Jun 2004)

\item[79.]
Bigazzi, A., \Brandenburg, \& Moss, D.\yjour{1998}{J.\ Phys.}{IV 8}{183}
{187}{Local models of small-scale dynamo action}
\doiB{10.1051/jp4:1998624}\ownE{1998/Biga+Bran+Moss98}
%J.de Phys. IV, vol.8 Pr6, 1998 pagg 183-187, edited by
%A.Vulpiani, M.ServA, G.Parisi, L.Peliti and  L.Pietronero.
% 0 citations (Jun 2004)

\item[78.]
\Brandenburg, \& Schmitt, D.\yana{1998}{338}{L55}
{L58}{Simulations of an alpha effect due to magnetic buoyancy}
\adsB{1998A\%26A...338L..55B}\ownE{1998/Bran+Schmitt98}
% 6 citations (Oct 2000)
% 24+1 citations (Jun 2004)

\item[77.]
Korpi, M. J., \Brandenburg, \& Tuominen, I.\yjour{1998}{Studia Geophys.\ et Geod.\ }{42}{410}
{418}{Driving interstellar turbulence by supernova explosions}
\ownS{1998/Korpi+Bran+Tuom98}
% 3 citations (Jun 2005)

\item[76.]
Tworkowski, A., Covas, E., Tavakol, R., \& \Brandenburg\yjour{1998}
{Studia Geophys.\ et Geod.\ }{42}{350}
{355}{Mean field dynamos with algebraic and dynamic alpha-quenchings}
\astrophB{9808214}\ownE{1998/Tworkowski_etal98}
% 0 citations (Jun 2004)

\item[75.]
\Brandenburg, \& Campbell, C. G.\ymn{1998}{298}{223}
{230}{The radial disc structure around a magnetic neutron star:
analytic and semi-analytic solutions}
\adsB{1998MNRAS.298..223B}\ownE{1998/Bran+Camp98}
% 7 citations (Jun 2004)

\item[74.]
\Brandenburg, Moss, D., \& Soward, A. M.\yjour{1998}
{Proc.\ Roy.\ Soc.\ A}{454}{1283}
{1300}{New results for the Herzenberg dynamo: steady and oscillatory solutions}
\adsB{1998RSPSA.454.1283B}\doiT{10.1098/rspa.1998.0207}\ownE{1998/Bran+Moss+Sow98}
%{http://www.journals.royalsoc.ac.uk/app/home/contribution.asp?wasp=6p8qd4b6lm6kvw40gndm&referrer=parent&backto=issue,3,14;journal,74,91;linkingpublicationresults,id:102023,1}
% 2 citations (Oct 2000)
% 3 citations (Jun 2004)

\item[73.]
\Brandenburg, Saar, S. H., \& Turpin, C. R.\yapjl{1998}{498}{L51}
{L54}{Time evolution of the magnetic activity cycle period}
\adsB{1998ApJ...498L..51B}\ownE{1998/Bran+Saar+Turp98}
% 9 citations (Oct 2000)
% 28 citations (Jun 2004)

\item[72.]
\Brandenburg, \& Urpin, V.\yana{1998}{332}{L41}
{L44}{Magnetic fields in young galaxies due to the cross-helicity effect}
\adsB{1998A\%26A...332L..41B}\ownE{1998/Bran+Urpin98}
% 3 citations (Oct 2000)
% 13 citations (Jun 2004)

\item[71.]
Hodgson, L. S., \& \Brandenburg\yana{1998}{330}{1169}
{1174}{Turbulence effects in planetesimal formation}
\adsB{1998A\%26A...330.1169H}\ownE{1998/Hodg+Bran98}
% 4 citations (Oct 2000)
% 10 citations (Feb 2004)
% 11 citations (Jun 2004)

\item[70.]
Covas, E., Tavakol, R., Tworkowski, A., \& \Brandenburg\yana{1998}{329}{350}
{360}{Axisymmetric mean field dynamos with dynamic and algebraic
alpha--quenchings}
\astrophB{9709062}\adsT{1998A\%26A...329..350C}\ownE{1998/Covas_etal98}
% 14 citations (Jun 2004)

\item[69.]
Tworkowski, A., Tavakol, R., \Brandenburg, Brooke, J. M., Moss, D.,
\& Tuominen, I.\ymn{1998}{296}{287}
{295}{Intermittent behaviour in axisymmetric mean field dynamo models}
\adsB{1998MNRAS.296..287T}\ownE{1998/Tworkowski_etal98b}
% 16 citations (Jun 2004)

\item[68.]
Urpin, V., \& \Brandenburg\ymn{1998}{294}{399}
{406}{Magnetic and vertical shear instabilities in accretion discs}
\adsB{1998MNRAS.294..399U}\ownE{1998/Urpin+Bran98}
% 1 citations (Oct 2000)
% 5 citations (Jun 2004)

\item[67.]
\Brandenburg\yjour{1997}{Acta Astron.\ Geophys.\ Univ.\ Comenianae}{XIX}{235}
{261}{Large scale turbulent dynamos}

\item[66.]
Covas, E., Tworkowski, A., Tavakol, R., \& \Brandenburg\ysph{1997}{172}{3}
{9}{Robustness of truncated alpha--Omega dynamos with a dynamic alpha}
\astrophB{9708094}\ownE{1997/Covas_etal97}
% 4 citations (Jun 2004)
% 4 citations (Jun 2005)

\item[65.]
Moss, D., \Brandenburg, \& Soward, A. M.\yjour{1997}
{Acta Astron.\ Geophys.\ Univ.\ Comenianae}{XIX}{43}
{50}{Steady and oscillatory solutions for the Herzenberg dynamo}

\item[64.]
\Brandenburg, \& Donner, K. J.\ymn{1997}{288}{L29}
{L33}{The dependence of the dynamo alpha on vorticity}
\adsB{1997MNRAS.288L..29B}\ownE{1997/Bran+Donn97}
% 11 citations (Oct 2000)
% 16 citations (May 2001)
% 20(26) citations (Dec 2001)
% 33 citations (Jun 2004)

\item[63.]
Zweibel, E. G., \& \Brandenburg\yapj{1997}{478}{563}
{568}{Current sheet formation in the interstellar medium}
\adsB{1997ApJ...478..563Z}\ownE{1997/Zwei+Bran97}
%Err: 485, 920
% 9 citations (Oct 2000)
% 10 citations (May 2001)
% 19 citations (Jun 2004)

\item[62.]
\Brandenburg, Enqvist, K., \& Olesen, P.\yjour{1997}
{Phys.\ Lett.\ B.}{392}{395}
{402}{The effect of Silk damping on primordial magnetic fields}
\hepphB{9608422}\adsT{1997PhLB..392..395B}\ownE{1997/Bran+Enq+Oles97}
% 6 citations (Oct 2000)
% 10 citations (Jun 2004)

\item[61.]
Covas, E., Tworkowski, A., \Brandenburg, \& Tavakol, R.\yana{1997}{317}{610}
{617}{Dynamos with different formulations of a dynamic alpha effect}
\astrophB{9708093}\adsT{1997A\%26A...317..610C}\ownE{1997/Covas_etal97b}
% 15 citations (Jun 2004)

\item[60.]
Vishniac, E. T., \& \Brandenburg\yapj{1997}{475}{263}
{274}{An incoherent alpha--Omega dynamo in accretion disks}
\astrophB{9510038}\adsT{1997ApJ...475..263V}\ownE{1997/Vish+Bran97}
% 9 citations (Oct 2000)
% 11 citations (May 2001)
% 11 citations (Oct 2001) 
% 12 citations (Apr 2002) 
% 18 citations (Jun 2004)

\item[59.]
Nordlund, \AA., Stein, R. F., \& \Brandenburg\yjour{1996}
{Bull.\ Astr.\ Soc.\ India}{24}{261}
{279}{Supercomputer windows into the solar convection zone}
\adsB{1996BASI...24..261N}\ownE{1996/Nord+Stein+Bran96}

\item[\important 58.]
Beck, R., \Brandenburg, Moss, D., Shukurov, A., \&
Sokoloff, D.\yaraa{1996}{34}{155}
{206}{Galactic magnetism: recent developments and perspectives}
\adsB{1996ARA\%26A..34..155B}\ownE{1996/Beck_etal96}
% 32 citations (March 1998)
% 83 citations (Oct 2000)
% 95 citations (May 2001)
% 105 citations (Oct 2001) 
% 149 citations (Dec 2002) 
% 159 citations (May 2003)
% 202 citations (Jun 2004)

\item[57.]
Torkelsson, U., \Brandenburg, Nordlund, \AA., \& Stein, R. F.\yjour{1996}
{Astrophys.\ Letter \& Comm.}{34}{383}
{388}{The turbulent viscosity in accretion discs}
\adsB{1996ApL\%26C..34..383T}\ownE{1996/BNST_ulf96}
% 5 citations (Oct 2001) 
% 5 citations (Jun 2004)

\item[56.]
Abramowicz, M. A., \Brandenburg, \& Lasota, J.-P.\ymn{1996}{281}{L21}
{L24}{The dependence of the viscosity in accretion discs on the
shear/vorticity ratio}
\adsB{1996MNRAS.281L..21A}\ownE{1996/Abramo+Bran+Lasota96}
% 1 citation (March 1998)
% 10 citations (May 2001)
% 14 citations (Jun 2004)

\item[55.]
\Brandenburg\yapjl{1996}{465}{L115}
{L118}{Testing Cowling's anti-dynamo theorem near a rotating black hole}
\adsB{1996ApJ...465L.115B}\ownE{1996/Bran96}
% 6 citations (March 1998)
% 10 citations (Oct 2000)
% 11 citations (Jul 2001)
% 11 citations (Feb 2004)
% 11 citations (Jun 2004)

\item[\important 54.]
\Brandenburg, Enqvist, K., \& Olesen, P.\yprd{1996}{54}{1291}
{1300}{Large-scale magnetic fields from hydromagnetic turbulence in the very early universe}
\astrophB{9602031}\adsT{1996PhRvD..54.1291B}\ownE{1996/Bran+Enq+Oles96}
% 17 citations (March 1998)
% 21 citations (Sept 1998)
% 27 citations (Oct 2000)
% 29 citations (Mar 2001)
% 31 citations (Jul 2001)
% 38 citations (Oct 2002)
% 39 citations (Dec 2002)
% 50 citations (Jun 2004)

\item[53.]
\Brandenburg, Nordlund, \AA., Stein, R. F., \&
Torkelsson, U.\yapjl{1996}{458}{L45}
{L48}{The disk accretion rate for dynamo generated turbulence}
\adsB{1996ApJ...458L..45B}\ownE{1996/BNST96}
% 9 citations (March 1998)
% 15 citations (Sept 1998)
% 29 citations (Oct 2001) 
% 38 citations (Jun 2004)

\item[\important 52.]
\Brandenburg, Jennings, R. L., Nordlund, \AA.,
Rieutord, M., Stein, R. F., \& Tuominen, I.\yjfm{1996}{306}{325}
{352}{Magnetic structures in a dynamo simulation}
\adsB{1996JFM...306..325B}\ownE{1996/BJNRRT96}
% 32 citations (Sept 2000)
% 38 citations (Jul 2001)
% 40 citations (Oct 2001) 
% 69 citations (Feb 2004)
% 72 citations (Jun 2004)

\item[51.]
\Brandenburg, Klapper, I., \& Kurths, J.\ypre{1995}{52}{R4602}
{R4605}{Generalized entropies in a turbulent dynamo simulation}
\adsB{1995PhRvE..52.4602B}\ownE{1995/Bran+Klapp+Kur95}
% 2 citations (March 1998)
% 7 citations (Sept 2000)
% 7 citations (Oct 2001) 
% 8 citations (Feb 2004)
% 9 citations (Jun 2004)

\item[50.]
Moss, D., \& \Brandenburg\ygafd{1995}{80}{229}
{240}{The generation of nonaxisymmetric magnetic fields in the giant planets}
\adsB{1995GApFD..80..229M}\ownE{1995/Moss+Bran95}
% 4 citations (Oct 2000)
% 12 citations (Jun 2004)

\item[49.]
\Brandenburg, Moss, D., \& Shukurov, A.\ymn{1995}{276}{651}
{662}{Galactic fountains as magnetic pumps}
\adsB{1995MNRAS.276..651B}\ownE{1995/Bran+Moss+Shuk95}
% 4 citations (March 1998)
% 6 citations (Oct 2000)
% 9 citations (Oct 2001) 
% 10 citations (Jun 2004) 

\item[48.]
\Brandenburg\ycsf{1995}{5}{2023}
{2045}{Flux tubes and scaling in MHD dynamo simulations}
\adsB{1995CSF.....5.2023B}\ownE{1995/Bran95}
% 0 citations (Oct 2001) 
% 0 citations (Jun 2004)

\item[47.]
Kerr, R. M., Herring, J. R., \& \Brandenburg\ycsf{1995}{5}{2047}
{2053}{Large-scale structure in Rayleigh-B\'enard convection
with impenetrable side-walls}
\adsB{1995CSF.....5.2047K}\ownE{1995/Kerr+Herr+Bran95}
% 4 citations (Oct 2001) 
% 5 citations (Jun 2004)

\item[46.]
Torkelsson, U., \& \Brandenburg\ycsfS{1995}{5}{1975}
{1984}{Chaos in accretion disk dynamos?}
\adsB{1995CSF.....5.1975T}\ownE{1995/Tork+Bran95}
% 3 citations (Oct 2001) 
% 4 citations (Jun 2004)

\item[45.]
\Brandenburg, \& Zweibel, E. G.\yapj{1995}{448}{734}
{741}{Effects of pressure and resistivity on the
ambipolar diffusion singularity: too little, too late}
\adsB{1995ApJ...448..734B}\ownE{1995/Bran+Zwei95}
% 15 citations (Sept 1998)
% 27 citations (Oct 2001) 
% 33 citations (Feb 2004)
% 35 citations (Jun 2004) 

\item[\important 44.]
\Brandenburg, Nordlund, \AA., Stein, R. F., \& Torkelsson, U.\yapj{1995}{446}{741}
{754}{Dynamo-generated turbulence and large scale magnetic fields
in a Keplerian shear flow}
\adsB{1995ApJ...446..741B}\ownE{1995/BNST95}
% 54 citations  (Mar 1998, 28 in 1996)
% 69 citations  (Sep 1998, 20 in 1997)
% 106 citations (Feb 2000, 30 in 1998)
% 119 citations (Sep 2000, 23 in 1999) 
% 128 citations (Jan 2001, 25 in 2000) 
% 159 citations (Oct 2001, 38 in 2001) 
% 165 (167) citations (Dec 2001) 
% 182 citations (Jul 2002)
% 189 citations (Oct 2002)
% 193 citations (Dec 2002)
% 197 citations (Mar 2003)
% 201 citations (May 2003)
% 204 citations (Aug 2003)
% 214 citations (Feb 2004)
% 224 citations (Jun 2004) 

\item[43.]
\Brandenburg, Procaccia, I., \& Segel, D.\ypp{1995}{2}{1148}
{1156}{The size and dynamics of magnetic flux structures in MHD turbulence}
\adsB{1995PhPl....2.1148B}\ownE{1995/Bran+Proc+Seg95}
% 4 citations (March 1998)
% 8 citations (Oct 2001) 
% 10 citations (Feb 2004)
% 11 citations (Jun 2004) 

\item[42.]
Muhli, P., \Brandenburg, Moss, D., \& Tuominen, I.\yana{1995}{296}{700}
{704}{Multiple far-supercritical solutions for an alpha--Lambda  dynamo}
\adsB{1995A\%26A...296..700M}\ownE{1995/Muhli_etal95}
% 4 citations (Oct 2001) 
% 5 citations (Jun 2004)

\item[41.]
R\"udiger, G., \& \Brandenburg\yana{1995}{296}{557}
{566}{A solar dynamo in the overshoot layer:
cycle period and butterfly diagram}
\adsB{1995A\%26A...296..557R}\ownE{1995/Rue+Bran95}
% 38 citations (Oct 2001) 
% 43 citations (Oct 2002) 
% 44 citations (Mar 2003) 
% 53 citations (Feb 2004)
% 54 citations (Jun 2004)

\item[40.]
Tavakol, R. K., Tworkowski, A. S., \Brandenburg, Moss, D., \&
Tuominen, I.\yana{1995}{296}{269}
{274}{Structural stability of axisymmetric dynamo models}
\adsB{1995A\%26A...296..269T}\ownE{1995/Tavakol_etal95}
% 17 citations (Oct 2001) 
% 19 citations (Jun 2004) 

\item[39.]
Moss, D., Barker, D. M., \Brandenburg, \& Tuominen, I.\yana{1995}{294}{155}
{164}{Nonaxisymmetric dynamo solutions and extended starspots on late type stars}
\adsB{1995A\%26A...294..155M}\ownE{1995/Barker_etal95}
% 21 citations (Oct 2001) 
% 32 citations (Jun 2004) 

\item[38.]
Torkelsson, U., \& \Brandenburg\yana{1994}{292}{341}
{349}{The many incarnations of accretion disk dynamos:
mixed parities and chaos for large dynamo numbers}
\adsB{1994A\%26A...292..341T}\ownE{1994/Torkel+Bran94b}
% 4 citations (March 1998)
% 8 citations (Sept 1999)
% 17 citations (Oct 2001) 
% 20 citations (Jun 2004) 

\item[37.]
Rieutord, M., \Brandenburg, Mangeney, A., \& Drossart, P.\yana{1994}{286}{471}
{480}{Reynolds stress and differential rotation
in Boussinesq convection in a rotating spherical shell}
\adsB{1994A\%26A...286..471R}\ownE{1994/Rieutord_etal94}
% 3 citations (Oct 2001) 
% 3 citations (Feb 2004)
% 3 citations (Jun 2004) 

\item[36.]
\Brandenburg, \& Zweibel, E. G.\yapjl{1994}{427}{L91}
{L94}{The formation of sharp structures by ambipolar diffusion}
\adsB{1994ApJ...427L..91B}\ownE{1994/Bran+Zwei94_col}
% 12 citations (March 1998)
% 22 citations (Sept 2000)
% 28 citations (Oct 2001) 
% 31 citations (Oct 2002) 
% 32 citations (Feb 2004)
% 33 citations (Jun 2004) 

\item[35.]
Tuominen, I., \Brandenburg, Moss, D., \& Rieutord, M.\yanas{1994}{284}{259}
{264}{Does solar differential rotation arise from a large scale instability?}
\adsB{1994A\%26A...284..259T}\ownE{1994/Tuom_etal94}
% 4 citations (Oct 2001) 
% 6 citations (Feb 2004)
% 6 citations (Jun 2004) 

\item[34.]
Torkelsson, U., \& \Brandenburg\yana{1994}{283}{677}
{691}{Turbulent accretion disk dynamos}
\adsB{1994A\%26A...283..677T}\ownE{1994/Torkel+Bran94a}
% 12 citations (March 1998)
% 13 citations (Sept 1999)
% 22 citations (Oct 2001) 
% 23 citations (Jun 2004) 

\item[33.]
Feudel, F., Feudel, U., \& \Brandenburg\ybif{1993}{3}{1299}
{1303}{On the bifurcation phenomena of the Kuramoto-Sivashinsky Equation}
\adsB{1993IJBC....3.1299F}\doiT{10.1142/S0218127493001045}\ownE{1993/Feudels+Bran93}

\item[32.]
Moss, D., \Brandenburg, Donner, K. J., \& Thomasson, M.\yapj{1993}{409}{179}
{189}{Models for the magnetic field of M81}
\adsB{1993ApJ...409..179M}\ownE{1993/Moss_etal93}
% No. 57.157.473
% Normal Galaxies
% 19 citations (Oct 2001) 
% 20 citations (Jun 2004) 

\item[31.]
\Brandenburg, Donner, K. J., Moss, D., Shukurov, A., Sokoloff, D. D.,
\& Tuominen, I.\yana{1993}{271}{36}
{50}{Vertical magnetic fields above the discs of spiral galaxies}
\adsB{1993A\%26A...271...36B}\ownE{1993/BDMSST93}
% No. 57.157.233
% Normal Galaxies
% 33 citations (Oct 2001) 
% 37 citations (Jun 2004) 

\item[30.]
Pulkkinen, P., Tuominen, I., \Brandenburg, Nordlund,
\AA., \& Stein, R. F.\yana{1993}{267}{265}
{275}{Rotational effects on convection simulated at different latitudes}
\adsB{1993A\%26A...267..265P}\ownE{1993/Pulkk_etal93}
% No. 57.080.018
% Atmosphere, Interior, Neutrinos
% 16 citations (Oct 2001) 
% 19 citations (Jun 2004) 

\item[29.]
Jennings, R. L., \Brandenburg, Nordlund, \AA., \& Stein, R.F.\ymn{1992}{259}{465}
{473}{Evolution of a magnetic flux tube in two dimensional penetrative
convection}
\adsB{1992MNRAS.259..465J}\ownE{1992/Jennings_etal92}
% No. 56.062.176
% Magnetohydrodynamics, Plasma
% 6 citations (Oct 2001) 
% 7 citations (Jun 2004) 

\item[28.]
Moss, D., \Brandenburg, Tavakol, R. K., \& Tuominen, I.\yana{1992}{265}{843}
{849}{Stochastic effects in mean field dynamos}
\adsB{1992A\%26A...265..843M}\ownE{1992/Moss_etal92}
% No. 56.062.051
% Magnetohydrodynamics, Plasma
% 11 citations (Oct 2001) 
% 15 citations (Jun 2004) 

\item[27.]
\Brandenburg, Procaccia, I., Segel, D., \& Vincent, A.\ypra{1992}{46}{4819}
{4828}{Fractal level sets and multifractal fields
in direct simulations of turbulence}
\adsB{1992PhRvA..46.4819B}\ownE{1992/Bran+Proc_etal92}
% 7 citations (Oct 2001) 
% 9 citations (Jun 2004) 

\item[26.]
\Brandenburg, Moss, D., \& Tuominen, I.\yana{1992}{265}{328}
{344}{Stratification and thermodynamics in mean-field dynamos}
\adsB{1992A\%26A...265..328B}\ownE{1992/Bran+Moss+Tuom92}
% No. 56.062.045
% Magnetohydrodynamics, Plasma
% 16 citations (Oct 2001) 
% 18 citations (Jun 2004) 

\item[25.]
\Brandenburg\yprl{1992}{69}{605}
{608}{Energy spectra in a model for convective turbulence}
\adsB{1992PhRvL..69..605B}\ownE{1992/B92}
% 11 citations (August 1999)
% 13 citations (September 2000)
% 13 citations (Oct 2001) 
% 17 citations (Jun 2004) 

\item[24.]
\Brandenburg, Donner, K. J., Moss, D., Shukurov, A., Sokoloff, D. D., \& Tuominen, I.\yana{1992}{259}{453}
{461}{Dynamos in discs and halos of galaxies}
\adsB{1992A\%26A...259..453B}\ownE{1992/BDMSST92}
% No. 55.157.248
% Normal Galaxies
% 29 citations (Oct 2001) 
% 31 citations (Jun 2004) 

\item[23.]
Procaccia, I., \Brandenburg, Jensen, M. H., Vincent, A.\yepl{1992}{19}{183}
{187}{The fractal dimension of iso-vorticity structures
in 3-dimensional turbulence}
\adsB{1992EL.....19..183P}\ownE{1992/Procaccia_etal92}
% 9 citations (Oct 2001) 
% 9 citations (Jun 2004) 

\item[\important 22.]
Nordlund, \AA., \Brandenburg, Jennings, R. L., Rieutord, M.,
Ruokolainen, J., Stein, R. F., \& Tuominen, I.\yapj{1992}{392}{647}
{652}{Dynamo action in stratified convection with overshoot}
\adsB{1992ApJ...392..647N}\ownE{1992/Nord_etal92}
% No. 55.062.177
% Magnetohydrodynamics, Plasma
% 76 citations (Sept 2000)
% 84 citations (Oct 2001) 
% 92 citations (Oct 2002) 
% 106 citations (Jun 2004) 

\item[21.]
Moss, D., \& \Brandenburg\yana{1992}{256}{371}
{374}{The influence of boundary conditions on the excitation
of disk dynamo modes}
\adsB{1992A\%26A...256..371M}\ownE{1992/Moss+Bran92}
% No. 55.062.056
% Magnetohydrodynamics, Plasma
% 17 citations (Oct 2001) 
% 17 citations (Jun 2004) 

\item[20.]
Kurths, J., \& \Brandenburg\ypra{1991}{44}{R3427}
{R3429}{Lyapunov exponents for hydromagnetic convection}
\adsB{1991PhRvA..44.3427K}\ownE{1991/Kurths+Bran91}
% 7 citations (Oct 2001) 
% 9 citations (Jun 2004) 

\item[19.]
Moss, D., \Brandenburg, \& Tuominen, I.\yana{1991}{247}{576}{579}
{Properties of mean field dynamos with nonaxisymmetric alpha effect}
\adsB{1991A\%26A...247..576M}\ownE{1991/Moss+Bran+Tuom91}
% No. 54.062.044
% Magnetohydrodynamics, Plasma
% 8 citations (Oct 2001) 
% 13 citations (Jun 2004) 

\item[18.]
\Brandenburg, Moss, D., R\"udiger, G., \& Tuominen, I.\ygafd{1991}{61}{179}
{198}{Hydromagnetic alpha--Omega-type dynamos with feedback
from large scale motions}
\adsB{1991GApFD..61..179B}\ownE{1991/Bran+Moss+Rue+Tuom91}
% No. 54.062.250
% Magnetohydrodynamics, Plasma
% 18 citations (Oct 2001) 
% 19 citations (Jun 2004) 

\item[17.]
Moss, D., Tuominen, I., \& \Brandenburg\yana{1991}{245}{129}{135}
{Nonlinear nonaxisymmetric dynamo models for cool stars}
\adsB{1991A\%26A...245..129M}\ownE{1991/Moss+Tuom+Bran91}
% No. 53.065.072
% Stellar
% 19 citations (Oct 2001) 
% 27 citations (Jun 2004) 

\item[16.]
Donner, K. J., \& \Brandenburg\yana{1990}{240}{289}
{298}{Generation and interpretation of galactic magnetic fields}
\adsB{1990A\%26A...240..289D}\ownE{1990/Donn+Bran90}
% No. 52.062.077
% 37 citations (Oct 2001) 
% 40 citations (Oct 2002)
% 41 citations (Mar 2003)
% 41 citations (Feb 2004)
% 41 citations (Jun 2004) 

\item[15.]
R\"adler, K.-H., Wiedemann, E., 
\Brandenburg, Meinel, R., \& Tuominen, I.\yana{1990}{239}{413}
{423}{Nonlinear mean-field dynamo models:
Stability and evolution of three-dimensional magnetic field configurations}
\adsB{1990A\%26A...239..413R}\ownE{1990/Raedler_etal90}
% No. 52.062.067
% 24 citations (Oct 2001) 
% 32 citations (Jun 2004) 

\item[14.]
Meinel, R., \& \Brandenburg\yana{1990}{238}{369}
{376}{Behavior of highly supercritical alpha effect dynamos}
\adsB{1990A\%26A...238..369M}\ownE{1990/Mein+Bran90}
% No. 52.062.058
% 11 citations (Oct 2001) 
% 12 citations (Jun 2004) 

\item[13.]
Moss, D., Tuominen, I., \& \Brandenburg\yana{1990}{240}{142}{149}
{Buoyancy limited thin shell dynamos}
\adsB{1990A\%26A...240..142M}\ownE{1990/Moss+Tuom+Bran90b}
% No. 52.062.071
% 20 citations (Oct 2001) 
% 23 citations (Jun 2004) 

\item[12.]
Donner, K. J., \& \Brandenburg\ygafd{1990}{50}{121}{129}
{Magnetic field structure in differentially rotating discs}
\adsB{1990GApFD..50..121D}\doiT{10.1080/03091929008219877}\ownE{1990/Donn+Bran90b}
% No. 51.062.236
% 8 citations (Oct 2001) 
% 8 citations (Jun 2004) 

\item[11.]
\Brandenburg, Moss, D., R\"udiger, G., \& Tuominen, I.\ysphs{1990}{128}{243}{251}
{The nonlinear solar dynamo and differential rotation: A Taylor number puzzle?}
\adsB{1990SoPh..128..243B}\ownE{1990/Bran+Moss+Rue+Tuom90}
% No. 52.080.055
% 20 citations (Oct 2001) 
% 20 citations (Jun 2004) 

\item[10.]
\Brandenburg, Tuominen, I., \& Krause, F.\ygafd{1990}{50}{95}
{112}{Dynamos with a flat alpha effect distribution}
\adsB{1990GApFD..50...95B}\ownE{1990/Bran+Tuom+Krau90}
% No. 51.062.231
% 12 citations (Oct 2001) 
% 13 citations (Jun 2004) 

\item[9.]
\Brandenburg, Nordlund, \AA., Pulkkinen, P., Stein, R.F., \& Tuominen, I.\yana{1990}{232}{277}
{291}{3-D Simulation of turbulent cyclonic magneto-convection}
\adsB{1990A\%26A...232..277B}\ownE{1990/BNPST90}
% No. 51.062.143
% 52 citations (Oct 2001)
% 56 citations (Oct 2002)
% 59 citations (Jun 2004) 

\item[8.]
Jennings, R. L., \Brandenburg, Moss, D., \& Tuominen, I.\yana{1990}{230}{463}{473}
{Can stellar dynamos be modelled in less than three dimensions?}
\adsB{1990A\%26A...230..463J}\ownE{1990/Jennings_etal90}
% No. 51.062.100
% 16 citations (Oct 2001) 
% 18 citations (Jun 2004) 

\item[7.]
Moss, D., Tuominen, I., \& \Brandenburg\yana{1990}{228}{284}{294}
{Nonlinear dynamos with magnetic buoyancy in spherical geometry}
\adsB{1990A\%26A...228..284M}\ownE{1990/Moss+Tuom+Bran90a}
% No. 51.062.050
% 26 citations (Jun 2004) 

\item[6.]
Vilhu, O., Ambruster, C. W., Neff, J. E., Linsky, J. L., \Brandenburg,
Ilyin, I. V., \& Shakhovskaya, N. I.\yana{1989}{222}{179}{186}
{IUE observations of the M dwarfs CM Draconis and Rossiter 137 B:
magnetic activity at saturated levels}
\adsB{1989A\%26A...222..179V}\ownE{1989/Vilhu_etal89}
% No. 50.117.086
% Stellar
% 14 citations (Oct 2001) 
% 14 citations (Jun 2004) 

\item[5.]
\Brandenburg, Tuominen, I., \& R\"adler, K.-H.\ygafd{1989}{49}{45}{55}
{On the generation of non-axisymmetric magnetic fields in mean-field dynamos}
\adsB{1989GApFD..49...45B}\doiT{10.1080/03091928908243462}\ownE{1989/Bran+Tuom+Raed89}
% No. 50.062.171
% 5 citations (Jun 2004) 

\item[4.]
\Brandenburg, Tuominen, I., \& Moss, D.\ygafd{1989}{49}{129}{141}
{On the nonlinear stability of dynamo models}
\adsB{1989GApFD..49..129B}\ownE{1989/Bran+Tuom+Moss89}
% No. 50.062.175
% 26 citations (Jun 2004) 

\item[\important 3.]
\Brandenburg, Krause, F., Meinel, R., Moss, D., \& Tuominen,
I.\yana{1989} {213}{411}
{422}{The stability of nonlinear dynamos and the limited role of kinematic growth rates}
\adsB{1989A\%26A...213..411B}\ownE{1989/BKMMT89}
% No. 49.062.036
% 91 citations (Oct 2001) 
% 98 citations (Dec 2002) 
% 105 citations (Jun 2004) 
 
\item[2.]
\Brandenburg, \& Tuominen, I.\yjour{1988}{Adv. Spa. Sci.}{8}{185}
{189}{Variation of magnetic fields and flows during the solar cycle}
\adsB{1988AdSpR...8..185B}\ownE{1988/Bran+Tuom88}
% No. 49.075.032
% Solar
 
\item[1.]
\Brandenburg\yana{1988}{203}{154}
{161}{Hydrodynamic Green's functions for atmospheric oscillations}
\adsB{1988A\%26A...203..154B}\ownE{1988/B88}
% No. 46.080.029
% 1 citations (Jun 2004) 

%\item.]
%Bigazzi, A.\& \Brandenburg\sana{1998}
%{An alpha-effect from magnetic buoyancy with shear}

%\item
%\Brandenburg, Brooks S.J., Shukurov A., \&
%Sokoloff D.D.\tjfm{2001}
%{Simulations of acoustic MHD turbulence}

%%  Statistics S t a t  STAT
%%  nl6/tex/nor/development/statistics/axel.dat
% yr  all first single conf rev event
%1987   0   0    0      0   0   Hamburg
%1988   2   2    1      3   0   Helsinki
%1989   4   3    0      3   0
%1990  10   3    0      6   0   PhD
%1991   4   1    0      5   1
%1992   9   4    2      2   0   Boulder
%1993   4   1    0      6   1
%1994   5   1    0      6   2   Nordita
%1995  13   6    1      2   0
%1996   8   4    1      0   1   Newcastle
%1997   8   3    1      2   2
%1998  13   5    0      1   2
%1999  10   0    0      6   1
%2000   5   2    1      3   1   Nordita (Copenhagen)
%2001  11   6    2      5   3
%2002  10   5    0      2   0
%2003   9   1    0      5   4
%2004  18   4    0      1   2
%2005  14   9    2      1   3
%2006  11   2    1      4   0   Nordita (Stockholm)
%2007   7   5    0      1   5
%2008  14   6    2      0   1
%2009  17   5    3      0   2   ERC grant
%2010  19   3    1      2   0
%2011  28   7    3     12   2
%2012  26   4    0      6   1
%2013  20   4    0      2   1   VR breakthrough
%2014  20   3    1      1   0   FRINATEK
%2015   8   2    0      0   1   Boulder
%2016  16   3    2      0   0   Boulder
%2017  16   8    1      0   0   Boulder
%2018  19   7    2      0   0   Boulder
%2019  12   6    3      0   0
%2020  21   9    2      2   1   VR grant
%2021  11   6    0      0   1   more COVID
%2022  12   1    0      0   1   more dynamo theory
%2023  17  11    1      0   0
%2024   7   3    0      1   0
% yr  all first single conf rev event

\end{itemize}

%\noindent{\it B. Invited conference reviews}
\subsection*{B. Invited reviews} %B B B 
\begin{itemize}

%{Turbulence from Earth to Planets, Stars and Galaxies—Commemorative Issue Dedicated to the Memory of Jackson Rea Herring}
%https://www.mdpi.com/books/book/10271-turbulence-from-earth-to-planets-stars-and-galaxies-commemorative-issue-dedicated-to-the-memory-of

\item[43.]
\Brandenburg, Larsson, G.\yprocS{2024}{123}
{139}{Turbulence with magnetic helicity that is absent on average}
{Turbulence from Earth to Planets, Stars and Galaxies—Commemorative Issue Dedicated to the Memory of Jackson Rea Herring}
{B. Galperin, A. Pouquet, \& P. Sullivan}
{MDPI Books}
\doiS{10.3390/books978-3-7258-2740-4}

\item[42.]
\Brandenburg\yprocS{2022}{15}
{35}{Chirality in Astrophysics}
{Proceedings to Nobel Symposium 167: Chiral Matter}
{E. Babaev, D. Kharzeev, M. Larsson, A. Molochkov, \& V. Zhaunerchyk}
{World Scientific}
\arxivB{2110.08117}\adsT{2021arXiv211008117B}\doiT{doi.org/10.1142/13107}\httpT{http://norlx65.nordita.org/~brandenb/tmp/Lidingo}\ownE{B/2022/Bran22}

\item[41.]
\Brandenburg\yprocS{2021}{87}
{115}{Homochirality: a prerequisite or consequence of life?}
{Prebiotic Chemistry and the Origin of Life}
{A.\ Neubeck, \& S. McMahon}
{Springer}
\arxivB{2012.12850}\adsT{2021pcol.book...87B}\doiT{10.1007/978-3-030-81039-9_4}\httpT{http://norlx51.nordita.org/~brandenb/tmp/anna}\ownE{B/2021/Bran21}

\item[40.]
\Brandenburg\yproc{2020}{169}
{180}{Magnetic field evolution in solar-type stars}
{IAUS 354: Solar and Stellar Magnetic Fields: Origins and Manifestations}
{A. Kosovichev, K. Strassmeier \& M. Jardine}
{Proc.\ IAU Symp., Vol.\ {\bf 354}}
\arxivB{2004.00439}\adsT{2020IAUS..354..169B}\doiT{10.1017/S1743921320001404}\ownE{B/2020/Bran20}
%tex/proc/copiapo19

\item[39.]
\Brandenburg, Candelaresi, S., \& Gent, F. A.\ygafd{2020}{114}{1}
{7}{Introduction to {\sf The Physics and Algorithms of the Pencil Code}}
\doiB{10.1080/03091929.2019.1677015}\adsT{2020GApFD.114....1B}\ownE{B/2020/Bran+Cand+Gent20}

\item[38.]
Losada, I. R., Warnecke, J., Glogowski, K., Roth, M., \Brandenburg, Kleeorin, N., \& Rogachevskii, I.\yproc{2017}{46}
{59}{A new look at sunspot formation using theory and observations}
{IAUS 327: Fine Structure and Dynamics of the Solar Atmosphere}
{S. Vargas Dom{\'{\i}}nguez, A.~G.\ Kosovichev, P. Antolin, \& L. Harra}
{Proc.\ IAU Symp., Vol.\ {\bf 12}}
\arxivB{1704.04062}\adsT{2017IAUS..327...46L}\doiT{10.1017/S1743921317004306}\ownE{B/2017/Losa_etal17}

\item[37.]
\Brandenburg\yproc{2015}{529}
{555}{Simulations of galactic dynamos}
{Magnetic fields in diffuse media}
{E. de Gouveia Dal Pino \& A. Lazarian}
%{Lecture Notes in Physics, Springer}
{Astrophys. Spa. Sci. Lib., Vol. {\bf 407}, Springer}
\arxivB{1402.0212}\adsT{2015ASSL..407..529B}\doiT{10.1007/978-3-662-44625-6_19}\ownE{B/2015/Bran15}

\item[36.]
\Brandenburg\yproc{2013}{387}
{398}{Non-linear and chaotic dynamo regimes}
{Solar and astrophysical dynamos and magnetic activity}
{A.\ G. Kosovichev, E.\ M. de Gouveia Dal Pino \& Y.\ Yan}
{Proc.\ IAU Symp., Vol.\ {\bf 294}}
\arxivB{1305.1952}\adsT{2013IAUS..294..387B}\doiT{10.1017/S1743921313002822}\ownE{B/2013/Bran13}
%is on solar physics archive

\item[35.]
\Brandenburg, \& Guerrero, G.\yproc{2012}{37}
{48}{Cycles and cycle modulations}
{Comparative Magnetic Minima: Characterizing quiet times in the Sun and stars}
{C.\ H.\ Mandrini \& D.\ F.\ Webb}{Proc.\ IAU Symp., Vol.\ {\bf 286}}
\arxivB{1111.3351}\adsT{2012IAUS..286...37B}\doiT{10.1017/S1743921312004619}\ownE{B/2012/Bran+Guer12}

\item[34.]
\Brandenburg\yproc{2011}{402}
{409}{Simulations of astrophysical dynamos}
{Advances in Plasma Astrophysics}
{A.\ Bonanno, E.\ de Gouveia dal Pino, \& A.\ Kosovichev}{Proc.\ IAU Symp., Vol.\ {\bf 274}}
\arxivB{1012.5079}\adsT{2011IAUS..274..402B}\doiT{10.1017/S174392131100737X}\ownE{B/2011/Bran11_naxos}
%http://journals.cambridge.org/action/displayIssue?decade=2010&jid=IAU&volumeId=6&issueId=S274&iid=8295041

\item[33.]
\Brandenburg, K\"apyl\"a, P. J., \& Korpi, M. J.\yproc{2011}{279}
{287}{From convective to stellar dynamos}
{Astrophysical Dynamics: From Stars to Galaxies}
{N.\ Brummell, A.\ S.\ Brun, M.\ S.\ Miesch, \& Y.\ Ponty}{Proc.\ IAU Symp., Vol.\ {\bf 271}}
\arxivB{1103.5475}\adsT{2011IAUS..271..279B}\doiT{10.1017/S1743921311017704}\ownE{B/2011/Bran+Kapy+Korp11_nice}

\item[32.]
\Brandenburg\yproc{2009}{433}
{442}{From fibril to diffuse fields during dynamo saturation}
{Solar-stellar dynamos as revealed by helio- and asteroseismology}
%proceedings of the GONG 2008/SOHO XXI workshop}
{M.\ Dikpati, Arentoft, T., Hern\'andez, I. G., Lindsey, C., \& Hill, F.}
{Astron. Soc. Pac. Conf. Ser., Vol. {\bf 416}}
\arxivB{0904.2842}\adsT{2009ASPC..416..433B}\ownE{B/2009/Bran09_boulder}

\item[31.]
\Brandenburg\yproc{2009}{3284}
{3300}{Nonlinear aspects of astrobiological research}
{Encyclopedia of Complexity and System Science}
{R. A. Meyers}{Springer}
%ISBN: 978-0-387-69572-3
\arxivB{0809.0261}\doiT{10.1007/978-0-387-30440-3_195}\httpT{http://www.springer.com/physics/complexity/book/978-0-387-69572-3}\ownE{B/2008/Bran08_encyc}
%DOI: 10.1007/978-0-387-30440-3_195
%Axel Brandenburg, "Exobiology (theoretical), Complexity in", Robert A.
%Meyers (ed.),  Encyclopedia of Complexity and Systems Science

\item[30.]
\Brandenburg\yproc{2008}{159}
{166}{Paradigm shifts in solar dynamo modeling}
{Cosmic Magnetic Fields: From Planets, to Stars and Galaxies,
Proceedings of the International Astronomical Union, IAU Symposium, Volume 259}
{K.\ G.\ Strassmeier, A.\ G. Kosovichev \& J.\ E. Beckman}
{Cambridge University Press}
\arxivB{0901.3789}\adsT{2009IAUS..259..159B}\doiT{10.1017/S1743921309030403}\ownE{B/2008/Bran08_tenerife}

\item[29.]
\Brandenburg: 2007, ``Hydromagnetic Dynamo Theory,'' {\it Scholarpedia},
p.10320 \\
(\url{http://www.scholarpedia.org/article/Hydromagnetic_Dynamo_Theory})
%\httpS{http://www.scholarpedia.org/article/Hydromagnetic\_Dynamo\_Theory})

\item[28.]
\Brandenburg, \& von Rekowski, B.\yproc{2007}{374}
{281}{Dynamos in accretion discs}
{Coronae of Stars and Accretion Disks}
{M.\ Massi \& T.\ Preibisch}
{Mem. Soc. Astron. Ital., {\bf 78}}
\astrophB{0702493}\adsT{2007MmSAI..78..374B}\ownE{B/2007/Bran+vReko07}

\item[27.]
\Brandenburg\yproc{2007}{457}
{466}{Near-surface shear layer dynamics}
{Convection in Astrophysics}
{F.\ Kupka, I.\ W.\ Roxburgh \& K.\ L.\ Chan}
{Proc.\ Int.\ Astron.\ Union, IAUS 239}
\astrophB{0701057}\adsT{2007IAUS..239..457B}\ownE{B/2007/Bran07_shear}

\item[26.]
\Brandenburg\yproc{2007}{27}
{54}{The solar interior -- radial structure, rotation, solar activity cycle}
{Handbook of Solar-Terrestrial Environment}
{Y.\ Kamide \& A.\ C.-L.\ Chian}{Springer}
\astrophB{0703711}\adsT{2007hste.book...28B}\ownE{B/2007/Bran07_interior}

\item[25.]
Pudritz, R. E., Ouyed, R., Fendt, C., \& \Brandenburg\yproc{2007}{277}
{294}{Disk winds, jets, and outflows: theoretical and computational foundations}
{Protostars and Planets V}{B. Reipurth, D. Jewitt, \& K. Keil}{LPI}
\astrophB{0603592}\adsE{2006astro.ph..3592P}

\item[24.]
\Brandenburg, Haugen, N. E. L., \& Mee, A. J.\yproc{2005}{139}
{146}{Nonhelical turbulent dynamos: shocks and shear}
{The magnetized plasma in galaxy evolution}
{K.T. Chy\.zy, K. Otmianowska-Mazur, M. Soida, and R.-J. Dettmar}
{Jagiellonian University}
\astrophB{0501006}\adsT{2005mpge.conf..139B}\ownE{B/2005/Bran+Hau+Mee05}

\item[23.]
\Brandenburg\yproc{2005}{219}
{253}{Importance of magnetic helicity in dynamos}
{Cosmic magnetic fields, Lect. Notes Phys., Vol. {\bf664}}
{R. Wielebinski \& R. Beck}{Springer}
\astrophB{0412366}\adsT{2005LNP...664..219B}\ownE{B/2005/Bran05_LNP}

\item[22.]
\Brandenburg, \& Blackman, E. G.\yproc{2005}{101}
{104}{Ejection of bi-helical fields from the sun}
{Magnetic field and Helicity in the Sun and the Heliosphere}
{D. Rust \& B. Schmieder}
{Highlights of Astronomy, Vol. {\bf 13}}
\astrophB{0312543}\adsE{2003IAUJD...3E..33B}

\item[21.]
\Brandenburg, Sandin, C., \& K\"apyl\"a, P. J.\yproc{2004}{57}
{64}{Helical coronal ejections and their role in the solar cycle}
{Multi-Wavelength Investigations of Solar Activity}
{A. V. Stepanov, E. E. Benevolenskaya \& A. G. Kosovichev}
{Proc.\ Int.\ Astron.\ Union, IAUS 223}
\astrophB{0407598}\adsT{2004IAUS..223...57B}\ownE{B/2004/Bran+San+Kap04}

\item[20.]
\Brandenburg, Dintrans, B., \& Haugen, N. E. L.\yproc{2004}{122}
{136}{Shearing and embedding box simulations of the magnetorotational
instability}
{MHD Couette flows: experiments and models}
{R. Rosner, G. R\"udiger, \& A. Bonanno}
{AIP Conf. Proc. {\bf 733}}
\astrophB{0412363}\adsT{2004AIPC..733..122B}\ownE{B/2004/Bran+Din+Hau04}
%http://proceedings.aip.org/dbt/dbt.jsp?KEY=APCPCS&Volume=733&Issue=1

\item[19.]
\Brandenburg, \& Blackman, E. G.\yproc{2003}{233}
{242}{Helical surface structures}
{Modelling of Stellar Atmospheres}
{N.~E.~Piskunov, W.~W.~Weiss, \& D.~F.~Gray}
{IAU Symp., Vol.\ {\bf 210}}
\astrophB{0212019}\adsE{2003IAUS..210..233B}

\item[18.]
\Brandenburg, Haugen, N. E. L., \& Dobler, W.\yproc{2003}{33}
{53}{MHD simulations of small and large scale dynamos}
{Turbulence, Waves, and Instabilities in the Solar Plasma}
{R. Erd\'elyi, K. Petrovay, B. Roberts, \& M. Aschwanden}
{Kluwer Acad. Publ., Dordrecht}
\astrophS{0303371}

\item[17.]
\Brandenburg\yproc{2003}{402}
{431}{The helicity issue in large scale dynamos}
{Simulations of magnetohydrodynamic turbulence in astrophysics}
{E. Falgarone, T. Passot}
{Lecture Notes in Physics, Vol. 614. Berlin: Springer}
\astrophB{0207394}\adsT{2003tmfa.conf..402B}\ownE{B/2003/Brand_Paris03}

\item[16.]
\Brandenburg\yproc{2003}{269}
{344}{Computational aspects of astrophysical MHD and turbulence}
{Advances in nonlinear dynamos
(The Fluid Mechanics of Astrophysics and Geophysics, Vol.\ {\bf9})}
{A. Ferriz-Mas \& M. N\'u\~nez}
{Taylor \& Francis, London and New York}
\astrophB{0109497}\adsT{2003and..book..269B}\adsT{2003eclm.book..269B}\ownE{B/2003/Bran_comp03}

\item[15.]
\Brandenburg\yproc{2001}{125}
{132}{The inverse cascade in turbulent dynamos}
{Dynamo and dynamics, a mathematical challenge}
{P. Chossat, D. Armbruster, \& O. Iuliana}
{Nato ASI Series {\bf 26}, Kluwer Publ.}
\astrophS{0012112}\ownE{B/2001/Bran01_NatoASI}

\item[14.]
\Brandenburg\yproc{2001}{144}
{151}{The solar dynamo: old, recent, and new problems}
{Recent Insights into the Physics of the Sun and Heliosphere:
Highlights from SOHO and Other Space Missions}
{P.\ Brekke, B. Fleck, \& J.\ B.\ Gurman}
{Astron. Soc. Pac. Conf. Ser., Vol. {\bf 203}}
\astrophB{0011579}\adsT{2001IAUS..203..144B}\ownE{B/2001/Bran01_IAU}

\item[13.]
\Brandenburg\yproc{2001}{1543}
{1547}{Magnetohydrodynamics of accretion discs}
{Encyclopedia of Astronomy and Astrophysics}{P.~Murdin}
{London: Nature Publishing Group, and Bristol: Institute of Physics Publishing}
(\url{http://www.ency-astro.com}\adsT{2000eaa..bookE2226B}\ownT{B/2001/Bran_encyc01})
%pdf file does exist.

\item[12.]
\Brandenburg \& Saar, S. H.\yproc{2000}{381}
{390}{Dynamo mechanisms}
{Stellar clusters and associations: convection, rotation, and dynamos}
{R.\ Pallavicini, G.\ Micela \& S.\ Sciortino}
{Astron. Soc. Pac. Conf. Ser., Vol. {\bf 198}}
\adsB{2000ASPC..198..381B}\ownE{B/2000/Bran+Saar00}

\item[11.]
\Brandenburg, Nordlund, \AA., \& Stein, R. F.\yproc{2000}{85}
{105}{Astrophysical convection and dynamos}
{Geophysical and Astrophysical Convection}
{P. A. Fox \& R. M. Kerr}{Gordon and Breach Science Publishers}
\adsB{2000gac..conf...85B}\ownE{B/2000/BNS00_gac}
%\psB{B/2000/BNS00_gac}\ownE{B/2000/BNS00_gac}
%\url{http://books.google.com/books?id=6gdsBIbA7pQC&pg=PA85&dq=Geophysical+and+Astrophysical+Convection&hl=en&sa=X&ved=0ahUKEwinhrrRp9bSAhUrsVQKHcy-Cu8Q6AEIFDAA#v=onepage&q=Geophysical%20and%20Astrophysical%20Convection&f=false}
%http://www.nordita.org/~brandenb/Own_Papers/B/2000/BNS00_gac.pdf
%http://books.google.com/books?hl=en&lr=&id=6gdsBIbA7pQC&oi=fnd&pg=PR12&dq=Geophysical+and+Astrophysical+Convection&ots=2GWgP9eqaO&sig=IHTxo8Yzp-wghcMauPRVcoXeNHI#v=onepage&q=Geophysical%20and%20Astrophysical%20ConvectionBrand&f=false

%1992PhRvA..46.4819B
%1995PhPl....2.1148B
%1995ApJ...446..741B
%1995MNRAS.276..651B
%1996JFM...306..325B
%1992ApJ...392..647N
%1991JFM...225....1V
%1989ApJ...342L..95S
%1989JFM...205..297M
%1995PhRvE..52..636P
%1996ApJ...473..494B


\item[10.]
\Brandenburg\yproc{1999}{65}
{73}{Helicity in large-scale dynamo simulations}
{Magnetic Helicity in Space and Laboratory Plasmas}
{M. R. Brown, R. C. Canfield, A. A. Pevtsov}
{Geophys. Monograph {\bf 111}, American Geophysical Union, Florida}
\adsB{1999GMS...111...65B}\ownE{B/1999/Bran99}

\item[9.]
\Brandenburg\yproc{1998}{61}
{86}{Disc Turbulence and Viscosity}
{Theory of Black Hole Accretion Discs}
{M. A. Abramowicz, G. Bj\"ornsson \& J. E. Pringle}
{Cambridge University Press}
\adsB{1998tbha.conf...61B}\ownE{B/1998/Bran98_iceland}

\item[8.]
\Brandenburg\yproc{1998}{173}
{191}{Theoretical Basis of Stellar Activity Cycles}
{Tenth Cambridge Workshop on Cool Stars, Stellar Systems, and the Sun}
{R. Donahue \& J. Bookbinder}{Astron.\ Soc.\ Pac.\ Conf.\ Ser., Col. {\bf 154}}
\adsB{1998ASPC..154..173B}\ownE{B/1998/Bran98_cool}

\item[7.]
\Brandenburg\yproc{1997}{359}
{388}{Recent developments in the theory of large-scale dynamos}
{Past and present variability of the solar-terrestrial system:
measurement, data analysis and theoretical models.
Proceedings of the International School of Physics
``Enrico Fermi'' Course CXXXIII}
{G. Cini Castagnoli \& A. Provenzale}{IOS Press, Amsterdam}
\adsS{1997ppvs.conf..359B}

\item[6.]
\Brandenburg, \& Campbell, C. G.\yproc{1997}{109}
{124}{Modelling magnetised accretion discs}
{Accretion disks -- New aspects}
{H. Spruit \& E. Meyer-Hofmeister}{Springer-Verlag}
\adsB{1997LNP...487..109B}\ownE{B/1997/Bran+Camp97}

\item[5.]
\Brandenburg, Nordlund, \AA., Stein, R. F., Torkelsson, U.\yproc{1996}{285}
{290}{Dynamo generated turbulence in disks: value and variability of alpha}
{Physics of Accretion Disks}
{S. Kato, S. Inagaki, S. Mineshige \& J. Fukue}
{Gordon and Breach Science Publishers}
\adsB{1996pada.conf..285B}\psT{B/1996/BNST_kyoto96}\ownE{B/1996/BNST_kyoto96}

\item[4.]
\Brandenburg\yproc{1994}{73}
{84}{Hydromagnetic simulations of the solar dynamo}
{Advances in Solar Physics}{G. Belvedere, W. Mattig \& M. Rodon\'o}
{Lecture Notes in Physics {\bf 432}, Springer-Verlag}
\adsB{1994LNP...432...73B}\ownE{B/1994/Bran94}

\item[3.]
\Brandenburg\yproc{1994}{117}
{159}{Solar Dynamos: Computational Background}
{Lectures on Solar and Planetary Dynamos}
{M. R. E. Proctor \&  A. D. Gilbert}
{Cambridge University Press}
\adsS{1994lspd.conf..117B}

\item[2.]
\Brandenburg\yproc{1993}{111}
{121}{Simulating the solar dynamo}
{The Cosmic Dynamo}{F.~Krause, K.-H.~R\"adler, \& G.~R\"udiger}
{Kluwer Acad. Publ., Dordrecht}
\adsB{1993IAUS..157..111B}\ownE{B/1993/Bran93}
% No. 58.080.031
% Atmosphere, Interior, Neutrinos

\item[1.]
\Brandenburg, Tuominen, I.\yproc{1991}{223}
{233}{The solar dynamo}
{The Sun and cool stars: activity, magnetism, dynamos, IAU Coll. 130}
{I. Tuominen, D. Moss \& G. R\"udiger}
{Lecture Notes in Physics {\bf 380}, Springer-Verlag}
\adsB{1991LNP...380..223B}\ownE{B/1991/Bran+Tuom91}
% No. 54.080.071
% Atmosphere, Interior, Neutrinos

\end{itemize}
\subsection*{C. Conference proceedings} %C C C

\begin{itemize}

%https://www.mdpi.com/books/reprint/10271-turbulence-from-earth-to-planets-stars-and-galaxies-commemorative-issue-dedicated-to-the-memory-of

\item[86.]
Kahniashvili, T., \Brandenburg, Kosowsky, A., Mandal, S., \& Roper Pol, A.\yproc{2020}{295}
{298}{Magnetism in the early universe}
{Astronomy in Focus, Vol.~14}
{M.\ T.\ Lago, ed.}{Proc.\ IAU Symp.\ A30}
\arxivB{1810.11876}\adsT{2020IAUGA..30..295K}\doiT{10.1017/S1743921319004447}\httpT{http://norlx51.nordita.org/~brandenb/tmp/FM8_IAU/}\ownE{C/2020/Kahn_etal20}

\item[85.]
Warnecke, \& \Brandenburg\yproc{2014}{134}
{137}{Coronal influence on dynamos}
{Magnetic fields throughout stellar evolution}
{M.\ Jardine}{Proc.\ IAU Symp., Vol.\ {\bf 302}}
\arxivB{1310.0787}\adsT{2014IAUS..302..134W}\doiT{10.1017/S1743921314001884}\ownE{C/2014/Warn+Bran14}

\item[84.]
Candelaresi, S., \& \Brandenburg\yproc{2013}{353}
{357}{Topological constraints on magnetic field relaxation}
{Solar and astrophysical dynamos and magnetic activity}
{A.\ Kosovichev}{Proc.\ IAU Symp., Vol.\ {\bf 294}}
\arxivB{1212.0879}\adsT{2013IAUS..294..353C}\doiT{10.1017/S1743921313002743}\ownE{C/2013/Cand_etal13}

\item[83.]
K\"apyl\"a, P. J., \Brandenburg, Kleeorin, N., Mantere, M. J., \& Rogachevskii, I.\yproc{2013}{283}
{288}{Flux concentrations in turbulent convection}
{Solar and astrophysical dynamos and magnetic activity}
{A.\ Kosovichev}{Proc.\ IAU Symp., Vol.\ {\bf 294}}
\arxivB{1211.2962}\adsT{2013IAUS..294..283K}\doiT{10.1017/S1743921313002640}\ownE{C/2013/Kapy_etal13b}

\item[82.]
Warnecke, J., K\"apyl\"a, P. J., Mantere, M. J., \& \Brandenburg\yproc{2013}{307}
{312}{Solar-like differential rotation and equatorward migration in a convective dynamo with a coronal envelope}
{Solar and astrophysical dynamos and magnetic activity}
{A.\ Kosovichev}{Proc.\ IAU Symp., Vol.\ {\bf 294}}
\arxivB{1211.0452}\adsT{2013IAUS..294..307W}\doiT{10.1017/S1743921313002676}\ownE{C/2013/Warn_etal13c}

\item[81.]
Candelaresi, S., \& \Brandenburg\yproc{2012}{49}
{53}{Magnetic helicity fluxes and their effect on stellar dynamos}
{Comparative Magnetic Minima: Characterizing quiet times in the Sun and stars}
{C.\ H.\ Mandrini \& D.\ F.\ Webb}{Proc.\ IAU Symp., Vol.\ {\bf 286}}
\arxivB{1111.2023}\adsT{2012IAUS..286...49C}\doiT{10.1017/S1743921312004620}\ownE{C/2012/Cand+Bran12}

\item[80.]
Warnecke, J., K\"apyl\"a, P. J., Mantere, M. J., \& \Brandenburg\yproc{2012}{154}
{158}{Coronal ejections from convective spherical shell dynamos}
{Comparative Magnetic Minima: Characterizing quiet times in the Sun and stars}
{C.\ H.\ Mandrini \& D.\ F.\ Webb}{Proc.\ IAU Symp., Vol.\ {\bf 286}}
\arxivB{1111.1763}\adsT{2012IAUS..286..154W}\doiT{10.1017/S1743921312004772}\ownE{C/2012/Warn_etal12}

\item[79.]
Del Sordo, F., Bonanno, A., \Brandenburg, \& Mitra, D.\yproc{2012}{65}
{69}{Spontaneous chiral symmetry breaking in the Tayler instability}
{Comparative Magnetic Minima: Characterizing quiet times in the Sun and stars}
{C.\ H.\ Mandrini \& D.\ F.\ Webb}{Proc.\ IAU Symp., Vol.\ {\bf 286}}
\arxivB{1111.1742}\adsT{2012IAUS..286...65D}\doiT{10.1017/S1743921312004644}\ownE{C/2012/DSor_etal12}

\item[78.]
Del Sordo, F., \& \Brandenburg\yprocS{2011}{373}
{375}{How can vorticity be produced in irrotationally forced flows?}
{Advances in Plasma Astrophysics}
{A.\ Bonanno, E.\ de Gouveia dal Pino, \& A.\ Kosovichev}{Proc.\ IAU Symp., Vol.\ {\bf 274}}
\arxivB{1012.4772}\adsT{2011IAUS..274..373S}\doiT{10.1017/S1743921311007307}\ownE{C/2011/DSor+Bran11}

\item[77.]
Kemel, K., \Brandenburg, Kleeorin, N., \& Rogachevskii, I.\yproc{2011}{473}
{475}{Turbulent magnetic pressure instability in stratified turbulence}
{Advances in Plasma Astrophysics}
{A.\ Bonanno, E.\ de Gouveia dal Pino, \& A.\ Kosovichev}{Proc.\ IAU Symp., Vol.\ {\bf 274}}
\arxivB{1012.4360}\adsT{2011IAUS..274..473K}\doiT{10.1017/S1743921311007538}\ownE{C/2011/Kem+Bran+Klee+Roga11_naxos}

\item[76.]
Candelaresi, S., \& \Brandenburg\yproc{2011}{464}
{466}{Magnetic helicity fluxes in alpha--Omega dynamos}
{Advances in Plasma Astrophysics}
{A.\ Bonanno, E.\ de Gouveia dal Pino, \& A.\ Kosovichev}{Proc.\ IAU Symp., Vol.\ {\bf 274}}
\arxivB{1012.4354}\adsT{2011IAUS..274..464C}\doiT{10.1017/S1743921311007502}\ownE{C/2011/Cand+Bran11_naxos}

\item[75.]
Warnecke, J., \Brandenburg, \& Mitra, D.\yproc{2011}{306}
{309}{Plasmoid ejections driven by dynamo action underneath a spherical surface}
{Advances in Plasma Astrophysics}
{A.\ Bonanno, E.\ de Gouveia dal Pino, \& A.\ Kosovichev}{Proc.\ IAU Symp., Vol.\ {\bf 274}}
\arxivB{1011.4299}\adsT{2011IAUS..274..306W}\doiT{10.1017/S1743921311007186}\ownE{C/2011/Warn+Bran+Mitra11b}

\item[74.]
Candelaresi, S., Del Sordo, F., \& \Brandenburg\yproc{2011}{461}
{463}{Decay of trefoil and other magnetic knots}
{Advances in Plasma Astrophysics}
{A.\ Bonanno, E.\ de Gouveia dal Pino, \& A.\ Kosovichev}{Proc.\ IAU Symp., Vol.\ {\bf 274}}
\arxivB{1011.0417}\adsT{2011IAUS..274..461C}\doiT{10.1017/S1743921311007496}\ownE{C/2011/Cand+DSor+Bran11_naxos}

\item[73.]
Cantiello, M., Braithwaite, J., \Brandenburg, Del Sordo, F., K\"apyl\"a, P., \& Langer, N.\yproc{2011}{200}
{203}{Turbulence and magnetic spots at the surface of hot massive stars}
{Physics of Sun and Star Spots}
{D. P. Choudhary \& K. G. Strassmeier}{Proc.\ IAU Symp., Vol.\ {\bf 273}}
\arxivB{1010.2498}\adsT{2011IAUS..273..200C}\doiT{10.1017/S1743921311015249}\ownE{C/2011/Cantiello_etal11b}

\item[72.]
Kemel, K., \Brandenburg, Kleeorin, N., \& Rogachevskii, I.\yproc{2011}{83}
{88}{The negative magnetic pressure effect in stratified turbulence}
{Physics of Sun and Star Spots}
{D. P. Choudhary \& K. G. Strassmeier}{Proc.\ IAU Symp., Vol.\ {\bf 273}}
\arxivB{1010.1659}\adsT{2011IAUS..273...83K}\doiT{10.1017/S1743921311015055}\ownE{C/2011/Kem+Rran+Klee+Roga11}

\item[71.]
Warnecke, J., \& \Brandenburg\yproc{2011}{256}
{260}{Dynamo generated field emergence through recurrent plasmoid ejections}
{Physics of Sun and Star Spots}
{D. P. Choudhary \& K. G. Strassmeier}{Proc.\ IAU Symp., Vol.\ {\bf 273}}
\arxivB{1010.0218}\adsT{2011IAUS..273..256W}\ownE{C/2011/Warn+Bran11b}

\item[70.]
Cantiello, M., Braithwaite, J., \Brandenburg, Del Sordo, F., K\"apyl\"a, P., \& Langer, N.\yproc{2011}{32}
{37}{3D MHD simulations of subsurface convection in OB stars}
{Active OB stars: structure, evolution, mass loss and critical limits}
{C. Neiner, G. Wade, G. Meynet, \& G. Peters}{Proc.\ IAU Symp., Vol.\ {\bf 272}}
\arxivB{1009.4462}\adsT{2011IAUS..272...32C}\doiT{10.1017/S174392131100994X}\ownE{C/2011/Cantiello_etal11a}

\item[69.]
Del Sordo, F., \& \Brandenburg\yproc{2011}{375}
{376}{Vorticity from irrotationally forced flow}
{Astrophysical Dynamics: from Stars to Galaxies}
{N.\ Brummell, A.\ S.\ Brun, M.\ S.\ Miesch, \& Y.\ Ponty}{Proc.\ IAU Symp., Vol.\ {\bf 271}}
\arxivB{1009.0147}\adsT{2011IAUS..271..375S}\doiT{10.1017/S1743921311017868}\ownE{C/2011/DSor+Bran11}

\item[68.]
Candelaresi, S., Del Sordo, F., \& \Brandenburg\yproc{2011}{369}
{370}{Influence of Magnetic Helicity in MHD}
{Astrophysical Dynamics: from Stars to Galaxies}
{N.\ Brummell, A.\ S.\ Brun, M.\ S.\ Miesch, \& Y.\ Ponty}{Proc.\ IAU Symp., Vol.\ {\bf 271}}
\arxivB{1008.5235}\adsT{2011IAUS..271..369C}\doiT{10.1017/S1743921311017832}\ownE{C/2010/Cand+DSor+Bran10}

\item[67.]
Warnecke, J., \& \Brandenburg\yproc{2011}{407}
{408}{Recurrent flux emergence from dynamo-generated fields}
{Astrophysical Dynamics: from Stars to Galaxies}
{N.\ Brummell, A.\ S.\ Brun, M.\ S.\ Miesch, \& Y.\ Ponty}{Proc.\ IAU Symp., Vol.\ {\bf 271}}
\arxivB{1008.5278}\adsT{2011IAUS..271..407W}\ownE{C/2010/Warn+Bran10}

\item[66.]
Mitra,  D., Tavakol, R., \Brandenburg, \& K\"apyl\"a, P. J.\yproc{2010}{197}
{201}{Oscillatory migratory large-scale fields in mean-field and direct simulations}
{Solar and Stellar Variability: Impact on Earth and Planets, Vol. {\bf 264}}
{A.\ Kosovichev et al.}
{CUP}
\adsB{2010IAUS..264..197M}\ownE{C/2010/Mitra_etal10}

\item[65.]
\Brandenburg, \& Del Sordo, F.\yproc{2010}{432}
{433}{Turbulent diffusion and galactic magnetism}
{Magnetic Fields in Diffuse Media}
{E.\ de Gouveia Dal Pino}{Highlights of Astronomy, Vol. {\bf 15}, CUP}
\arxivB{0910.0072}\adsT{2010HiA....15..432B}\doiT{10.1017/S1743921310010148}\ownE{C/2009/Bran+DSor09}

\item[64.]
Gustafsson, M., \Brandenburg, Lemaire, J. L., \&
Field, D.\yproc{2007}{183}
{187}{Probing turbulence in OMC1 at the star forming scale: observations and simulations}
{Triggered Star Formation in a Turbulent ISM}
{B.\ G.\ Elmegreen \& J.\ Palous}{Proc.\ IAU Symp., Vol.\ {\bf 237}}
\ownS{C/2007/Gustafsson_etal07}

\item[63.]
\Brandenburg\pproc{2006}
{Why coronal mass ejections are necessary for the dynamo, JD-8}
{Highlights of Astronomy, Vol. {\bf 14}}
{K.\ G.\ Strassmeier \& A.\ Kosovichev}
{CUP}
\astrophB{0701056}\adsT{2007HiA....14..291B}\ownE{C/2007/Bran07_cycle}

\item[62.]
\Brandenburg, \& K\"apyl\"a, P. J.\pproc{2006}
{Connection between active longitudes and magnetic helicity}
{Solar activity: exploration, understanding and prediction}{H. Lundstedt}
{ESA, ESTEC Noordwijk, The Netherlands}
(\astroph{0512639})

\item[61.]
\Brandenburg\pproc{2006}
{Distributed versus tachocline dynamos}
{Solar activity: exploration, understanding and prediction}{H. Lundstedt}
{ESA, ESTEC Noordwijk, The Netherlands}
(\astroph{0512638})

\item[60.]
\Brandenburg\yproc{2006}{121}
{126}{Location of the solar dynamo and near-surface shear}
{Solar MHD: Theory and Observations -- a High Spatial Resolution Perspective}
{J. W. Leibacher, H. Uitenbroek, \& R. F. Stein}
{Astron. Soc. Pac. Conf. Ser., Vol. {\bf 354}}
\astrophB{0512637}\adsT{2006ASPC..354..121B}\ownE{C/2006/Bran06_SacPeak}

\item[59.]
\Brandenburg, K\"apyl\"a, P., \& Mohammed, A.\yproc{2005}{3}
{6}{Passive scalar diffusion as a damped wave}
{Progress in Turbulence}{J. Peinke, A. Kittel, S. Barth, \& M. Oberlack}
{Springer-Verlag}
\physicsS{0404118}

\item[58.]
von Rekowski, B., \& \Brandenburg\yproc{2004}{476}
{479}{Structured, dynamo driven stellar and disc winds}
{Asymmetrical Planetary Nebulae III: Winds, Structure and the Thunderbird}
{M. Meixner, J. H. Kastner, B. Balick \& N. Soker}
{Astron. Soc. Pac. Conf. Ser., Vol. {\bf 313}}
\astrophB{0310398}\adsE{2004ASPC..313..476V}

\item[57.]
Hindmarsh, M., Christensson, M., \& \Brandenburg\pproc{2003}
{Decay of magnetic fields in the early universe}
{Strong and Electroweak Matter}
{M. G. Schmidt}{World Scientific, Singapore}
\astrophB{0302320}\adsE{2003sem..conf..482H}

\item[56.]
Hindmarsh, M., Christensson, M., \& \Brandenburg\pproc{2003}
{MHD inverse cascade in the early Universe}
{COSMO-01}
{}
{}
\astrophS{0201466}

\item[55.]
Dintrans, B., \Brandenburg, Nordlund, \AA., \& Stein, R. F.\yproc{2003}{511}
{514}{On the generation of internal gravity waves by penetrative convection}
{SF2A-2003: Semaine de l'Astrophysique Fran{\c c}aise}
{F. Combes, D. Barret, \& T. Contini}{EdP-Sciences, Conference Series}
\adsS{2003sf2a.conf..511D}

\item[54.]
Dintrans, B., \& \Brandenburg\yproc{2003}{243}
{244}{The analytic subspace to measure internal gravity waves in hydrosimulations}
{SF2A-2003: Semaine de l'Astrophysique Fran{\c c}aise}
{F. Combes, D. Barret, \& T. Contini}{EdP-Sciences, Conference Series}

\item[53.]
Dintrans, B., \Brandenburg, Nordlund, \AA., \& Stein, R. F.\yproc{2003}{216}
{219}{Stochastic excitation of internal gravity waves by overshooting convection}
{SF2A-2003: Semaine de l'Astrophysique Fran{\c c}aise}
{F. Combes, D. Barret, \& T. Contini}{EdP-Sciences, Conference Series}

\item[52.]
\Brandenburg, \& Blackman, E. G.\yproc{2002}{805}
{810}{Magnetic helicity and the solar dynamo}
{Solar variability: from core to outer frontiers}{A. Wilson}
{ESA SP-506, Volume 2, ESTEC Noordwijk, The Netherlands}
\adsB{2002svco.conf..805B}\ownE{C/2002/Bran+Black02}

\item[51.]
\Brandenburg\yproc{2002}{742}
{744}{Numerical simulations of turbulent dynamos}
{Highlights of Astronomy}
{D. Moss, R. Beck, \& A. Shukurov}
{Astron. Soc. Pac. Conf. Ser., Vol. {\bf 12}}
\astrophB{0010495}\adsT{2002HiA....12..742B}

\item[50.]
Saar, S. H., \& \Brandenburg\yproc{2001}{231}
{234}{Further analysis of stellar magnetic cycle periods}
{Magnetic Fields across the Hertzsprung-Russell Diagram}
{G. Mathys, S.K. Solanki, \& D.T. Wickramasinghe}
{Astron. Soc. Pac. Conf. Ser., Vol. {\bf 248}}
\astrophB{0105070}\adsT{2001mfah.conf..231S}

\item[49.]
Torkelsson, U., Ogilvie, G. I., \Brandenburg, Pringle, J. E.,
Nordlund, \AA., \& Stein, R. F.\yproc{2001}{681}
{686}{Magnetohydrodynamic turbulence in warped accretion discs}
{20th Texas Symposium on Relativistic Astrophysics}
{J. C. Wheeler \& H. Martel}
{American Institute of Physics (AIP) Press}
\astrophB{0103057}\adsT{2001tsra.conf..681T}

\item[48.]
\Brandenburg, \& Kerr, R. M.\yproc{2001}{358}
{365}{Helicity in hydro and MHD reconnection}
{Quantized Vortex Dynamics and Superfluid Turbulence}
{C. F. Barenghi, R. J. Donnelly, \& W. F. Vinen}
{Lecture Notes in Physics, Vol.\ {\bf571}, Springer Verlag}
\physicsB{0012210}\adsT{2001LNP...571..358B}\ownE{C/2001/Kerr+Bran01}

\item[47.]
Bigazzi, A., \Brandenburg, \& Subramanian, K.\yproc{2001}{117}
{124}{Sheared helical turbulence and the helicity constraint in
large-scale dynamos}{Dynamo and dynamics, a mathematical challenge}
{P. Chossat, D. Armbruster, \& O. Iuliana}
{Nato ASI Series {\bf 26}, Kluwer Publ.}
\astrophS{0012240}

\item[46.]
von Rekowski, B., Dobler, W., Shukurov, A., \& \Brandenburg\yproc{2001}{305}
{312}{Two-dimensional disk dynamos with vertical outflows into a halo}
{Dynamo and dynamics, a mathematical challenge}
{P. Chossat, D. Armbruster, \& O. Iuliana}
{Nato ASI Series {\bf 26}, Kluwer Publ.}
\astrophS{0012013}

\item[45.]
Kerr, R. M., \& \Brandenburg\unpub{2000}
{New tests for a singularity of ideal MHD}
\physicsB{0001016}\adsT{2000physics...1016K}\ownE{C/2000/Kerr+Bran00}

\item[44.]
\Brandenburg\yproc{2000}{1}
{8}{The dynamo effect in stars}
{Pacific Rim Conference}
{K. S. Cheng, H. F. Chau, K. L. Chan, \& K. C. Leung}
{Kluwer Acad. Publ., Dordrecht}
\adsB{2000stas.conf....1B}\ownE{C/2000/Bran00}

\item[43.]
Torkelsson, U., \Brandenburg, Nordlund, \AA., \&
Stein, R. F.\yproc{2000}{241}
{242}{Magnetohydrodynamic turbulence in accretion discs}
{Highly Energetic Physical Processes and Mechanisms for Emission
from Astrohysical Plasmas}
{P.\ C.\ H.\ Martens \& S.\ Tsuruta}
{Astron. Soc. Pac. Conf. Ser., Vol. {\bf 195}}

\item[42.]
Dobler, W., \Brandenburg, \& Shukurov, A.\yproc{1999}{347}
{352}{Pressure-driven outflow and magneto-centrifugal wind from a
dynamo active disc}
{Plasma Turbulence and Energetic Particles in Astrophysics}
{M.\ Ostrowski \& R.\ Schlickeiser}
{Publ.\ Astron.\ Obs.\ Jagiellonian Univ., Cracow}
\adsB{1999ptep.proc..347D}\ownE{C/1999/Dobl+Bran+Shuk99}
 
\item[41.]
Stein, R. F., Georgobiani, D., Bercik, D. J., \Brandenburg,
Nordlund, \AA.\yproc{1999}{193}
{193}{Magneto-convection}
{Stellar Structure: Theory and Tests of Convective Energy Transport}
{A. Gimenez, E. F. Guinan \& B. Montesinos}
{Astron. Soc. Pac. Conf. Ser., Vol. {\bf 173}}
\adsB{1999ASPC..173..193S}\ownE{C/1999/Stein_etal99}

\item[40.]
\Brandenburg\yproc{1999}{13}
{21}{Simulations and observations of stellar dynamos: evidence for a
magnetic alpha-effect}
{Stellar dynamos: nonlinearity and chaotic flows}
{M. N\'u\~nez \& A. Ferriz-Mas}
{Astron. Soc. Pac. Conf. Ser., Vol. {\bf 178}}
\adsB{1999ASPC..178...13B}\ownE{C/1999/Bran99_FerrizMas}

\item[39.]
Korpi, M. J., \Brandenburg, Shukurov, A. \& Tuominen, I.\yproc{1999}{445}
{448}{A local three-dimensional model of the supernova regulated ISM}
{A. R. Taylor, T. L. Landecker, \& G. Joncas}
{New Perspectives on the Interstellar Medium}
{Astron. Soc. Pac. Conf. Ser., Vol. {\bf 168}}
\adsB{1999ASPC..168..445K}\ownE{C/1999/Korpi_etal99_ASP}

\item[38.]
Torkelsson, U., Ogilvie, G. I., \Brandenburg, Pringle, J. E.,
Nordlund, \AA., \& Stein, R. F.\yproc{1999}{422}
{427}{The dynamics of turbulent viscosity}
{High Energy Processes in Accreting Black Holes}
{J.\ Poutanen \& R.\ Svensson}
{Astron. Soc. Pac. Conf. Ser., Vol. {\bf 161}}
\adsB{1999ASPC..161..422T}\ownE{C/1999/Torkel_etal99_ASP}

\item[37.]
Korpi, M. J., \Brandenburg, Shukurov, A. \&
Tuominen, I.\yproc{1999}{127}
{131}{Vortical motions driven by supernova explosions}
{J. Franco \& A. Carrami\~nana}
{Interstellar Turbulence}{Cambridge University Press}
\adsS{1999intu.conf..127K}

\item[36.]
Torkelsson, U., Ogilvie, G. I., \Brandenburg, Nordlund, \AA ., Stein,
R. F.\yproc{1998}{69}
{72}{Exploring magnetohydrodynamic turbulence on the computer}
{Accretion processes in astrophysical systems:  Some like it hot}
{S. S. Holt \& T. R. Kallman}
{American Institute of Physics Conf. Proc.}
\adsS{1998AIPC..431...69T}

\item[35.]
Torkelsson, U., \Brandenburg, Nordlund, \AA., Stein, R. F.\yproc{1997}{210}
{214}{Magnetohydrodynamic turbulence in accretion discs:
towards more realistic models}
{Accretion phenomena and related outflows, IAU Coll. 163}
{D. T. Wickramasinghe, G. V. Bicknell \& L. Ferrario}
{Astron.\ Soc.\ Pac.\ Conf.\ Ser., Col. {\bf 121}}
\adsB{1997ASPC..121..210T}\ownE{C/1997/Tork+Bran+Nord+Stein97}

\item[34.]
Torkelsson, U., Ogilvie, G. I., \Brandenburg, Nordlund, \AA.,
Stein, R. F.\yproc{1997}{134}
{153}{The nonlinear evolution of a single mode of the
magnetic shearing instability}
{Accretion disks - New aspects}
{H. Spruit, \& E. Meyer-Hofmeister}
{Springer-Verlag}
\ownS{C/1997/Tork+Ogil+Bran+Nord+Stein97}

\item[33.]
Miesch, M., \Brandenburg, Zweibel, E., \& Toomre, J.\yproc{1995}{253}
{260}{Non-local transport in turbulent MHD convection}
{Proceedings of Fourth SOHO Workshop: Helioseismology}
{ESA SP-376, Volume 2}{Pacific Grove, California}
\adsB{1995ESASP.376b.253M}\ownE{C/1995/Miesch+Bran+Zwei+Toom95}

\item[32.]
\Brandenburg, Nordlund, \AA., Stein, R. F., Torkelsson, U.\yproc{1995}{385}
{390}{Dynamo generated turbulence is discs}
{Small-scale structures in three-dimensional hydro and
magnetohydrodynamic turbulence}
{M. Meneguzzi, A. Pouquet, \& P. L. Sulem}
{Lecture Notes in Physics {\bf 462}, Springer-Verlag}

\item[31.]
\Brandenburg\yproc{1994}{409}
{411}{Generation of field-aligned current tubes in magnetospheric shear layers}
{Second International Conference on Substorms}
{J. R. Kan, J. D. Craven \& S.-I. Akasofu}{Geophysical Institute}

\item[30.]
Keppens, R., Charbonneau, P., MacGregor, K. B., \Brandenburg\yproc{1994}{193}
{195}{Angular momentum loss from the young sun: improved wind and dynamo models}
{Eighth Cambridge Workshop on Cool Stars, Stellar Systems, and the Sun}
{J.-P. Caillault}{Astron. Soc. Pac. Conf. Ser., Vol. {\bf 64}}
\adsB{1994ASPC...64..193K}\ownE{C/1994/KCMGB94}

\item[29.]
Saar, S. H., \Brandenburg, Donahue, R. A., \& Baliunas, S. L.\yproc{1994}{468}
{470}{The evolution of stellar dynamo variations}
{Eighth Cambridge Workshop on Cool Stars, Stellar Systems, and the Sun}
{J.-P. Caillault}{Astron. Soc. Pac. Conf. Ser., Vol. {\bf 64}}
\adsB{1994ASPC...64..468S}\ownE{C/1994/SBDBR94}

\item[28.]
\Brandenburg, Charbonneau, P., Kitchatinov, L. L., \& R\"udiger, G.\yproc{1994}{354}
{356}{Stellar Dynamos: The Rossby number dependence}
{Eighth Cambridge Workshop on Cool Stars, Stellar Systems, and the Sun}
{J.-P. Caillault}{Astron. Soc. Pac. Conf. Ser., Vol. {\bf 64}}
\adsB{1994ASPC...64..354B}\ownE{C/1994/BCKR94}

\item[27.]
\Brandenburg, Saar, S. H., Moss, D., Tuominen, I.\yproc{1994}{357}
{359}{Stellar dynamo models: from F to K}
{Eighth Cambridge Workshop on Cool Stars, Stellar Systems, and the Sun}
{J.-P. Caillault}{Astron. Soc. Pac. Conf. Ser., Vol. {\bf 64}}
\adsB{1994ASPC...64..357B}\ownE{C/1994/BSMT94}

\item[26.]
Kurths, J., Feudel, U., \Brandenburg: 1994,
``Complexity in inhomogeneous chaotic systems,''
in {\em The Paradigm of Self-Organization II}
Ed. G. J. Dalenoort, Gordon \& Breach, Ser. Studies in Cybernetics 24, 1994,
p.\ 157--170.

\item[25.]
\Brandenburg, Procaccia, I., Segel, D., Vincent, A.,
Manzini, M.\yproc{1993}{35}
{42}{Multifractality, near-singularities and the role of
stretching in turbulence}
{Solar and Planetary Dynamos}
{M. R. E. Proctor, P. C. Matthews \& A. M. Rucklidge}
{Cambridge University Press}
\adsS{1993spd..conf...35B}

\item[24.]
Moss, D., \Brandenburg\yproc{1993}{219}
{224}{The excitation of nonaxisymmetric magnetic fields in galaxies}
{Solar and Planetary Dynamos}
{M. R. E. Proctor, P. C. Matthews \& A. M. Rucklidge}
{Cambridge University Press}
\adsS{1993spd..conf..219M}

\item[23.]
Pulkkinen, P., Tuominen, I., \Brandenburg,
Nordlund, \AA. \& Stein, R.F.\yproc{1993}{123}
{127}{Reynolds stresses derived from simulations}
{The Cosmic Dynamo}{F.~Krause, K.-H.~R\"adler, \& G.~R\"udiger}
{Kluwer Acad. Publ., Dordrecht}
\adsB{1993IAUS..157..123P}\ownE{C/1993/PTB93}
% No. 58.062.019
% Magnetohydrodynamics, Plasma

\item[22.]
Kurths, J., \Brandenburg, Feudel, U., Jansen, W.\yproc{1993}{83}
{89}{Chaos in nonlinear dynamo models}
{The Cosmic Dynamo}{F.~Krause, K.-H.~R\"adler, \& G.~R\"udiger}
{Kluwer Acad. Publ., Dordrecht}
\adsB{1993IAUS..157...83K}\ownE{C/1993/KBFJ93}
% No. 58.062.017
% Magnetohydrodynamics, Plasma

\item[21.]
Donner, K. J., \Brandenburg, Thomasson, M.\yproc{1993}{333}
{337}{Galactic dynamos and dynamics}
{The Cosmic Dynamo}{F.~Krause, K.-H.~R\"adler, \& G.~R\"udiger}
{Kluwer Acad. Publ., Dordrecht}
\adsB{1993IAUS..157..333D}\ownE{C/1993/DBT93}
% No. 58.157.064
% Normal Galaxies

\item[20.]
Moss, D., \Brandenburg, Donner, K. J., Thomasson, M.\yproc{1993}{339}
{343}{Towards the magnetic field of M81}
{The Cosmic Dynamo}{F.~Krause, K.-H.~R\"adler, \& G.~R\"udiger}
{Kluwer Acad. Publ., Dordrecht}
\adsB{1993IAUS..157..339M}\ownE{C/1993/MBDT93}
% No. 58.157.065
% Normal Galaxies

\item[19.]
\Brandenburg, Moss, D., Tuominen, I.\yproc{1992}{536}
{542}{Turbulent pumping in the solar dynamo}
{The Solar Cycle}{K. L. Harvey}
{ASP Conf. Series, Vol. {\bf 27}}
\adsB{1992socy.work..536B}\ownE{C/1992/Bran+Moss+Tuom92_pump}
% No. 56.080.020
% Atmosphere, Interior, Neutrinos

\item[18.]
Stein, R.F., \Brandenburg, Nordlund, \AA.\yproc{1992}{148}
{157}{Magneto-Convection}
{Cool Stars, Stellar Systems, and the Sun}
{M. S. Giampapa \& J. A. Bookbinder}
{ASP Conf. Series, Vol. {\bf 26}}
\adsB{1992ASPC...26..148S}\ownE{C/1992/Stein+Bran+Nord92}
% No. 56.065.094
% Stellar Structure, Evolution

\item[17.]
\Brandenburg, Jennings, R. L., Nordlund, \AA., Stein, R.F.\yproc{1991}{371}
{374}{Magnetic flux tubes as coherent structures}
{Spontaneous formation of space-time structures and criticality}
{T. Riste \& D. Sherrington}{Nato ASI Series}

\item[16.]
\Brandenburg, Moss, D., Rieutord, M., R\"udiger, G., Tuominen, I.\yproc{1991}
{147}{150}{alpha--Lambda dynamos}
{The Sun and cool stars: activity, magnetism, dynamos, IAU Coll. 130}
{I. Tuominen, D. Moss \& G. R\"udiger}
{Lecture Notes in Physics {\bf 380}, Springer-Verlag}
\adsB{1991LNP...380..147B}\ownE{C/1991/BMRRT91}
% No. 54.062.109
% Magnetohydrodynamics, Plasma

\item[15.]
Pulkkinen, P., Tuominen, I., \Brandenburg, Nordlund,
\AA., Stein, R.F.\yproc{1991}{98}
{100}{Simulation of rotational effects on turbulence
in the solar convective zone}
{The Sun and cool stars: activity, magnetism, dynamos} 
{I. Tuominen, D. Moss \& G. R\"udiger} 
{Lecture Notes in Physics {\bf 380}, Springer-Verlag}
\adsB{1991LNP...380...98P}\ownE{C/1991/PTBNS91}
% No. 54.080.065
% Atmosphere, Interior, Neutrinos

\item[14.]
Jennings, R. L., \Brandenburg, Nordlund, \AA., Stein, R.F., Tuominen, I.: 
1991, 
``Magnetic tubes in  overshooting compressible convection,''
in {\em The Sun and cool stars: activity, magnetism, dynamos},
IAU Coll. 130, eds. I. Tuominen, D. Moss \& G. R\"udiger,
Lecture Notes in Physics, Springer-Verlag, pp. 92-94
\adsB{1991LNP...380...92J}\ownE{C/1991/JBNST91}
% No. 54.062.105
% Magnetohydrodynamics, Plasma

\item[13.]
\Brandenburg, Jennings, R. L., Nordlund, \AA., Stein, R. F.,
Tuominen, I.\yproc{1991}{86}
{88}{The role of overshoot in solar activity: 
A direct simulation of the dynamo}
{The Sun and cool stars: activity, magnetism, dynamos} 
{I. Tuominen, D. Moss \& G. R\"udiger} 
{Lecture Notes in Physics {\bf 380}, Springer-Verlag}
\adsB{1991LNP...380...86B}\ownE{C/1991/BJNST91}
% No. 54.080.064
% Atmosphere, Interior, Neutrinos

\item[12.]
Tuominen, I., Piskunov, N. E., Moss, D., \Brandenburg\yproc{1990}{73}
{75}{Surface imaging of giant stars and nonlinear dynamos}
{Sixth Cambridge Workshop on Cool Stars, Stellar Systems, and the Sun}
{G. Wallerstein}{ASP Conf. Series, Vol. {\bf 9}}
% No. 52.116.061
  
\item[11.]
Donner, K. J., \Brandenburg\yproc{1990}{85}
{88}{Effect of a conducting halo on the structure of galactic magnetic fields}
{Proc. Nordic-Baltic Astronomy Meeting}
{C.-I. Lagerkvist, D. Kiselman \& M. Lindgren}
{Uppsala Univ. (Sweden), Astronomiska Observatoriet}
\adsS{1990nba..meet...85D}
% No. 52.151.104
 
\item[10.]
Tuominen, I., R\"udiger, G., \Brandenburg\yproc{1990}{387}
{390}{Torsional oscillations and the solar dynamo regime}
{Solar Photosphere: Structure, Convection and Magnetic Fields} 
{J. O. Stenflo}{Kluwer Acad. Publ., Dordrecht}
\adsB{1990IAUS..138..387T}\ownE{C/1990/TRB90}
% No. 51.080.013
 
\item[9.]
\Brandenburg, Meinel, R., Moss, D., Tuominen, I.\yproc{1990}{379}
{382}{Variation of even and odd parity in solar dynamo models}
{Solar Photosphere: Structure, Convection and Magnetic Fields}  
{J. O. Stenflo}{Kluwer Acad. Publ., Dordrecht}
\adsB{1990IAUS..138..379B}\ownE{C/1990/Bran+Mein+Moss+Tuom90}
% No. 51.080.012
 
\item[8.]
\Brandenburg, Nordlund, \AA., Pulkkinen, P.,
Stein, R.F., Tuominen, I.: 1990,
``Turbulent diffusivities derived from simulations,''
in {\em Proceedings of the Finnish Astronomical Society 1990},
eds. K. Muinonen, M. Kokko, S. Pohjolainen, and P. Hakala, 
Helsinki 1990, p. 1-4
\adsB{1990fas..conf....1B}\ownE{C/1990/Bran+Nord+Pulkk+Stei+Tuom90}
% No. 51.062.217

\item[7.]
Donner, K.J., \Brandenburg: 1990,
``Non-axisymmetric magnetic fields in turbulent gas discs,''
in {\em Dynamics of Astrophysical Discs},
ed. J. A. Sellwood, Cambridge University Press, p. 151-152
\adsS{1989dad..conf..151D}
% No. 50.151.079

\item[6.]
\Brandenburg, Tuominen, I.: 1989, ``Solar magnetic fields and dynamo 
process,'' in: {\em Solar and Stellar Flares}, 
eds. B. M. Haisch, \& M. Rodon\`o, Publ. Astrophys. Obs. Catania, p. 369-372
% No. 51.080.060

\item[5.]
\Brandenburg, Pulkkinen, P., Tuominen, I.,
Nordlund, \AA, Stein, R. F.: 1989, 
``Simulation of MHD convection as a test for mean field theories,''
in {\em Turbulence and Nonlinear Dynamics in MHD Flows},
eds. M. Meneguzzi, A. Pouquet, \& P. L. Sulem, Elsevier Science Publ. B.V.
(North-Holland), p. 125-130

\item[4.]
\Brandenburg, Krause, F., Tuominen, I.\yproc{1989}{35}
{40}{Parity selection in nonlinear dynamos}
{Turbulence and Nonlinear Dynamics in MHD Flows}
{M. Meneguzzi, A. Pouquet \& P. L. Sulem}
{Elsevier Science Publ. B.V., North-Holland}
\adsB{1989tndm.conf...35B}\ownE{C/1989/Bran+Krau+Tuom89}
% No. 50.062.220

\item[3.]
Tuominen, I., R\"udiger, G., \Brandenburg\yproc{1988}{13}
{20}{Observational constraints for solar type dynamos}
{Activity in Cool Star Envelopes}{O. Havnes {\em et al.}}{Kluwer Acad. Publ.}
\adsB{1988ASSL..143...13T}\ownE{C/1988/Tuom+Rued+Bran88}
% No. 46.064.134

\item[2.]
\Brandenburg: 1988, ``Solar oscillations in the two year range,''
in {\em Proceedings of the Sixth Soviet-Finnish Astronomical Meeting}, 
eds. U. H\"anni, \& I. Tuominen, p. 34-39
\adsS{1988sfam.conf...34B}
% No. 49.080.049

\item[1.]
\Brandenburg: 1988,
``Gravity wave generation by large scale bubbles,''
in {\em Advances in Helio- and Asteroseismology}, 
eds. J. Christensen-Dalsgaard and S. Frandsen, Reidel Dordrecht, p. 383-386
\adsB{1988IAUS..123..383B}\ownE{C/1988/Bra88}
% No. 45.064.008

\end{itemize}

\subsection*{D. Public datasets, Zenodo publications, etc.}%D D D

\begin{itemize}

\item[{55.}~]
\Brandenburg, \& Vishniac, E. T.\ybook{2024}
%{Datasets for ``Magnetic helicity fluxes in dynamos from rotating inhomogeneous turbulence'' v2024.12.xx}{Zenodo, DOI:10.5281/zenodo.xxxxxxxx}
%\httpB{http://norlx65.nordita.org/~brandenb/projects/Omega-Gradu/}\doiE{10.5281/zenodo.xxxxxxxx}
{Datasets for ``Magnetic helicity fluxes in dynamos from rotating inhomogeneous turbulence''}
{(\url{http://norlx65.nordita.org/~brandenb/projects/Omega-Gradu/})}

\item[{54.}~]
Dehman, C., \& \Brandenburg\ybook{2024}
{Datasets for ``Reality of Inverse Cascading in Neutron Star Crusts'' v2024.12.18}{Zenodo, DOI:10.5281/zenodo.14513354}
\httpB{http://norlx65.nordita.org/~brandenb/projects/Reality-InvCasc-NS/}\doiE{10.5281/zenodo.14513354}

\item[{53.}~]
\Brandenburg, Iarygina, O., Sfakianakis, E. I., \& Sharma, R.\ybook{2024}
{Datasets for ``Magnetogenesis from axion-SU(2) inflation'' v2024.12.13}{Zenodo, DOI:10.5281/zenodo.14434086}
\httpB{http://norlx65.nordita.org/~brandenb/projects/magnetogenesis-SU2/}\doiE{10.5281/zenodo.14434086}

\item[{52.}~]
Vachaspati, T., \& \Brandenburg\ybook{2024}
{Datasets for ``Spectra of magnetic fields from electroweak symmetry breaking''}
{(\url{http://norlx65.nordita.org/~brandenb/projects/EW-B-statistics})}
%\httpS{http://norlx65.nordita.org/~brandenb/projects/EW-B-statistics}

\item[{51.}~]
Sharma, R., \Brandenburg, Subramanian, K., \& Vikman, A.\ybook{2024}
{Datasets for ``Lattice simulations of axion-U(1) inflation: gravitational waves, magnetic fields, and black holes'' v2024.06.18}{Zenodo, DOI:10.5281/zenodo.10527437}
\httpB{http://norlx65.nordita.org/~brandenb/projects/axion-U1/}\doiE{10.5281/zenodo.10527437}

\item[{50.}~]
\Brandenburg, \& Banerjee, A.\ybook{2024}
{Datasets for ``Turbulent magnetic decay controlled by two conserved quantities'' v2024.11.02}{Zenodo, DOI:10.5281/zenodo.14028931}
\httpB{http://norlx65.nordita.org/~brandenb/projects/Two-Conserved/}\doiE{10.5281/zenodo.14028931}

\item[{49.}~]
\Brandenburg, Neronov, A., \& Vazza, F.\ybook{2024}
{Datasets for ``Resistively controlled primordial magnetic turbulence decay'' v2024.01.18}{Zenodo, DOI:10.5281/zenodo.10527437}
\httpB{http://norlx65.nordita.org/~brandenb/projects/Reconnection-Decay/}\doiT{10.5531/sd.astro.9}\doiE{10.5281/zenodo.10527437}

\item[{48.}~]
Sharma, R., Dahl, J., \Brandenburg, \& Hindmarsh, M.\ybook{2023}
{Datasets for ``Shallow relic gravitational wave spectrum with acoustic peak'' v2023.11.10}{Zenodo, DOI:10.5281/zenodo.10101985}
\httpB{http://norlx65.nordita.org/~brandenb/projects/ShallowGW/}\doiE{10.5281/zenodo.10101985}

\item[{47.}~]
Iarygina, O., Sfakianakis, E. I., Sharma, R.. \& \Brandenburg\ybook{2023}
{Datasets for ``Backreaction of axion-SU(2) dynamics during inflation'' v2023.11.05}{Zenodo, DOI:10.5281/zenodo.10072163}
\httpB{http://norlx65.nordita.org/~brandenb/projects/axion-SU-2-inflation}\doiE{10.5281/zenodo.10072163}

\item[{46.}~]
\Brandenburg, \& Protiti, N. N.\ybook{2023}
{Datasets for ``Electromagnetic conversion into kinetic and thermal energies'' v2023.08.01}{Zenodo, DOI:10.5281/zenodo.8203242}
\httpB{http://norlx65.nordita.org/~brandenb/projects/EMconversion/}\doiE{10.5281/zenodo.8203242}

\item[{45.}~]
\Brandenburg, Clarke, E., Kahniashvili, T., Long, A. J., \& Sun, G.\ybook{2023}
{Datasets for ``Relic gravitational waves from the chiral plasma instability in the standard cosmological model'' v2023.07.17}{Zenodo, DOI:10.5281/zenodo.8157463}
\httpB{http://norlx65.nordita.org/~brandenb/projects/GWfromSM/}\doiE{10.5281/zenodo.8157463}

\item[{44.}~]
\Brandenburg, Sharma, R., \& Vachaspati, T.\ybook{2023}
{Datasets for ``Inverse cascading for initial MHD turbulence spectra between Saffman and Batchelor'' v2023.07.09}{Zenodo, DOI:10.5281/zenodo.8128611}
\httpB{http://norlx65.nordita.org/~brandenb/projects/Cubic-Hosking/}\doiE{10.5281/zenodo.8128611}

\item[{43.}~]
\Brandenburg, Kamada, K., Mukaida, K., Schmitz, K., \& Schober, J.\ybook{2023}
{Datasets for ``Chiral magnetohydrodynamics with zero total chirality'' v2023.08.20}{Zenodo, DOI:10.5281/zenodo.8267336}
\httpB{http://norlx65.nordita.org/~brandenb/projects/ZeroTotalChirality/}\doiE{10.5281/zenodo.8267336}

\item[{42.}~]
Sarin, N., \Brandenburg, \& Haskell, B.\ybook{2023}
{Supplemental Material to ``Confronting the neutron star population with inverse cascades'' v2023.06.27}{Zenodo, DOI:10.5281/zenodo.8088084}
\httpB{http://norlx65.nordita.org/~brandenb/projects/NSpop+InvCasc/}\doiE{10.5281/zenodo.8088084}

\item[{41.}~]
Mizerski, K. A., Yokoi, N., \& \Brandenburg\ybook{2023}
{Datasets for ``Cross-helicity effect on $\alpha$-type dynamo in non-equilibrium turbulence'' v2023.02.28}{Zenodo, DOI:10.5281/zenodo.7683615}
\httpB{http://norlx65.nordita.org/~brandenb/projects/nonEquilibriumDynamo/}\doiE{10.5281/zenodo.7683615}

\item[{40.}~]
\Brandenburg, Kamada, K., \& Schober, J.\ybook{2023}
{Datasets for ``Decay law of magnetic turbulence with helicity balanced by chiral fermions'' v2023.02.01}{Zenodo, DOI:10.5281/zenodo.7499431}
\httpS{http://norlx65.nordita.org/~brandenb/projects/DecayWithChiral/}

\item[{39.}~]
\Brandenburg\ybook{2023}
{Datasets for ``Quadratic growth during the COVID-19 pandemic: merging hotspots and reinfections'' v2023.01.02}{Zenodo, DOI:10.5281/zenodo.7499431}
\httpB{http://norlx65.nordita.org/~brandenb/projects/MergingHotspots/}\doiE{10.5281/zenodo.7499431}

\item[{38.}~]
\Brandenburg\ybook{2022}
{Datasets for ``Hosking integral in nonhelical Hall cascade'' v2022.11.24}{Zenodo, DOI:10.5281/zenodo.7357799}
\httpB{http://norlx65.nordita.org/~brandenb/projects/HallNonhel/}\doiE{10.5281/zenodo.7357799}

\item[{37.}~]
He, Y., Roper Pol, A., \& \Brandenburg\ybook{2022}
{Datasets of ``Modified propagation of gravitational waves from the early radiation era'' v2022.11.16}{Zenodo, DOI:10.5281/zenodo.7327770}
\httpB{http://norlx65.nordita.org/~yutong/projects/Horndeski/}\doiE{10.5281/zenodo.7327770}

\item[{36.}~]
\Brandenburg, Rogachevskii, I., \& Schober, J.\ybook{2022}
{Datasets for ``Dissipative magnetic structures and scales in small-scale dynamos'' v2022.9.18}{Zenodo, DOI:10.5281/zenodo.7090887}
\httpB{http://norlx65.nordita.org/~brandenb/projects/keta_vs_PrM}\doiE{10.5281/zenodo.7090887}

\item[{35.}~]
Carenza, P., Sharma, R., Marsh, M. C. D., \Brandenburg, M\"uller, E.\ybook{2022}
%{Datasets for ``Magnetohydrodynamics predicts heavy-tailed distributions of axion-photon conversion'' v2022.8.8}{Zenodo, DOI:10.5281/zenodo.6974228}
%\httpB{http://norlx65.nordita.org/~brandenb/projects/ALP_MHD}\doiE{10.5281/zenodo.6974228}
{Datasets for ``Magnetohydrodynamics predicts heavy-tailed distributions of axion-photon conversion'' v2022.8.8}{Zenodo, DOI:10.5281/zenodo.8342138}
\httpB{http://norlx65.nordita.org/~brandenb/projects/ALP_MHD}\doiT{10.5281/zenodo.8342138}\doiE{10.5281/zenodo.6974228}

\item[{34.}~]
\Brandenburg, Zhou, H., \& Sharma, R.\ybook{2022}
{Datasets for ``Batchelor, Saffman, and Kazantsev spectra in galactic small-scale dynamos'' v2022.07.19}{Zenodo, DOI:10.5281/zenodo.7281479}
\httpB{http://norlx65.nordita.org/~brandenb/projects/Kazantsev-Subinertial}\doiE{10.5281/zenodo.7281479}

\item[{33.}~]
Zhou, H., Sharma, R., \& \Brandenburg\ybook{2022}
{Datasets for ``Scaling of the Hosking integral in decaying magnetically-dominated turbulence'' v2022.06.14}{Zenodo, DOI:10.5281/zenodo.7112885}
\httpB{http://norlx65.nordita.org/~brandenb/projects/Saffman/}\doiE{10.5281/zenodo.7112885}

\item[{32.}~]
Sharma, R., \& \Brandenburg\ybook{2022}
{Supplemental Material and Datasets for ``Low frequency tail of gravitational wave spectra from hydromagnetic turbulence'' v2022.08.22}{Zenodo, DOI:10.5281/zenodo.7014823}
\httpB{http://norlx65.nordita.org/~brandenb/projects/LowFreqTail/}\doiE{10.5281/zenodo.7014823}

\item[{31.}~]
K\"apyl\"a, M. J., Rheinhardt, M., \& \Brandenburg\ybook{2022}
{Datasets for ``Fully compressible test-field method and its application to shear dynamos'' v2022.03.24}{Zenodo, DOI:10.5281/zenodo.6383190}
\httpB{http://www.nordita.org/~brandenb/projects/CompressibleTestfield/}\doiE{10.5281/zenodo.6383190}

\item[{30.}~]
\Brandenburg, \& Ntormousi, E.\ybook{2021}
{Datasets for ``Dynamo effect in unstirred self-gravitating turbulence'' v2021.12.06}{Zenodo, DOI:10.5281/zenodo.5760126}
\httpB{http://www.nordita.org/~brandenb/projects/SelfGrav/}\doiE{10.5281/zenodo.5760126}

\item[{29.}~]
Kahniashvili, T., Clarke, E., Stepp, J., \& \Brandenburg\ybook{2021}
{Datasets for ``Big bang nucleosynthesis limits and relic gravitational waves detection prospects'' v2021.11.18}{Zenodo, DOI:10.5281/zenodo.5709176}
\httpB{http://www.nordita.org/~brandenb/projects/GWs-BBN/}\doiE{10.5281/zenodo.5709176}

\item[{28.}~]
Roper Pol, A., Mandal, S., \Brandenburg, \& Kahniashvili, T.\ybook{2021}
{Datasets for ``Polarization of gravitational waves from helical MHD turbulent sources'' v2021.09.24}{Zenodo, DOI:10.5281/zenodo.5525504}
\httpB{http://www.nordita.org/~brandenb/projects/NonlinGW/}\doiE{10.5281/zenodo.5525504}

\item[{27.}~]
He, Y., Roper Pol, A., \& \Brandenburg\ybook{2021}
{Datasets for ``Leading-order nonlinear gravitational waves from reheating magnetogeneses'' v2021.09.23}{Zenodo, DOI:10.5281/zenodo.5524454}
%\httpB{http://www.nordita.org/~brandenb/projects/NonlinGW/}\doiE{10.5281/zenodo.5524454}
\httpB{http://www.nordita.org/~brandenb/projects/NonlinGW/}\doiE{10.5281/zenodo.5603013}

\item[{26.}~]
\Brandenburg, He, Y., \& Sharma, R.\ybook{2021}
{Datasets for ``Simulations of helical inflationary magnetogenesis and gravitational waves'' v2021.07.26}{Zenodo, DOI:10.5281/zenodo.5137202}
\httpB{http://www.nordita.org/~brandenb/projects/HelicalInflationaryGW/}\doiE{10.5281/zenodo.5137202}

\item[{25.}~]
\Brandenburg, \& Sharma, R.\ybook{2021}
{Datasets for ``Simulating relic gravitational waves from inflationary magnetogenesis'' v2021.06.04}{Zenodo, DOI:10.5281/zenodo.4900075}
\httpB{http://www.nordita.org/~brandenb/projects/InflationaryMagnetoGW/}\doiE{10.5281/zenodo.4900075}

\item[{24.}~]
Li, X.-Y., Mehlig, B., Svensson, G., \Brandenburg, \& Haugen, N. E. L.\ybook{2021}
{Datasets for ``Collision fluctuations of lucky droplets with superdroplets'' v2021.05.07}{Zenodo, DOI:10.5281/zenodo.4742786}
\httpB{http://www.nordita.org/~brandenb/projects/lucky/}\doiE{10.5281/zenodo.4742786}

\item[{23.}~]
Haugen, N. E. L., \Brandenburg, Sandin, C., \& Mattsson, L.\ybook{2021}
{Datasets for ``Spectral characterisation of inertial particle clustering in turbulence'' v2021.05.02}{Zenodo, DOI:10.5281/zenodo.4733175}
\httpB{http://www.nordita.org/~brandenb/projects/isoth_expwave/}\doiE{10.5281/zenodo.4733175}

\item[{22.}~]
He, Y., \Brandenburg, \& Sinha, A.\ybook{2021}
{Datasets for ``Spectrum of turbulence-sourced gravitational waves as a constraint on graviton mass'' v2021.04.06}{Zenodo, DOI:10.5281/zenodo.4666074}
\httpB{http://www.nordita.org/~brandenb/projects/GravitonGW/}\doiE{10.5281/zenodo.4666074}

\item[{21.}~]
\Brandenburg, Clarke, E., He, Y., \& Kahniashvili, T.\ybook{2021}
{Datasets for ``Can we observe the QCD phase transition-generated gravitational waves through pulsar timing arrays?'' v2021.02.24}{Zenodo, DOI:10.5281/zenodo.4560423}
\httpB{http://www.nordita.org/~brandenb/projects/GWfromQCD/}\doiE{10.5281/zenodo.4560423}

\item[{20.}~]
\Brandenburg, He, Y., Kahniashvili, T., Rheinhardt, M., \& Schober, J.\ybook{2021}
{Datasets for ``Gravitational waves from the chiral magnetic effect,''
v2021.01.18}{Zenodo, DOI:10.5281/zenodo.4448211}
\httpB{http://www.nordita.org/~brandenb/projects/GWfromCME/}\doiE{10.5281/zenodo.4448211}

\item[{19.}~]
Haugen, N. E. L., \& \Brandenburg\ybook{2020}
{Datasets for ``Hydrodynamic and hydromagnetic energy spectra from large eddy simulations,''
v2020.12.08}{DOI:10.5281/zenodo.4311391}
\httpB{http://www.nordita.org/~brandenb/projects/HydroMHDspec_from_LES/}\doiE{10.5281/zenodo.4311391}

\item[{18.}~]
Jakab, P., \& \Brandenburg\ybook{2020}
{Datasets for ``The effect of a dynamo-generated field on the Parker wind,''
v2020.11.22}{DOI:10.5281/zenodo.4284439}
\httpB{http://www.nordita.org/~brandenb/projects/StellarWind/}\doiE{10.5281/zenodo.4284439}

\item[{17.}~]
Kahniashvili, T., \Brandenburg, Gogoberidze, G., Mandal, S., \& Roper~Pol, A.\ybook{2020}
{Datasets for ``Circular polarization of gravitational waves from early-universe helical turbulence,''
v2020.11.07}{DOI:10.5281/zenodo.4256906}
\httpB{http://www.nordita.org/~brandenb/projects/CircPol/}\doiE{10.5281/zenodo.4256906}

\item[{16.}~]
\Brandenburg, \& Das, U.\ybook{2020}
{Datasets for ``Turbulent radiative diffusion and turbulent Newtonian cooling,'' v2020.10.13b}{DOI:10.5281/zenodo.4086046}
\httpB{http://www.nordita.org/~brandenb/projects/TurbNewtonianCooling/}\doiE{10.5281/zenodo.4086046}

\item[{15.}~]
Merski, M.\ybook{2020}
{Datasets for ``A simple model to predict future SARS-CoV-2 infections on a national level,''
v2020.10.11}{DOI:10.5281/zenodo.4256906}
\httpB{http://zenodo.org/record/4256906}\doiE{10.5281/zenodo.4256906}

\item[{14.}~]
\Brandenburg\ybook{2020}
{Datasets for ``Piecewise quadratic growth during the 2019 novel coronavirus epidemic,'' v2020.09.07}{DOI:10.5281/zenodo.4016941}
\httpB{http://www.nordita.org/~brandenb/projects/COVID-19/}\doiE{10.5281/zenodo.4016941}

\item[{13.}~]
\Brandenburg\ybook{2020}
{Datasets for ``Hall cascade with fractional magnetic helicity in neutron star crusts,'' v2020.07.20}{DOI:10.5281/zenodo.3951873}
\httpB{http://zenodo.org/record/3951873}\httpT{http://www.nordita.org/~brandenb/projects/HallCascade/}\doiE{10.5281/zenodo.3951873}

\item[{12.}~]
\Brandenburg, \& Furuya, R. S.\ybook{2020}
{Datasets for ``Application of a helicity proxy to edge-on galaxies,'' v2020.06.17}{DOI:10.5281/zenodo.3897954}
\httpB{http://zenodo.org/record/3897954}\httpT{http://www.nordita.org/~brandenb/projects/EBpol_EdgeOn/}\doiE{10.5281/zenodo.3897954}

\item[{11.}~]
Prabhu, A., \Brandenburg, K\"apyl\"a, M. J., \& Lagg, A.\ybook{2020}
{Datasets for ``Helicity proxies from linear polarisation of solar active regions,'' v2020.06.10}{DOI:10.5281/zenodo.3888575}
\httpB{http://zenodo.org/record/3888575}\doiE{10.5281/zenodo.3888575}

\item[{10.}~]
Pusztai, I., Juno, J., \Brandenburg, TenBarge, J. M., Hakim, A., Francisquez, M., \& Sundstr\"{o}m, A.\ybook{2020}
{Datasets for ``Dynamo in weakly collisional non-magnetized plasmas impeded by Landau damping of magnetic fields,'' v1}
{DOI:10.5281/zenodo.3886562}
\httpB{http://zenodo.org/record/3886562}\httpT{http://www.nordita.org/~brandenb/projects/kineticDynamo/}\doiE{10.5281/zenodo.3886562}

\item[{9.}~]
\Brandenburg, \& Br\"uggen, M.\ybook{2020}
{Datasets for ``Hemispheric handedness in the Galactic synchrotron polarization foreground,'' v2020.05.24}{DOI:10.5281/zenodo.3841900}
\httpB{http://zenodo.org/record/3841900}\httpT{http://www.nordita.org/~brandenb/projects/GalHemiHandedness/}\doiE{10.5281/zenodo.3841900}

\item[{8.}~]
Roper Pol, A., Mandal, S., \Brandenburg, Kahniashvili, T., \& Kosowsky, A.\ybook{2020}
{Datasets for ``Numerical Simulations of Gravitational Waves from Early-Universe Turbulence''' v2020.02.28}{DOI:10.5281/zenodo.3692072}
\httpB{http://zenodo.org/record/3692072}\doiE{10.5281/zenodo.3692072}

\item[{7.}~]
\Brandenburg, \& Scannapieco, E.\ybook{2019}
{Datasets for ``Magnetic helicity dissipation and production in an ideal MHD code,'' v2019.11.11}{DOI:10.5281/zenodo.3534739}
\httpB{http://zenodo.org/record/3534739}\doiE{10.5281/zenodo.3534739}

\item[{6.}~]
\Brandenburg, \& Chen, L.\ybook{2019}
{Datasets for ``The nature of mean-field generation in three classes of optimal dynamos,'' v2019.11.02}{DOI:10.5281/zenodo.3526056}
\httpB{http://zenodo.org/record/3526056}\doiE{10.5281/zenodo.3526056}

\item[{5.}~]
\Brandenburg\ybook{2019}
{Scientific usage of the Pencil Code, v2019.10.01}{DOI:10.5281/zenodo.3466444}
\adsB{2020zndo...3466444B}\httpT{http://zenodo.org/record/3466444}\httpT{http://www.nordita.org/~brandenb/pencil-code/PCSC/}\doiE{10.5281/zenodo.3466444}
%\httpB{http://zenodo.org/record/3466444}\httpT{http://www.nordita.org/~brandenb/pencil-code/PCSC/}\doiT{10.5281/zenodo.3466445}\doiT{10.5281/zenodo.3947506}\doiE{10.5281/zenodo.3466444} %contains older version

\item[{4.}~]
\Brandenburg, Kahniashvili, T., Mandal, S., Roper Pol, A., Tevzadze, A. G., \& Vachaspati, T.\ybook{2019}
{Datasets for ``Dynamo effect in decaying helical turbulence,'' v2019.07.21}
{DOI:10.5281/zenodo.3345134}
\httpB{http://zenodo.org/record/3345134}\httpT{http://www.nordita.org/~brandenb/projects/decdynamo/}\doiE{10.5281/zenodo.3345134}

\item[{3.}~]
Gosain, S., \& \Brandenburg\ybook{2019}
{Datasets for ``Spectral magnetic helicity of solar active regions between 2006 and 2017,'' v2019.07.16}{DOI:10.5281/zenodo.3338302}
\httpB{http://zenodo.org/record/3338302}\httpT{http://www.nordita.org/~brandenb/projects/Hinode/}\doiE{10.5281/zenodo.3338302}

\item[{2.}~]
Li, X.-Y., Svensson, G., \Brandenburg, \& Haugen, N. E. L.\ybook{2019}
{Datasets for ``Cloud droplet growth due to supersaturation fluctuations in stratiform clouds,'' v2019.01.11}{DOI:10.5281/zenodo.2538027}
\httpB{http://zenodo.org/record/2538027}\httpT{http://www.nordita.org/~brandenb/projects/supersat/}\doiE{10.5281/zenodo.2538027}

\item[{1.}~]
\Brandenburg, on behalf of the Pencil Code Collaboration\ybook{2018}
{Pencil Code v2018.12.16}{DOI:10.5281/zenodo.2315093 and 3961647}
%\httpB{http://zenodo.org/record/2315093}\doiE{10.5281/zenodo.2315093} %old record
%\adsB{2020zndo...3961647A}\httpT{http://zenodo.org/record/3961647}\doiE{10.5281/zenodo.3961647}
\adsB{2020zndo...3961647A}\httpT{http://zenodo.org/record/4553325}\doiE{10.5281/zenodo.4553325}

\end{itemize}

\subsection*{E. Other publications, public outreach, interviews, and other public appearances of the name}%E E E

\begin{itemize}

\item[{76.}~]
Catanzaro, M.\ybook{2024} %{Science}{February}{24}
{COVID-19 scientists who faced huge bills after speaking in webinars win in court}
{(\url{http://doi.org/10.1126/science.zns9u7b})}
\doiB{10.1126/science.zns9u7b}\ownE{D/2024/Catanzaro24_orig}
%https://www.science.org/content/article/covid-19-scientists-who-faced-huge-bills-after-speaking-webinars-win-court

\item[{75.}~]
Catanzaro, M.\yjour{2023}{Science}{381}{258}
{259}{Costly invite? Scientists hit with massive bills after speaking at COVID-19 {\em webinars}}
\doiB{10.1126/science.adj8536}\ownE{D/2023/Catanzaro23}
% https://www.science.org/content/article/costly-invite-scientists-hit-with-massive-bills-after-speaking-at-covid-19-conferences
%A version of this story appeared in Science, Vol 381, Issue 6655.

\item[{74.}~]
Gurgenidze, M., Clarke, E., Kahniashvili, T., \& \Brandenburg, A.\ybook{2023}
{Circularly Polarized Gravitational Waves from the Ealy Universe as a Probe of New Physics}
{AAS Meeting \#241, id.435.06}
{Am. Astron. Soc. Meet.~\#241, id.~435.06. Bull. Am. Astron. Soc., Vol.~55, No.~2 e-id 2023n2i435p06}
\adsS{2023AAS...24143506G}

\item[{73.}~]
Sun, G., Clarke, E., Kahniashvili, T., \& \Brandenburg, A.\ybook{2023}
{Chiral Magnetic Effect and Gravitational Waves in the View of Big Bang Nucleosynthesis}
{AAS Meeting \#241, id.435.06}
{Am. Astron. Soc. Meet.~\#241, id.~435.04. Bull. Am. Astron. Soc., Vol.~55, No.~2 e-id 2023n2i215p04}
\adsS{2023AAS...24121504S}

\item[{72.}~]
\Brandenburg\ygafd{2022}{116}{537}
{539}{Astrophysical Magnetic Fields: From Galaxies to the Early Universe}
\adsB{2015GApFD.109..615B}\doiT{10.1080/03091929.2022.2156188}\ownE{D/2022/Bran_BookRev22}

\item[{71.}~]
\Brandenburg, \& Hochberg, D.\yoleb{2022}{52}{1}
{2}{Introduction to Origins of Biological Homochirality}
\doiB{10.1007/s11084-022-09629-4}\adsT{2022OLEB...52....1B}\adsT{2022OLEB..tmp...17B}\ownE{D/2022/Bran+Hoch22}

\item[{70.}~]
Clarke, E., Kahniashvili, T., Stepp, J., \& \Brandenburg, A.\ybook{2022}
{Big Bang Nucleosynthesis Limits and Relic Gravitational Wave Detection Prospects}
{APS April Meeting 2022, abstract id.T14.003}
\adsS{2022APS..APRT14003C}

\item[{69.}~]
Stepp, J., Kahniashvili, T., Clarke, E., \& \Brandenburg\ybook{2022}
{Chiral Magnetic Fields and Gravitational Waves}
{Am. Astron. Soc. Meet.~\#240, id.~202.02. Bull. Am. Astron. Soc., Vol.~54, No.~6 e-id 2022n6i202p02}
\adsS{2022AAS...24020202S}

\item[{68.}~]
\Brandenburg\ybook{2022}
{Skumanich-55 revisited}
{Fifty Years of the Skumanich Relations, Proceedings of the conference held 8-11 March 2022 in Boulder, Colorado. Online at https://skumanich.wdrc.org/, id.53}
\adsS{2022fysr.confE..53B}

\item[{67.}~]
He, Y., Roper Pol, A., \& \Brandenburg\yprep{2024}
{Leading-order nonlinear gravitational waves from reheating magnetogeneses}
(\arxiv{2110.14456}\adsT{2021arXiv211014456H}\ownT{2021/He+RopP+Bran21})

\item[{66.}~]
Sinha, S., Gupta, O., Singh, V., Lekshmi, B., Nandy, D., Mitra, D., Chatterjee, S., Bhattacharya, S., Chatterjee, S., Srivastava, N., \& \Brandenburg\ybook{2021}
{A Comprehensive Analysis of Machine Learning Approaches for Solar Flare Prediction}
{AGU Fall Meeting 2021, id. NG45B-0546}
\adsS{2021AGUFMNG45B0546S}

\item[{65.}~]
Clarke, E., \Brandenburg, He, Y., \& Kahniashvili, T.\ybook{2021}
{Can We Observe QCD Phase Transition-Generated Gravitational Waves Through Pulsar Timing Arrays?}
{Am. Astron. Soc. Meet.~\#238, id.~230.06. Bull. Am. Astron. Soc., Vol.~53, No.~6 e-id 2021n6i230p06}
\adsS{2021AAS...23823006C}

\item[{64.}~]
He, Y., \Brandenburg, Kahniashvili, T., Rheinhardt, M., \& Schober, J.\ybook{2021}
{Relic Gravitational Waves From The Chiral Magnetic Effect}
{Am. Astron. Soc. Meet.~\#238, id.~230.05. Bull. Am. Astron. Soc., Vol.~53, No.~6 e-id 2021n6i230p05}
\adsS{2021AAS...23823005H}

\item[{63.}~]
Mtchedlidze, S., Dom{\'\i}nguez-Fern{\'a}ndez, P., Du, X., \Brandenburg, \& Kahniashvili, T.\ybook{2021}
{Primordial Magnetic Fields through Large Scale Structure}
{Am. Astron. Soc. Meet.~\#238, id.~230.05. Bull. Am. Astron. Soc., Vol.~53, No.~6 e-id 2021n6i109p09}
\adsS{2021AAS...23810909M}

\item[{62.}~]
\Brandenburg\ybook{2020}
{Magnetic Helicity: diagnostic signatures and effects}
{American Geophysical Union, Fall Meeting 2020, abstract \#NG011-02}
\adsS{2020AGUFMNG011..02B}

\item[{61.}~]
Shukurov, A., \Brandenburg, Brooke, J., Sokoloff, D., \& Tavakol, R.\yjourN{2020}{Astron. Geophys.}{61}{4.12}
{David Moss (1943-2020)}
\adsB{2020A\&G....61d4.12S}\doiT{10.1093/astrogeo/ataa050}\ownE{D/2020/DMoss43to20}

\item[{60.}~]
\Brandenburg, \& R\"udiger, G.\yanN{2020}{341}{365}
{Karl-Heinz R\"adler (1935--2020)}
\adsB{2020AN....341..365B}\doiT{10.1002/asna.202013814}\ownE{D/2020/KHRaedler35to20}

\item[{59.}~]
Bhat, P., Subramanian, K., \& \Brandenburg\yprep{2019}
{Efficient quasi-kinematic large-scale dynamo as the small-scale dynamo saturates}
(\arxiv{1905.08278}\adsT{2019arXiv190508278B}\httpT{http://norlx65.nordita.org/~brandenb/tmp/quasikinLSD/}\ownT{2019/Bhat+Subr+Bran19})

\item[{58.}~]
Pusztai, I., Sundstrom, A., \Brandenburg, Juno, J., Tenbarge, J. M., \& Hakim, A.\ybook{2019}
{Towards a fully kinetic dynamo simulation}
{APS Division of Plasma Physics Meeting 2019, abstract id.CP10.025}
\adsS{2019APS..DPPC10025P}

\item[{57.}~]
Mandal, S., \Brandenburg, Durrer, R., Kahniashvili, T., Roper Pol, A., Tevzadze, A., Vachaspati, T., \& Yin, W.\ybook{2019}
{The evolution of primordial magnetic fields due to magnetohydrodynamic turbulence, and their cosmological applications}
{APS April Meeting 2019, abstract id.B11.001}
\adsS{2019APS..APRB11001M}

\item[{56.}~]
\Brandenburg\ybook{2019}
{Learning about solar/stellar dynamo physics from the variability}
{Sol. Atmosph. Interplan. Env., http://shinecon.org/shine2019/ViewAbstract.php?idabs=526}
\adsB{2019shin.confE.220B}\httpE{http://shinecon.org/shine2019/ViewAbstract.php?idabs=526}

\item[{55.}~]
EPIC~206038483, the 60th planetary system confirmed with the
extended Kepler K2 mission (K2-60), adopted by Travis Metcalfe
for Axel Brandenburg on the occasion of his 60th anniversary
(\url{http://adoptastar.org/planetary-systems-k2/}\httpT{http://nonprofit.adoptastar.org/stars/206038483}\ownT{D/2019/206038483})

\item[{54.}~]
Giampapa, M. S., Cody, A. M., \Brandenburg, Skiff, B. A., \& Hall, J. C.\ybook{2017}
{The rotation and chromospheric activity of the solar-type stars in the open cluster M67}
{(\url{http://norlx51.nordita.org/~brandenb/tmp/m67/})}

\item[{53.}~]
Li, X.-Y., \Brandenburg, Svensson, G., Haugen, N., \& Rogachevskii, I.\ybook{2017}
{Turbulence effect on coagulational growth of cloud droplets}
{American Geophysical Union, Fall Meeting 2017, abstract \#A11I-1984}
\adsS{2017AGUFM.A11I1984L}

\item[{52.}~]
Li, X.-Y., \Brandenburg, Svensson, G., Haugen, N., \& Rogachevskii, I.\ybook{2017}
{Turbulence effect on coagulational growth of cloud droplets}
{APS Division of Fluid Dynamics (Fall) 2017, abstract id.L16.004}
\adsS{2017APS..DFDL16004L}

\item[{51.}~]
\Brandenburg, Petrie, G., \& Singh, N.\ybook{2016}
{Two-scale Analysis of Solar Magnetic Helicity}
{SDO 2016: Unraveling the Sun's Complexity, Proceedings of the
conference held 17-21 October, 2016 in Burlington, VT. Online at
\url{http://sdo-2016.lws-sdo-workshops.org/}, id.110}
\adsS{2016usc..confE.110B}

\item[{50.}~]
Yokoi, N., \& \Brandenburg\ybook{2016}
{Vortex generation due to inhomogeneous turbulent helicity}
{EGU General Assembly 2016, 17-22 April, Vienna Austria, p.8135}
\adsB{2016EGUGA..18.8135Y}\ownE{D/2016/Yoko_Bran16}

\item[{49.}~]
Cauzzi, G., Shchukina, N., Kosovichev, A., Bianda, M., \Brandenburg, Chou, D.-Y., Dasso, S., Ding, M.-D., Jefferies, S., Krivova, N., Kuznetsov, V. D., \& Moreno-Insertis, F.\yjour{2016}{Trans. IAU}{29A}{278}
{299}{Commission 12: Solar Radiation and Structure}
%\adsB{2016IAUTA..29..278C}\ownE{D/2016/Cauzzi_etal16}
\adsS{2016IAUTA..29..278C}

\item[{48.}~]
{\it Boosting solar physics}. Article by Travis Metcalfe on
DKIST/NSO with interview of Axel Brandenburg
(\url{http://www.boulderweekly.com/features/lab-notes/boosting-solar-physics})

\item[{47.}~]
\Brandenburg, Zhang, H., \& Sokoloff, D.\ybook{2016}
{The magnetic helicity spectrum from solar vector magnetograms}
{Am. Astron. Soc., SPD mtg 47, id.\ 201.03}
\adsS{2016SPD....4720103B}

\item[{46.}~]
Anders, E. H., Brown, B., \Brandenburg, \& Rast, M.\ybook{2016}
{The structure and evolution of boundary layers in stratified convection}
{Am. Astron. Soc., SPD mtg 47, id.\ 7.12}
\adsS{2016SPD....47.0712A}

\item[{45.}~]
Singh, N., Raichur, H., \& \Brandenburg\ybook{2016}
{High-wavenumber solar f-mode strengthening prior to active region formation}
{Am. Astron. Soc., SPD mtg 47, id.\ 7.11}
\adsS{2016SPD....47.0711S}

\item[{44.}~]
Cataldi, G., Brandeker, A., Thebault, P., Singer, K., Ahmed, E., \Brandenburg, Olofsson, G., \& de Vries, B.\yprocN{2015}{49}
{Characterizing the three-dimensional ozone distribution of a tidally locked Earth-like planet}
{Pathways Towards Habitable Planets, Proceedings of a conference held 13-17 July}
{Dawn Gelino}{Bern, Switzerland}
%\url{http://pathways2015.sciencesconf.org/program}
\adsS{2015pthp.confE..49C}

\item[{43.}~]
Yokoi, N., \& \Brandenburg\ybook{2015}
{Large-scale flow generation due to inhomogeneous turbulent helicity}
{American Geophysical Union, Fall Meeting 2015, abstract NG33A-1847}
\adsS{2015AGUFMNG33A1847Y}

\item[{42.}~]
\Brandenburg\ygafd{2015}{109}{615}
{616}{Magnetohydrodynamics of the Sun. By E.R. Priest, Cambridge Univ. Press}
\adsB{2015GApFD.109..615B}\doiT{10.1080/03091929.2015.1088694}\ownE{D/2015/Bran_BookRev15}

\item[{41.}~]
Kosovichev, A., Cauzzi, G., Pillet, V. M., Asplund, M., \Brandenburg, Chou, D.-Y., Christensen-Dalsgaard, J., Gan, W., Kuznetsov, V. D., Rovira, M. G., Shchukina, N., Venkatakrishnan, P.\yjour{2015}{Trans. IAU {\bf T28}}{10}{109}
{111}{Division II: Commission 12: Solar Radiation and Structure}
\adsB{2015IAUTB..28..109K}\ownE{D/2015/Kosovichev_etal15}

\item[{40.}~]
Kemel, K., \Brandenburg, Kleeorin, N., Mitra, D., \& Rogachevskii, I.\yproc{2014}{293}
{313}{Active region formation through the negative effective magnetic pressure instability}
{Solar Dynamics and Magnetism from the Interior to the Atmosphere}
{N. N. Mansour, A. G. Kosovichev, R. Komm, \& D. Longcope}
{Springer--New York}
\arxivB{1203.1232}\adsT{2013SoPh..287..293K}\doiT{10.1007/978-1-4899-8005-2_19}\ownE{2013/Kem+Bran+Klee+Mitr+Roga13}
(Reprint from journal publication listed above as A.292)

\item[{39.}~]
\Brandenburg\yjour{2014}{Fysikaktuellt}{2014-1}{22}
{23}{S\"okandet efter en ny teori f\"or solfl\"ackar}\\
\url{http://www.fysikersamfundet.se/Fysikaktuellt/2014_1.pdf}

\item[{38.}~]
\Brandenburg, \& Lazarian, A.\yproc{2014}{87}
{124}{Astrophysical hydromagnetic turbulence}
{Microphysics of Cosmic Plasmas, Space Sciences Series of ISSI, Volume 47}
{A. Balogh, A. Bykov, P. Cargill, R. Dendy, T. Dudok de Wit, \& J. Raymond}
{Springer Science+Business Media New York}
\arxivB{1307.5496}\adsT{2014mcp..book...87B}\doiT{10.1007/978-1-4899-7413-6_5}\ownE{D/2014/Bran+Laza14}
(Reprint from journal publication listed above as A.297)

\item[{37.}~]
Bykov, A. M., \Brandenburg, Malkov, M. A., \& Osipov, S. M.\yproc{2014}{125}
{156}{Microphysics of cosmic ray driven plasma instabilities}
{Microphysics of Cosmic Plasmas, Space Sciences Series of ISSI, Volume 47}
{A. Balogh, A. Bykov, P. Cargill, R. Dendy, T. Dudok de Wit, \& J. Raymond}
{Springer Science+Business Media New York}
\arxivB{1304.7081}\adsT{2014mcp..book..125B}\doiT{10.1007/978-1-4899-7413-6_6}\ownE{D/2014/Bykov_etal14}
(Reprint from journal publication listed above as A.296)

\item[36.]
\Brandenburg, \& Rogachevskii, I.\ygafd{2013}{107}{1}
{2}{Introduction to Special Issue: From Mean-Field to Large-Scale Dynamos}
\doiB{10.1080/03091929.2013.764770}\ownE{2013/Bran+Roga13}

\item[35.]
``Towards an understanding of the Sun's Butterfly Diagram,''
public outreach article by Sabine Hossenfelder on her blog
\url{http://backreaction.blogspot.jp/} of 8 October 2012.\\
\url{http://backreaction.blogspot.jp/2012/10/towards-understanding-of-suns-butterfly.html}

\item[34.]
Kosovichev, A., Lundstedt H., \& \Brandenburg\ypsN{2012}{86}{010201}
{Special issue on current research in astrophysical magnetism}
\adsB{2012PhyS...86a0201K}\doiT{10.1088/0031-8949/86/01/010201}\ownE{D/2012/Koso+Lund+Bran12}

\item[33.]
Chian, A. C.-L., \Brandenburg, Proctor, M. R. E., \& Rempel, E. L.\yjourN{2012}{Geophys. Res. Abstracts}{14}{2444}
{On-off intermittency and Lagrangian coherent structures in solar dynamo}
\adsB{2012EGUGA..14.2444C}\ownE{D/2012/Chian_etal12}

\item[32.]
Kosovichev, A., Cauzzi, G., Pillet, V. M., Asplund, M., \Brandenburg, Chou, D.-Y., Christensen-Dalsgaard, J., Gan, W., Kuznetsov, V. D., Rovira, M. G., Shchukina, N., Venkatakrishnan, P.\yjour{2012}{Transactions IAU}{7}{81}
{94}{Commission 12: Solar Radiation and Structure}
\adsB{2012IAUTB..28...81K}\ownE{D/2012/Kosovichev_etal12}

\item[31.]
\Brandenburg, \& Dobler, W.: 2010,
Pencil Code: Finite-difference Code for Compressible Hydrodynamic Flows,
Astrophysics Source Code Library, record ascl:1010.060
\adsS{2010ascl.soft10060B}

\item[30.]
``Cycles of the Sun,''
Interview with Axel Brandenburg by British Publishers, January 2010,
also published in EU Research, ``The latest research from FP7,''
pp.\ 114-115, June 2010, www.euresearch.com\\
\url{http://www.nordita.org/~brandenb/AstroDyn/material/Solar_Activity_10.pdf}%\ownT{D/2010/Solar_Activity_10})

\item[29.]
Plasson, R., Bersini, H., \& \Brandenburg\yoleb{2009}{39}{263}
{264}{Emergence of protometabolisms and the self-organization of
non-equilibrium reaction networks}
\ownS{D/2009/Plas+Bers+Bran09_abstr}

\item[28.]
\Brandenburg, 2009, Res.\ Astron.\ Astrophys.\ {\bf 9},
``A record low in solar activity inspires theorists about grand minima''
\ownB{D/2009/Bran09_recordlow}\httpE{http://www.raa-journal.org/docs/newsandviews/0909_newsandviews_abrandenburg.html}\\
\url{http://www.raa-journal.org/docs/newsandviews/0909_newsandviews_abrandenburg.html}

\item[27.]
Kosovichev, A.G., Arlt, R., Bonanno, A., \Brandenburg, Brun,
A.S., Busse, F., Dikpati, M., Hill, F., Gilman, P.A., Nordlund, A.,
R\"udiger, G., Stein, R.F., Sekii, T., Stenflo, J.O., Ulrich, R.K.,
Zhao, J.\yproc{2009}{1}
{8}{Solar dynamo and magnetic self-organization}
{Astro2010: The Astronomy and Astrophysics Decadal Survey}
{Science White Papers}
{No.\ 160}
\adsB{2009astro2010S.160K}\ownE{D/2009/Kosovichev_etal09}

\item[26.]
Green, C., \Brandenburg, \& Kosovichev, A.\ybook{2008}
{Non-linear Modeling of Wavelike Behaviour of Supergranulation}
{Am. Geophys. Union, Spring Meeting 2008, abstract SP21A-04}
\adsS{2008AGUSMSP21A..04G}

\item[25.]
Interview with Axel Brandenburg by the Swedish Research Council
(Siv Engelmark Cederborg), 2008
\httpS{http://www.vr.se/huvudmeny/internationellasamarbeten/europeanresearchcouncilerc/ercadvancedinvestigatorgrant/beviljadebidragtillforskareisverige2008/axelbrandenburg.4.427cb4d511c4bb6e386800019637.html}

\item[24.]
Plasson, R., Bersini, H., \& \Brandenburg\ypreprint{2008}
{Decomposition of complex reaction networks into reactons}
(\arxiv{0803.1385}\adsT{2008arXiv0803.1385P}\ownT{2008/Plasson_etal08})

\item[23.]
Podcast with Simon Mitton interviewing Axel Brandenburg: 2008,
{\it Is All Life Left-Handed?}
\httpS{http://www.astrobio.net/amee/summer_2008/Radio/radio.php}

\item[22.]
Publication based on text by \Brandenburg, \& K\"apyl\"a, P. J.\ybook{2007}
{Uusimmat turbulenssilaskut osoittavat pullonkaulaefektin olevan aito}
{CSC Uutiset}
\httpS{http://www.csc.fi/csc/ajankohtaista/uutiset/pullonkaulaefekti-2007-08-27}

\item[21.]
\Brandenburg, Lehto, H. J., \& Lehto, K. M.\yija{2007}{6}{80}
{80}{Origin of homochirality in an early peptide world}
\ownS{D/2007/Bran+Lehtos_abs07}

\item[20.]
\Brandenburg, Andersen, A. C., \& Multam\"aki, T.\yjour{2006}{BioZoom}{9/2}{8}
{11}{Homochirality and the moment when life came about}
\ownB{D/2006/Bran+And+Mult06b}\httpE{http://www.biokemi.org/biozoom/2006_3/bz_0306c.htm}

\item[19.]
\Brandenburg, Andersen, A. C., \& Multam\"aki, T.\yjour{2006}{Gamma}{142}{22}
{31}{Homochirality -- The problem of left handed amino acids}
\ownS{D/2006/Bran+And+Mult06}

\item[18.]
Poole, A. M., Hode, T., \Brandenburg, Hjalmarson, \AA.,
\& Holm, N. G.\yjour{2006}{Astrobiol.}{6}{815}
{818}{Life up North}
\ownS{D/2006/Poole_etal06}

\item[17.]
Andersen, A. C., \Brandenburg,
\& Multam\"aki, T.\yjourS{2005}{Kvant}{9/16}{18}
{21}{Er det en naturlov at aminosyrer er venstredrejede?}
\ownS{D/2005/And+Bran05_kvant}

\item[16.]
Andersen, A. C., \& \Brandenburg\yjour{2005}{BioZoom}{8/2}{7}
{8}{Nordisk astrobiologi}
\ownB{D/2005/And+Bran05_BioZoom}\httpE{http://www.biokemi.org/biozoom/2005_2/bz_0205b.htm}

\item[15.]
Andersen, A. C., \& \Brandenburg\yija{2005}{4}{1}
{2}{Editorial: astrobiological problems for physicists and biologists}
\adsB{2005IJAsB...4....1A}\ownE{D/2005/And+Bran05}

\item[14.]
\Brandenburg\yjfm{2004}{503}{378}
{379}{Magnetohydrodynamic Turbulence. By Dieter Biskamp. Cambridge
University Press, 2003. 310 pp}
\adsB{2004JFM...503..378B}\ownE{D/2004/Bran_BookRev04}

\item[13.]
\Brandenburg, Andersen, A. C., H\"ofner, S., \& Nilsson, M.\yjour{2004}
{Int. J. Astrobio. Suppl.}{3}{106}
{106}{Homochiral growth through enantiomeric cross-inhibition}

\item[12.]
\Brandenburg, Andersen, A. C., H\"ofner, S., \& Nilsson, M.\yjour{2004}
{Geochimica et Cosmochimica Acta}{68}{A792}
{A792}{Homochiral growth through enantiomeric cross-inhibition}

\item[11.]
von Rekowski, B., \& \Brandenburg\yan{2003}{324}{68}
{68}{Outflows and accretion in a protostellar star-disc system}
\adsS{2003ANS...324...68V}

\item[10.]
\Brandenburg\ybook{2002}
{The solar dynamo: worrying about magnetic helicity}
{Presented at the KITP Conference: Observational Challenges for the Next
Decade of Solar Magnetohydrodynamics, Jan 16, 2002, Kavli Institute for
Theoretical Physics, University of California, Santa Barbara}
\ads{2002ocnd.confE..23B}

\item[9.]
\Brandenburg \& Boldyrev, S. H.\yaas{2001}{199}{149.01}
{Burgers Turbulence and the Problem of Star Formation}
\adsS{2001AAS...19914901B}

\item[8.]
\Brandenburg \& Boldyrev, S. H.\yaps{2000}{42}{BP1.059}
{Small-scale kinematic dynamo with helicity}
\adsS{2000APS..DPPBP1059B}

\item[7.]
Saar, S. H. \& \Brandenburg\yaas{1998}{193}{4404}
{Time evolution of the magnetic activity cycle period:
results for an expanded stellar sample}
\adsS{1998AAS...193.4404S}

\item[6.]
Stein, R. F., Bercik, D. J., \Brandenburg, Georgobiani, D.,
Nordlund, \AA.\yjour{1997}{Bull. American Astron. Soc.}{74}{17}
{17}{Solar Magneto-Convection}
\adsS{1997AAS...191.7417S}

\item[5.]
Vishniac, E. T., \& \Brandenburg\yaas{1995}{187}{10409}
{An incoherent alpha--Omega dynamo mechanism for accretion disks}
\adsS{1995astro.ph.10038V}

\item[4.]
\Brandenburg, Tuominen, I., \& Ruokolainen, J.\yjour{1991}{/csc/news}{3}{3}
{5}{Simulating solar hydromagnetism}

\item[3.]
\Brandenburg, Krause, F., Moss, D., \& Tuominen, I.\yjourS{1991}
{Astron. Ges. Abstr. Ser.}{6}{32}
{32}{Can the Lorentz force accelerate magnetic field expansion?}
\adsB{1991AGAb....6...32B}\ownE{D/1991/Bran+Krau+Moss+Tuom91}

\item[2.]
\Brandenburg\yjour{1991}{Nachr.\ Akad.\ Wiss.\ G\"ottingen
II.\ Mathem.\ Phys.\ Klasse}{2}{16}
{17}{Challenges for solar dynamo theory:
alpha effect, differential rotation and stability}
\adsS{1991NAWG.1991...26B}

\item[1.]
Donner, K. J., \Brandenburg\yprocN{1990}{16}
{Structure of dynamo generated galactic magnetic fields}
{Proc. Nordic-Baltic Astronomy Meeting}
{C.-I. Lagerkvist, D. Kiselman \& M. Lindgren}
{Uppsala Univ. (Sweden), Astronomiska Observatoriet}
\adsS{1990apsu.conf...16D}
 
%\item
%\Brandenburg, Krause, F., Nordlund, \AA., Ruzmaikin, A. A.,
%Stein, R.F., Tuominen, I.\sapj{1992}
%{On the magnetic fluctuations produced by a large scale magnetic field}

%Popular articles

\end{itemize}
\verb|$Id: curri.tex,v 1.2714 2025/01/31 12:43:18 brandenb Exp $|

%\input meetings_short.tex
\end{document}
